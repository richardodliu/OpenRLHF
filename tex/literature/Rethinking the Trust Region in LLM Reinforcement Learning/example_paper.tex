%%%%%%%% ICML 2026 EXAMPLE LATEX SUBMISSION FILE %%%%%%%%%%%%%%%%%

\documentclass{article}

% Recommended, but optional, packages for figures and better typesetting:
\usepackage{microtype}
\usepackage{graphicx}
\usepackage{subcaption}
\usepackage{booktabs} % for professional tables

% hyperref makes hyperlinks in the resulting PDF.
% If your build breaks (sometimes temporarily if a hyperlink spans a page)
% please comment out the following usepackage line and replace
% \usepackage{icml2026} with \usepackage[nohyperref]{icml2026} above.
\usepackage{hyperref}

% \usepackage{amsmath, amssymb, graphicx, algorithm, algpseudocode}  

% Attempt to make hyperref and algorithmic work together better:
\newcommand{\theHalgorithm}{\arabic{algorithm}}

% Use the following line for the initial blind version submitted for review:
% \usepackage{icml2026}

% For preprint, use
\usepackage[preprint]{icml2026}

% If accepted, instead use the following line for the camera-ready submission:
% \usepackage[accepted]{icml2026}
\usepackage{multirow}

\usepackage{amsmath}
\usepackage{amssymb}
\usepackage{mathtools}
\usepackage{amsthm}
\usepackage{enumitem}

% if you use cleveref..
\usepackage[capitalize,noabbrev]{cleveref}
\usepackage[dvipsnames]{xcolor}
\usepackage[absolute,overlay]{textpos}
\usepackage{fontawesome}


%%%%%%%%%%%%%%%%%%%%%%%%%%%%%%%%
% THEOREMS
%%%%%%%%%%%%%%%%%%%%%%%%%%%%%%%%
\theoremstyle{plain}
\newtheorem{theorem}{Theorem}[section]
\newtheorem{proposition}[theorem]{Proposition}
\newtheorem{lemma}[theorem]{Lemma}
\newtheorem{corollary}[theorem]{Corollary}
\theoremstyle{definition}
\newtheorem{definition}[theorem]{Definition}
\newtheorem{assumption}[theorem]{Assumption}
\theoremstyle{remark}
\newtheorem{remark}[theorem]{Remark}

% Todonotes is useful during development; simply uncomment the next line
%    and comment out the line below the next line to turn off comments
%\usepackage[disable,textsize=tiny]{todonotes}
\usepackage[textsize=tiny]{todonotes}
% \usepackage{listings}
% \lstset{
% basicstyle=\small\ttfamily,
% columns=flexible,
% breaklines=true
% }
% \lstset{escapeinside={<@}{@>}}



%%%%%%%%%%%%%%%%%%%%%%%%%%%%%%%
%%%%%%%% AI box new 
\usepackage[most]{tcolorbox}
\tcbset{
  aibox/.style={
    enhanced,
    breakable,
    width=\linewidth,
    top=7pt,
    bottom=2pt,
    colback=blue!6!white,
    colframe=black,
    colbacktitle=black,
    coltitle=white,
    attach boxed title to top left={yshift=-0.1in,xshift=0.15in},
    boxed title style={boxrule=0pt, colframe=white},
  },
  aiboxgrey/.style={
    enhanced,
    breakable,
    width=\linewidth,
    top=7pt,
    bottom=2pt,
    colback=gray!5!white, % Changed from RGB to grayscale specification
    colframe=black,
    colbacktitle=black,
    coltitle=white,
    attach boxed title to top left={yshift=-0.1in,xshift=0.15in},
    boxed title style={boxrule=0pt, colframe=white},
  }
}

\newtcolorbox{AIbox}[2][]{aibox, title={#2}, #1}
\newtcolorbox{AIboxgrey}[2][]{aiboxgrey, title={#2}, #1}

%%%%%%%%%% for listings %%%%%%%%%%%%%%
\usepackage{listings}
\lstset{
    basicstyle = \ttfamily\small, % 等宽字体,小号字
    % basicstyle = \ttfamily, % 等宽字体
    breaklines = true,            % 启用自动换行
    breakatwhitespace = true,    % 允许在任意位置换行(而不仅限空格处)
    breakindent=0pt,
    xleftmargin = 0pt,    % Key: removes left indentation of the listing block
    % frame = single,               % 添加边框
    % backgroundcolor = \color{gray!10}, % 背景色
    % numbers = left,               % 显示行号
    % numberstyle = \tiny\color{gray}, % 行号样式
    % postbreak=\raisebox{0pt}[0pt][0pt]{\ensuremath{\color{blue}\hookrightarrow\space}}, % 在换行处添加一个红色右箭头并保留一个空格
    % escapeinside = {@(}{)}        % 允许在代码中嵌入 LaTeX 命令(如数学公式)
    escapeinside={<@}{@>}
}
%%%%%%%%%% for listings %%%%%%%%%%%%%%




\DeclareMathOperator*{\argmax}{arg\,max}  
\DeclareMathOperator*{\rolloutpi}{{\textcolor{black}{\mu}}}
\DeclareMathOperator*{\trainerpi}{{\textcolor{black}{\pi}}}
\DeclareMathOperator*{\redmu}{{\textcolor{red}{\mu}}}
\DeclareMathOperator*{\bluepi}{{\textcolor{red}{\pi}}}


%%%%%% add by xiangxin for highlighting comment 
% \newcommand{\xiangxin}[1]{{{\textcolor{red}{[Xiangxin: #1]}}}}





%%%%%%% added by xiangxin for compressing spaces
% Spacing adjustments
% \setlength{\textfloatsep}{5pt}
% \setlength\floatsep{3pt}
% \setlength\intextsep{1pt}
% \setlength{\abovecaptionskip}{0.1em}
% \setlength{\belowcaptionskip}{0.1em}
% \setlength{\parskip}{0.1em}
% \usepackage[compact]{titlesec}
% \usepackage{sidecap}
% \titlespacing*{\section}{0pt}{*0.1}{*0.1}
% \titlespacing*{\subsection}{0pt}{*0.1}{*0.1}
% \titlespacing*{\subsubsection}{0pt}{*0.1}{*0.1}
% \allowdisplaybreaks







% The \icmltitle you define below is probably too long as a header.
% Therefore, a short form for the running title is supplied here:
\icmltitlerunning{Rethinking the Trust Region in LLM Reinforcement Learning}

\begin{document}

\twocolumn[
  \icmltitle{Rethinking the Trust Region in LLM Reinforcement Learning}

  % It is OKAY to include author information, even for blind submissions: the
  % style file will automatically remove it for you unless you've provided
  % the [accepted] option to the icml2026 package.

  % List of affiliations: The first argument should be a (short) identifier you
  % will use later to specify author affiliations Academic affiliations
  % should list Department, University, City, Region, Country Industry
  % affiliations should list Company, City, Region, Country

  % You can specify symbols, otherwise they are numbered in order. Ideally, you
  % should not use this facility. Affiliations will be numbered in order of
  % appearance and this is the preferred way.
  \icmlsetsymbol{equal}{*}
  \icmlsetsymbol{prior}{$\dagger$}

  \begin{icmlauthorlist}
    \icmlauthor{Penghui Qi}{equal,sea,nus}
    \icmlauthor{Xiangxin Zhou}{equal,sea}
    \icmlauthor{Zichen Liu}{nus}
    \icmlauthor{Tianyu Pang}{sea}
    \icmlauthor{Chao Du}{sea}
    \icmlauthor{Min Lin}{sea}
    \icmlauthor{Wee Sun Lee}{nus}
    %\icmlauthor{}{sch}
    % \icmlauthor{Firstname8 Lastname8}{sch}
    % \icmlauthor{Firstname8 Lastname8}{yyy,comp}
    %\icmlauthor{}{sch}
    %\icmlauthor{}{sch}
  \end{icmlauthorlist}

  \icmlaffiliation{sea}{Sea AI Lab, Singapore}
  \icmlaffiliation{nus}{School of Computing, National University of Singapore}
  % \icmlaffiliation{prior}{Work done in Sea AI Lab}
  
  % \icmlaffiliation{sch}{School of ZZZ, Institute of WWW, Location, Country}

  \icmlcorrespondingauthor{Wee Sun Lee}{leews@comp.nus.edu.sg}
  \icmlcorrespondingauthor{Penghui Qi}{penghuiq@comp.nus.edu.sg}

  % You may provide any keywords that you find helpful for describing your
  % paper; these are used to populate the "keywords" metadata in the PDF but
  % will not be shown in the document
  \icmlkeywords{LLMs, Reinforcement Learning, Trust Region, Training Stability, Training Efficiency}

  \vskip 0.3in

  \noindent\begin{minipage}{\textwidth}
  \vspace{8.4cm}
  \end{minipage}
]

% this must go after the closing bracket ] following \twocolumn[ ...

% This command actually creates the footnote in the first column listing the
% affiliations and the copyright notice. The command takes one argument, which
% is text to display at the start of the footnote. The \icmlEqualContribution
% command is standard text for equal contribution. Remove it (just {}) if you
% do not need this facility.

% Use ONE of the following lines. DO NOT remove the command.
% If you have no special notice, KEEP empty braces:
% \printAffiliationsAndNotice{}  % no special notice (required even if empty)
% Or, if applicable, use the standard equal contribution text:
\printAffiliationsAndNotice{\icmlEqualContribution}

\begin{textblock*}{17cm}(2.1cm,6cm)
    \centering
    \vspace{-0.6cm}
    \begin{center}
        \faGithub~\url{https://github.com/sail-sg/Stable-RL}
    \end{center}
    \vspace{-0.1cm}
    \includegraphics[width=0.85\linewidth]{figs/ppo_vs_dppo.pdf}
    \captionof{figure}{Comparison of \textcolor{red}{PPO} and the proposed \textcolor{blue}{DPPO} (the Binary-TV variant in \Cref{sec:method_binary}). (\textbf{Left}) The surrogate objective and corresponding masks for PPO and DPPO. PPO (and variants like GRPO) employs a heuristic mask based on the probability ratio, which \textbf{over-penalizes low-probability tokens} and \textbf{under-penalizes high-probability ones} (\Cref{sec:method_limitations}). In contrast, DPPO utilizes a more principled mask based on a direct approximation of policy divergence (e.g., Total Variation), ensuring updates stay within a theoretically grounded trust region (\Cref{sec:llm_tr}). (\textbf{Right}) Experimental results on the AIME24 using Qwen3-30B-A3B-Base. DPPO significantly outperforms GRPO baselines, achieving superior training efficiency and stability even without rollout routing replay (R3) (\Cref{sec:scaling_exp}).}
    \label{fig:ppo_vs_dppo}
\end{textblock*}


\begin{abstract}  
Reinforcement learning (RL) has become a cornerstone for fine-tuning Large Language Models (LLMs), with Proximal Policy Optimization (PPO) serving as the de facto standard algorithm. Despite its ubiquity, we argue that the core ratio clipping mechanism in PPO is structurally ill-suited for the large vocabularies inherent to LLMs. PPO constrains policy updates based on the probability ratio of sampled tokens, which serves as a noisy single-sample Monte Carlo estimate of the true policy divergence. This creates a sub-optimal learning dynamic: updates to low-probability tokens are aggressively over-penalized, while potentially catastrophic shifts in high-probability tokens are under-constrained, leading to training inefficiency and instability. To address this, we propose \textbf{Divergence Proximal Policy Optimization (DPPO)}, which substitutes heuristic clipping with a more principled constraint based on a direct estimate of policy divergence (e.g., Total Variation or KL). To avoid huge memory footprint, we introduce the efficient Binary and Top-K approximations to capture the essential divergence with negligible overhead. Extensive empirical evaluations demonstrate that DPPO achieves superior training \textbf{stability} and \textbf{efficiency} compared to existing methods, offering a more robust foundation for RL-based LLM fine-tuning. 
\end{abstract}

\begin{figure*}[h]
\centering
\includegraphics[width=0.99\textwidth]{figures/main_results.pdf}
\caption{Performance comparison of \model{} and existing open-weights and close-weights$^{\dagger}$ models across benchmarks.}
\label{fig:ring-lite-performance}
\end{figure*}

\section{Introduction}
\label{sec:intro}

Artificial intelligence is undergoing a pivotal transition: Large Language Models (LLMs) are advancing beyond static corpora of human knowledge, becoming dynamic processors that transform information into actionable insights and understanding~\citep{kimiteam2025kimik2openagentic,deepseekai2025deepseekr1}. This progression towards more general intelligence is empirically validated by their core capability – complex, adaptive problem-solving. 
Recent breakthroughs in solving high-difficulty human competition problems provide concrete evidence of significantly advanced reasoning abilities in large language models. For instance, models~\citep{gpt5, qwen3max} have achieved 100\% accuracy on the AIME-2025~\citep{aime} and HMMT-2025~\footnote{https://www.hmmt.org/www/archive/problems}, and reached medal-level performance at the International Mathematical Olympiad (IMO)~\citep{gpt5}—a hallmark of sophisticated human intellect. This evolution beyond static knowledge repositories is driven by training on trillions of tokens across diverse domains, coupled with reinforcement learning-optimized reasoning techniques~\citep{openai2024openaio1card, deepseekai2025deepseekr1} that enable models to dynamically scale their capabilities with thinking effort, pointing toward higher levels of general intelligence.

While related work~\citep{deepseekai2025deepseekr1,glm46} has made valuable contributions to the open-source community, the frontier of trillion-parameter thinking models remains uncharted territory. Scaling to this level introduces formidable challenges, such as severe training instability and prohibitive computational costs. In this work, we introduce \model{}, a novel Mixture-of-Experts (MoE) thinking model scaled to unprecedented size—and demonstrate breakthrough methodologies for efficient trillion-parameter training. By solving fundamental stability and efficiency challenges at this scale, we enable robust large-scale reasoning training while providing extensive implementation insights.


Our \model{}, the first open-source reasoning model with one trillion total parameters, is built upon the Ling 2.0~\citep{lingv2} architecture and trained from the Ling-1T-base. 
With approximately 50 billion activated parameters per token, \model{} achieves state-of-the-art performance across multiple challenging benchmarks—despite relying solely on natural language reasoning capabilities. It significantly outperforms existing open-source models, achieving scores of 93.4 on AIME-2025, 86.72 on HMMT-2025, 2088 on CodeForces, and 55.94 on ARC-AGI-v1. Remarkably, in the IMO-2025 evaluation within AWorld~\footnote{https://github.com/inclusionAI/AWorld}, \model{} achieved a silver medal-level result by correctly solving four problems and partially proving Problem 2, all within a single submission, and without relying on code generation or external symbolic solvers.
Realizing this breakthrough required addressing fundamental challenges in trillion-scale RL training. We pioneered three interconnected innovations:
\begin{itemize}
    \item \textbf{IcePop} eliminates catastrophic training-inference misalignment in RL training by clipping excessive-discrepancy tokens. This selective correction pops out unstable contributions while preserving efficient updates, thereby stabilizing training without slowing inference.
    \item \textbf{C3PO++} introduces a budget-controlled rollout scheduling mechanism that eliminates rollout-stage bottlenecks. Thus, it avoids inefficient single-pass processing of oversized sequences, reducing computational overhead while enabling their efficient reuse through batched continuation.
    \item \textbf{ASystem} is a high-performance reinforcement learning (RL) framework designed for large-scale asynchronous training. It adopts a SingleController + SPMD (Single Program, Multiple Data) to enable fully asynchronous operations, multi-phase masking acceleration, and efficient data packing/sharding. 
\end{itemize}

The structure of this paper is organized as follows: Section~\ref{sec:method} describes our comprehensive training methodology, which includes Long Chain-of-Thought Supervised Fine-Tuning (Long-CoT SFT) and large-scale reinforcement learning (RL), including our key algorithmic contributions, IcePop and C3PO++, as well as the underlying training framework, Asystem. Finally, Section~\ref{sec:eval} presents a thorough evaluation of our model's performance against leading open-weights and closed-weights models on established benchmarks.


\section{Background}  
\label{sec:background}  

% In this section, we briefly review the standard trust region approach in classical RL paradigm.
  
\subsection{Policy Performance Difference}  
  
We begin with the standard formulation of a Markov Decision Process (MDP), defined by the tuple $\mathcal{M} = (\mathcal{S}, \mathcal{A}, P, r, \rho_0, \gamma)$, which includes the state space $\mathcal{S}$, action space $\mathcal{A}$, transition dynamics $P(s'|s,a)$, reward function $r(s,a)$, initial state distribution $\rho_0(s)$, and a discount factor $\gamma \in [0,1]$. A stochastic policy $\pi(a | s)$ generates trajectories $\tau = (s_0, a_0, r_0, s_1, a_1, r_1, \ldots)$ by sampling actions $a_t \sim \pi(\cdot | s_t)$ and transitioning to states $s_{t+1} \sim P(\cdot | s_t, a_t)$. The central goal of RL is to find a policy that maximizes the expected discounted return:
$$  
\eta(\pi) = \mathbb{E}_{\tau \sim \pi} \Bigg[ \sum_{t=0}^{\infty} \gamma^t r_t \Bigg].  
$$  
To facilitate policy optimization, we define the standard value functions under a policy $\pi$: the state-value function $V^\pi(s) = \mathbb{E}_{\tau \sim \pi}\Big[ \sum_{t=0}^{\infty} \gamma^t r_t \big| s_0 = s \Big]$, the action-value function $Q^\pi(s,a) = \mathbb{E}_{\tau \sim \pi}\Big[ \sum_{t=0}^{\infty} \gamma^t r_t \big| s_0 = s, a_0 = a \Big]$, and the advantage function $A^\pi(s,a) = Q^\pi(s,a) - V^\pi(s)$. A key theoretical tool for relating the performance of two distinct policies is the policy performance difference theorem \cite{kakade2002approximately}. It states that for any two policies, a target policy (to be optimized) $\trainerpi$ and a behavior policy (for rollout) $\rolloutpi$, their expected returns are related by:
\begin{equation}
\eta(\trainerpi) - \eta(\rolloutpi) = \frac{1}{1-\gamma} \, \mathbb{E}_{s \sim \rho^{\bluepi}, \, a \sim \bluepi(\cdot| s)} \big[ A^{\rolloutpi}(s,a) \big].  
\label{eq:perf-diff}
\end{equation}
Here, $\rho^{\trainerpi}(s) = (1-\gamma) \sum_{t=0}^{\infty} \gamma^t \Pr(s_t = s | \trainerpi)$ is the normalized discounted state-visitation distribution induced by the policy $\trainerpi$. This identity is fundamental, as it implies that any policy update that results in a non-negative expected advantage guarantees monotonic performance improvement, i.e., $\eta(\trainerpi) \ge \eta(\rolloutpi)$.

\subsection{Policy Improvement Bound}  

While Equation \ref{eq:perf-diff} provides a direct expression for policy improvement, its dependence on the state-visitation distribution $\rho^{\trainerpi}$ of the new policy makes it intractable for direct optimization. To overcome this, \citet{schulman2015trust} derive a lower bound on performance improvement that can be estimated using samples from the behavior policy $\rolloutpi$, with a penalty term that measures the divergence between the old and new policies. This lower bound forms the basis of trust-region methods.

\begin{theorem}\label{schulman2015trust}
	\citep{schulman2015trust, achiam2017constrained} Given any two policies, $\rolloutpi$ and $\trainerpi$, the following bound holds:
    \begin{equation} 
    \label{eq:tv-bound}  
    \begin{split}
        \!\!\!
        \eta(\trainerpi) \! - \! \eta(\rolloutpi)  
        \ge & 
        \frac{1}{1-\gamma} \mathbb{E}_{s \sim \rho^{\redmu},\, a \sim \redmu(\cdot| s)} \Bigg[\frac{\trainerpi(a| s)}{\rolloutpi(a| s)} A^{\rolloutpi}(s,a)\Bigg] \\ 
        & - \frac{2 \xi \gamma}{(1-\gamma)^2} {D_{\mathrm{TV}}^{\mathrm{max}}(\rolloutpi \| \trainerpi)}^2,
    \end{split}
    \end{equation}  
    where $\xi = \max_{s,a} \Big| A^{\rolloutpi}(s,a) \Big|$ and $D_{\mathrm{TV}}^{\mathrm{max}}(\rolloutpi \| \trainerpi) = \max_s D_{\mathrm{TV}}\big( \rolloutpi(\cdot| s) \|  \trainerpi(\cdot| s) \big)$, which is the maximum Total Variation (TV) divergence among all states.
\end{theorem}

This bound provides a direct path to guaranteed policy improvement. The right-hand side of the inequality forms a surrogate objective that is a tight lower bound on the true performance improvement, touching the objective when $\trainerpi = \rolloutpi$. Therefore, iteratively maximizing this surrogate guarantees monotonic improvement in $\eta(\trainerpi)$, following the principles of the Minorize-Maximization (MM) algorithm \citep{hunter2004tutorial, schulman2015trust}.  

\subsection{Trust Region Policy Optimization}

The policy improvement bound in \Cref{eq:tv-bound} directly justifies a surrogate objective,
\begin{equation}  
\label{eq:surrogate}
L_{\rolloutpi}(\trainerpi) = \frac{1}{1 - \gamma}  \mathbb{E}_{s \sim \rho^{\rolloutpi},\,a \sim \rolloutpi(a|s)} \left[ \frac{\trainerpi(a| s)}{\rolloutpi(a| s)} A^{\rolloutpi}(s,a) \right].  
\end{equation}
This objective serves as a \textbf{first-order approximation} of the true performance improvement $\eta(\trainerpi) - \eta(\rolloutpi)$, as their values and gradients match at the point of expansion $\trainerpi=\rolloutpi$ \citep{kakade2002approximately, schulman2015trust, zheng2025stabilizing}.
Therefore, maximizing $L_{\rolloutpi}(\trainerpi)$ within a small \textit{trust region} guarantees stable and meaningful policy improvement. This insight motivates the trust-region optimization approach \citep{schulman2015trust, xie2024simple}, which involves maximizing $L_{\rolloutpi}(\trainerpi)$ subject to a constraint that keeps the new policy $\trainerpi$ within a trust region around the current policy $\rolloutpi$, thereby ensuring the validity of the approximation. Formally, this is expressed as the following constrained optimization problem:
\begin{equation}   
\label{eq:trpo-obj}  
\begin{split}
\max_{\trainerpi} \quad & L_{\rolloutpi}(\trainerpi) \\  
\text{s.t.} \quad & D_{\mathrm{TV}}^{\mathrm{max}}(\rolloutpi \| \trainerpi) \le \delta, 
\end{split}
\end{equation}
% \begin{equation}   
% \label{eq:trpo-obj}  
% \max_{\trainerpi} \quad L_{\rolloutpi}(\trainerpi), \quad \quad 
% \text{s.t.} \quad D_{\mathrm{TV}}^{\mathrm{max}}(\rolloutpi \| \trainerpi) \le \delta, 
% \end{equation} 
where the constraint can also be applied on a KL divergence $D_{\mathrm{KL}}$, justified via Pinsker’s inequality:
$$D_{\mathrm{TV}}(\rolloutpi \| \trainerpi)^2 \le \tfrac{1}{2} D_{\mathrm{KL}}(\rolloutpi \| \trainerpi).$$  



\section{Trust Region Under LLM Regime}
\label{sec:llm_tr}  

In this section, we adapt the trust region framework to the specific context of LLM fine-tuning. This setting differs from the classical RL paradigm in two crucial ways. First, the learning problem is structured as an undiscounted ($\gamma=1$) episodic task with a finite horizon $T$, which makes the original bound in \Cref{eq:tv-bound} ill-defined, as the $\frac{1}{1-\gamma}$ term diverges to infinity. Second, due to the sparse reward nature, advantages are often estimated at the sequence level \citep{shao2024deepseekmath}, rather than on a per-token basis.

Formally, given a prompt $x$, a policy $\pi$ (the LLM) generates a response $y=(y_1, \dots, y_T)$ by sequentially sampling tokens. At each step $t$, the policy defines a conditional distribution $\pi(y_t|s_t)$ over the vocabulary $\mathcal{A}$, where the state $s_t = (x, y_1, \dots, y_{t-1})$ consists of the prompt and previously generated tokens. The probability of the complete response is the product of these conditional probabilities:  
$
\pi(y|x) = \prod_{t=1}^{T} \pi(y_t | s_t).  
$
After the full response is generated, a scalar reward $R(y, x)$ is provided. For brevity, we will omit the dependency on the initial prompt $x$ and write the objective function as: 
$$  
\mathcal{J}(\pi) = \mathbb{E}_{y \sim \pi}[R(y)].  
$$  
We now derive performance difference identity and policy improvement bound tailored to this regime.  

\begin{theorem}[Performance Difference Identity for LLMs]  
\label{lem:llm_identity} 
In a finite-horizon setting ($T$) with no discount ($\gamma=1$), for any two policies $\trainerpi$ and $\rolloutpi$, the performance difference can be decomposed as:  
\begin{equation}  
\mathcal{J}(\trainerpi) - \mathcal{J}(\rolloutpi) = L_{\rolloutpi}'(\trainerpi) - \Delta(\rolloutpi, \trainerpi),  \nonumber
\end{equation}  
where $L_{\rolloutpi}'(\trainerpi)$ is a surrogate objective defined as:  
\begin{equation}  
L_{\rolloutpi}'(\trainerpi) = \mathbb{E}_{y \sim \redmu} \left[ R(y) \sum_{t=1}^{|y|} \left( \frac{\trainerpi(y_t|s_t)}{\rolloutpi(y_t|s_t)} - 1 \right) \right],  
\label{eq:llm_surrogate}  
\end{equation}  
and $\Delta(\rolloutpi, \trainerpi)$ is an error term given by:  
\begin{align} 
\Delta(\rolloutpi, \trainerpi) =& \mathbb{E}_{y \sim \redmu} \Big[ R(y) \\
& \sum_{t=1}^{|y|} \left( \frac{\trainerpi(y_t|s_t)}{\rolloutpi(y_t|s_t)} \!-\! 1 \right) \! \left( 1 \!-\! \prod_{j=t+1}^{T} \frac{\trainerpi(y_j|s_j)}{\rolloutpi(y_j|s_j)} \right) \Big]. \nonumber
% & \sum_{t=1}^{|y|} \left( \frac{\trainerpi(y_t|s_t)}{\rolloutpi(y_t|s_t)} \!-\! 1 \right) \left( 1 \!-\! \frac{\trainerpi(y_{>t}|s_{t+1})}{\rolloutpi(y_{> t}|s_{t+1})} \right) \Big].  
\end{align} 
\end{theorem}  
This theorem provides an exact expression for the policy improvement. The surrogate $L_{\rolloutpi}'(\trainerpi)$ represents a first-order approximation, while the error term $\Delta$ captures the higher-order effects of the policy change. To make this practical for optimization, we bound the error term.  
  
\begin{theorem}[Policy Improvement Bound for LLMs]  
\label{thm:llm_tr_bound}  
In a finite-horizon setting ($T$) with no discount ($\gamma=1$), the policy improvement is lower-bounded by:  
\begin{equation}  
\label{eq:llm-tv-bound}  
\mathcal{J}(\trainerpi) - \mathcal{J}(\rolloutpi) \ge L_{\rolloutpi}'(\trainerpi) - 2 \xi T(T-1) \cdot {D_{\mathrm{TV}}^{\max}(\rolloutpi \| \trainerpi)}^2,  
\end{equation}  
where $\xi = \max_{y} |R(y)|$ is the maximum absolute reward, and $D_{\mathrm{TV}}^{\max}(\rolloutpi \|  \trainerpi) = \max_{s_t} D_{\mathrm{TV}}\big(\rolloutpi(\cdot|s_t)  \|  \trainerpi(\cdot|s_t)\big)$ is the maximum Total Variation (TV) divergence over all states.  
\end{theorem}  
This theorem establishes a lower bound on policy improvement, and it is structurally analogous to the bound in \Cref{schulman2015trust} (see Appendix \ref{app:compare_classical_rl}), with the horizon $T$ playing a role similar to the effective horizon $\frac{1}{1-\gamma}$ in the discounted setting. It provides a clear theoretical justification for adapting the trust region approach into LLM regime.
Similar to \Cref{eq:trpo-obj}, we can solve the following constrained optimization problem to guarantee stable learning:
\begin{equation}
\label{eq:llm-trpo-obj}  
\begin{split}
\max_{\trainerpi} \quad & L_{\rolloutpi}'(\trainerpi) \\  
\text{s.t.} \quad & D_{\mathrm{TV}}^{\max}(\rolloutpi \| \trainerpi) \le \delta,  
% \text{s.t.} \quad & \mathbb{E}_{y \sim {\rolloutpi}} \left[ \sum_{t=1}^{|y|} D\big(\rolloutpi(\cdot| s_t)\,\|\,\trainerpi(\cdot| s_t)\big) \right] \le \delta, \nonumber  
\end{split}
\end{equation}   
% \begin{equation}
% \label{eq:llm-trpo-obj} 
% \max_{\trainerpi} \quad  L_{\rolloutpi}'(\trainerpi), \quad \quad
% \text{s.t.} \quad  D_{\mathrm{TV}}^{\max}(\rolloutpi \| \trainerpi) \le \delta,  
% \end{equation}   
where the constraint can also be applied on a KL divergence.

The proofs for \Cref{lem:llm_identity} and \Cref{thm:llm_tr_bound} are deferred to Appendix~\ref{app:llm_tr_proof}. In Appendix~\ref{app:tighter_bound}, we further derive a more practical bound that depends linearly, rather than quadratically, on the horizon length $T$.


\section{Approach}
\label{sec:method}
The \model{} model was developed via a multi-stage pipeline, commencing with long-CoT SFT and advancing through reasoning and general RL phases. 
While the complete pipeline is integral to the model's performance, the primary focus of this report is on RL components. We first summarize the long-CoT SFT process that primes the model for RL in Section~\ref{sec:sft}. 
The core of our discussion then turns to the novel RL algorithm, which is elaborated upon in Section~\ref{sec:rl}.
Additionally, we introduce ASystem in Section~\ref{subsec:asys}, the distributed system that enabled the large-scale RL training necessary for this project.

\subsection{Training Pipeline}
\label{sec:pipeline}

\begin{figure}[!hbt]
    \centering
    \includegraphics[width=0.86\linewidth]{figures/Framework.pdf}
    \caption{The training pipeline of \model.}
    \label{fig:training_framework}
\end{figure}


We use Ling-1T-base model~\citep{lingv2}, a novel Mixture-of-Experts model with a total of \textbf{1 Trillion} parameters and an activation of \textbf{50 Billion} parameters, as our base model. The training of \model{} consists of three stages, comprising long-CoT SFT, reasoning-oriented RL, and general-oriented RL, to cultivate a powerful thinking model.
\begin{itemize}
    \item \textbf{Long-CoT SFT:} We collect and synthesize a large amount of multi-domain reasoning trajectory data covering mathematics, code, science, etc. Through large-scale supervised fine-tuning, the model learns general reasoning patterns and domain-specific reasoning skills, establishing a solid foundation for large-scale RL training.
    \item \textbf{Reasoning RL:} We construct a comprehensive, challenging, and high-quality RL dataset encompassing math, code, science, and logic tasks with verifiable outcomes. 
    The model's comprehensive reasoning performance is enhanced via \textit{RLVR (Reinforcement Learning from Verifiable Rewards)}. This process involves sampling extensive reasoning trajectories and refining the policy using verifiable rewards, which are provided by carefully designed multi-domain verifiers.
    \item \textbf{General RL:} Following large-scale reinforcement learning on verifiable tasks, we conduct a second RL stage focused on general tasks. This phase employs \textit{RLHF (Reinforcement Learning from Human Feedback)} to recalibrate the model's capability distribution, preserving its core reasoning strength while enhancing human alignment, instruction following, creative writing, safety, and overall usability.
    
\end{itemize}





\subsection{Long-CoT SFT}
\label{sec:sft}

In this stage, we aim to endow the base model with fundamental long-chain reasoning abilities through Long Chain-of-Thought Supervised Fine-Tuning (Long-CoT SFT). This process serves as the foundation for subsequent reinforcement learning, equipping it with the capability to sustain coherent, multi-step thinking processes for complex problems.

\paragraph{Data Collection} A comprehensive, high-quality dataset featuring Long-CoT reasoning patterns was constructed to effectively activate the base model’s reasoning capability. The query pool originates from a tripartite sourcing strategy: open-source repositories, expert manual generation, and LLM-based synthesis. 
Following data collection, rigorous data-cleansing protocols were applied. 
To ensure the resulting dataset's quality, we designed a rigorous data processing pipeline comprising four sequential steps: 1) Deduplication, where we employed exact matching to remove repetitive samples; 2) Harmful Content Filtering, where data samples containing toxic or harmful information were identified and purged; 3) Data Decontamination, where we utilized both hashing and exact string matching techniques to detect and eliminate any samples that overlap with existing benchmarks; and 4) Low-Quality Sample Filtering, which removeed various noise sources including invisible control codes and extraneous Unicode characters. The final data is predominantly composed of four domains: Mathematics (46\%), STEM (26\%), Code (20\%), and Others (8\%).

\paragraph{Training} We conduct Long-CoT SFT on our Ling-1T-base model~\citep{lingv2} to obtain a model with preliminary thinking capabilities. The training data are packed into 64k-length sequences. For this stage, the model was trained for 3 epochs with a learning rate of $2 \times 10^{-4}$. We employed a cosine decay scheduler with 30 warmup steps and applied a weight decay of 0.1 throughout the process.

\subsection{Large-Scale Reinforcement Learning}
\label{sec:rl}
Reinforcement learning (RL) is the critical next step for translating knowledge from pre-training and SFT into advanced thinking. To enable RL at the unprecedented scale of \textbf{1 Trillion} parameters, we developed a novel RL algorithm and a specialized infrastructure, Asystem. This integrated solution overcomes fundamental challenges in training efficiency and stability, allowing us to systematically explore the model's complex problem-solving capabilities.

\subsubsection{RL Data}
\label{subsec:rl-data}


High-quality and diverse data are critical for effective reinforcement learning. To this end, we introduce a carefully curated, multi-domain dataset spanning five core areas: math, code, science, logic, and general domains:
\begin{itemize}
    \item \textbf{Math:} We extend the dataset from~\cite{team2025ring} with mathematically rigorous problems from authoritative sources. Our curation ensures completeness, high complexity, and verifiable solutions, yielding a high-quality corpus for large-scale reinforcement learning.
    
    \item \textbf{Code:} Aside from the dataset employed in~\cite{team2025ring}, we develop a multi-phase workflow for synthesizing, validating, quality-scoring, and selecting additional test cases. This process ensures that each problem is equipped with a sufficient number of high-quality test cases. The final dataset contains programming problems with verified correct solutions and carefully tested cases.
     
    \item \textbf{Science:} We developed a crowdsourced science dataset of high-difficulty problems spanning physics, chemistry, and biology. To ensure complexity for reinforcement learning, all multiple-choice questions were reformatted into an open-ended format. 
    For organic chemistry, we established a dedicated image-semantization pipeline that converts visual information such as molecular structures into structured textual descriptions. 
    Finally, we applied a Pass-rate filtering strategy to select only the highest-quality items.
    
    \item \textbf{Logic:} Our logic reasoning dataset spans five domains: visual pattern induction ~\citep{Chollet2025}, grid puzzles (Sudoku), pathfinding (mazes), arithmetic reasoning (24 Game), and propositional logic (Knights and Knaves). We synthesized problems by integrating public resources such as \cite{Hodel2024}, \cite{Li2025InternBootcamp}, and \cite{Liu2025SynLogic} into an in-house game generator, enabling scalable and controlled creation. A quality control process ensures each task is solvable and non-trivial during both generation and post-processing. The final curated collection balanced across domains and complexity levels for reinforcement learning.

    \item \textbf{General Data:} We constructed a comprehensive dataset for general reasoning by aggregating problems from two primary sources: public repositories and real-world user interactions. From public sources, we incorporated established general datasets including Magpie~\citep{xu2024magpiealignmentdatasynthesis}, WMT~\citep{feng2025mtr1zero}, RLVR-IFEval~\footnote{https://huggingface.co/datasets/allenai/RLVR-IFeval}, and AutoIF~\footnote{https://huggingface.co/datasets/Post-training-Data-Flywheel/AutoIF-instruct-61k}. To enhance practical alignment, we further integrated real-world user preference data such as arena-human-preference-100k and arena-human-preference-140k~\footnote{https://huggingface.co/datasets/lmarena-ai}. Additionally, we augmented this collection with problems sourced from social media platforms such as Zhihu and StackOverflow. 
\end{itemize}

Finally, we employ a multi-stage curation pipeline involving parsing, reformulation, and deduplication, with quality assured by a dual scoring system of LLMs and rule-based metrics. Furthermore, fine-grained metadata annotations on each sample enable dynamic sampling and cross-domain blending, a strategy that significantly improves training efficiency and model performance on complex tasks.

\begin{figure}
    \centering
    \includegraphics[width=0.96\linewidth]{figures/rl_overview_preview.pdf}
    \caption{We integrate C3PO++ and IcePop into \model, which enhances both training efficiency and effectiveness of RL.}
    \label{fig:rl_overview}
\end{figure}

\subsubsection{IcePop: Discard All Noisy Gradient Updates}
Contemporary reinforcement learning training frameworks typically utilize distinct engines for model training and inference processes. Throughout our experiments, we have observed that this separation may lead to discrepancies in probability calculations, potentially introducing instability to RL training. This problem is particularly pronounced in the training of MoE models with RL due to the inherent usage of the dynamic routing mechanism. Additionally, in long CoT settings, these discrepancies can gradually accumulate across iterations and become further amplified.

\begin{theorem}{(Compounding Probability Discrepancy)}\label{theo:prob_dis}
Let $\pi_{\mathrm{infer}}(\cdot;\theta)$ and $\pi_{\mathrm{train}}(\cdot;\theta)$ be the policy model loaded by inference and training engines, and $\delta_t \;=\; D_{\mathrm{KL}}\!\big(\pi_{\mathrm{infer}}(\cdot;\theta_t)\,\|\,\pi_{\mathrm{train}}(\cdot;\theta_t)\big)$ be probability discrepancy at step $t$. 
Under certain conditions and a step size $\mu>0$,
there exist a constant $\eta>0$ such that $\delta_{t+1} \;\ge\; \big(1 + \tfrac{\eta}{2}\,\mu\big)\,\delta_t.$ 
\end{theorem}

To address this compounding mismatch issue in MoE RL, we propose \textbf{IcePop}, a variant of GRPO that suppresses unstable training updates through double-sided masking calibration. IcePop only calibrates gradients within the acceptable region and discards all noisy gradient updates beyond that boundary, effectively aligning $\pi_{\textcolor{blue}{\text{train}}}$ with $\pi_{\textcolor{teal}{\text{infer}}}$. This is achieved through two key techniques:
\begin{itemize}
    \item \textbf{Double-sided calibration}: We calibrate token-level gradients within a region defined by lower and upper limits, well preserving the alignments between training and inference probabilities.
    \item \textbf{Masking}: We exclude tokens with excessive probability deviation from gradient computation, constraining gradient updates in a stable region.
\end{itemize}
Integrating these techniques yields the following objective function for IcePop:
\begin{equation}
\begin{aligned}
\mathcal{J}_{{\text{IcePop}}}(\theta) &= \mathbb{E}_{x \sim \mathcal{D}, \{y_i\}_{i=1}^G \sim \pi_{\textcolor{teal}{\text{infer}}}(\cdot \mid x; \theta_{\rm old})} \left[ \frac{1}{G} \sum_{i=1}^G \frac{1}{|y_i|} \sum_{t=1}^{|y_i|} \Big[\mathcal{M}\Bigl(\frac{\pi_{\textcolor{blue}{\text{train}}}(y_{i,t} \mid x, y_{i,<t};\theta_{\text{old}})}{\pi_{\textcolor{teal}{\text{infer}}}(y_{i,t} \mid x, y_{i,<t}; \theta_{\mathrm{old}})}; \alpha, \beta\Bigr) \right. \\ &\left. \qquad \qquad \qquad \qquad \quad \qquad \cdot \min \left( r_{i,t}\widehat{A}_{i,t}, \text{clip} \left( r_{i,t}, 1 - \varepsilon, 1 + \varepsilon \right) \widehat{A}_{i,t} \right)  - \gamma D_{\text{KL}}(\pi_{\theta}\|\pi_{\text{ref}})\right]\Bigg], &
\end{aligned}
\end{equation}
where $r_{i,t} = \frac{\pi_{\textcolor{blue}{\text{train}}}(y_{i,t} \mid x, y_{i,<t}; \ \theta)}{\pi_{\textcolor{blue}{\text{train}}}(y_{i,t} \mid x, y_{i,<t}; \ \theta_{\text{old}})}$, $\mathcal{M}(k)$ is the masking function defined as below:

\begin{equation}
\mathcal{M}(k) =\begin{cases} k & \text{if \ } k \in [\alpha, \beta], \\ 
0 & \text{otherwise}\end{cases}
\end{equation}
where $\alpha$,  $\beta$ controls the lower and upper limits.
Thus, the gradient of IcePop is
\begin{equation}
\label{eq:gradient_icepop}
\nabla_\theta \mathcal{J}_{\text{IcePop}}(\theta) \sim \mathbb{E}_{a \sim \textcolor{teal}{\pi_{\text{infer}}}(\theta_{\text{old}})} \Bigg[\mathcal{M}\Bigg(\frac{\textcolor{blue}{\pi_{\text{train}}}(a;\theta_{\text{old}})}{\textcolor{teal}{\pi_{\text{infer}}}(a;\theta_{\text{old}})}\Bigg ) \cdot \nabla_\theta \log \textcolor{blue}{\pi_{\text{train}}}(a;\theta) \cdot \hat{A} \cdot r(a)\Bigg)\Bigg].
\end{equation}
For $\dfrac{\textcolor{blue}{\pi_{\text{train}}}(a;\theta_{\text{old}})}{\textcolor{teal}{\pi_{\text{infer}}}(a;\theta_{\text{old}})} < \alpha$ and $\dfrac{\textcolor{blue}{\pi_{\text{train}}}(a;\theta_{\text{old}})}{\textcolor{teal}{\pi_{\text{infer}}}(a;\theta_{\text{old}})} > \beta$, IcePop discards all noisy gradients outside the region that may introduce potential instability into the training process. A more detailed introduction of IcePop could refer to our blog~\footnote{https://ringtech.notion.site/icepop}.


\subsubsection{C3PO++: Dynamically Partition Rollouts with Budget} 

\begin{figure}[!hbt]
    \centering
    \includegraphics[width=0.9\linewidth]{figures/c3po++_overview.pdf}
    \caption{C3PO++ improves reinforcement learning efficiency for large thinking models by maintaining a rollout buffer across policy model versions. Once the rollout in an iteration reaches the token budget, optimization is performed; unfinished rollouts are stored in the buffer and resumed by the updated policy in the next iteration.}
    \label{fig:c3po++}
\end{figure}

We introduce C3PO++, an extension of C3PO~\citep{team2025ring} that incorporates a budget-controlled rollout partition mechanism. This approach dynamically partitions rollout generation to prevent idleness of computational resources caused by individual long rollouts. The system incorporates two modules: a high-throughput inference pool $P_\text{infer} $ with capacity $\Omega_{\text{infer}}$ for parallel generation, and a training pool $Q_\text{train}$ with capacity $\Omega_\text{train}$ for collecting completed trajectories. Similar to the idea of C3PO, we regulate the rollout generation with a \textit{token budget} ($\Phi$), which stabilizes the training updates and enables a highly efficient rollout process.

The C3PO++ procedure is detailed in Algorithm \ref{algo:c3po++}. At iteration $t$, the inference engine $\pi_{\text{infer};\theta_t}$ populates the inference pool by generating rollouts in parallel, while tracking the cumulative number of generated tokens $C$ in real-time. When a rollout reaches a terminal state (i.e., $\texttt{[EOS]}$), it will be moved from $P_{\text{infer}}$ to the training pool $Q_{\text{train}}$ and counted towards the training tokens $C$. Inference proceeds until $C$ reaches the token budget $\Phi$. At this point, the training engine $\pi_{\text{train};\theta_t}$ updates the parameters with the completed trajectories in $Q_{\text{train}}$ regulated by the token budget, which may include samples resumed from earlier inference versions. We denote the number of partitions a sequence has undergone as the retention period. For each iteration, the retention period of unfinished rollouts will be automatically increased by 1. Before each iteration, rollouts whose retention period exceeds a threshold $\sigma$ are purged from $P_{\text{infer}}$. Meanwhile, new prompts may be sampled to refill $P_{\text{infer}}$ until it reaches capacity $\Omega_{\text{infer}}$. After the model parameters are updated to $\theta_{t+1}$, the inference engine $\pi_{\text{infer};\theta_{t+1}}$ initiates a new iteration of rollout generation, continuing the process for rollouts within the valid retention period and monitored by the token budget.


\begin{algorithm}[!tbh]
\caption{C3PO++}
\label{algo:c3po++}
\KwIn{
initial parameters $\theta_0$; inference engine $\pi_{\text{infer};\theta}$; training engine $\pi_{\text{train};\theta}$;
token budget $\Phi$;
inference pool capacity $\Omega_{\text{infer}}$;
retention threshold $\sigma$.
}
\KwOut{Sequence of parameter updates $\theta_0 \rightarrow \theta_1 \rightarrow \cdots$}
\textbf{State:} Inference pool $P_{\text{infer}}$ (capacity $\Omega_{\text{infer}}$); training pool $Q_{\text{train}}$.
\Begin{
  $P_{\text{infer}} \gets \emptyset$; $Q_{\text{train}} \gets \emptyset$; $t \gets 0$ \;
  \While{not converged}{
    $C \gets 0$\;
    \ForPar{$o \in P_{\text{infer}}$}{
        \If{$\text{retention}(o, \sigma)$}{
          $P_{\text{infer}} \gets P_{\text{infer}} \backslash \{o\}$  \tcp*{remove overextended rollouts from inference pool}
        }
      }
    \While{$(C < \Phi)$}{
      \If{$|P_{\text{infer}}| < \Omega_{\text{infer}}$}{
        $P_{\text{infer}} \gets P_{\text{infer}} \cup \text{sample\_prompt}()$ \tcp*{maintain a full inference pool}
      }
      \ForPar{$o \in P_{\text{infer}}$}{
        $o \gets \pi_{\text{infer};\theta_t}(o)$ \tcp*{generate the next token for rollouts in parallel}
        \If{$\text{terminal}(o)$}{
          $C \gets C + |o|$ \tcp*{cumulate token amount}
          $Q_{\text{train}} \gets Q_{\text{train}} \cup \{o\}$ \tcp*{save completed rollouts for training}
          $P_{\text{infer}} \gets P_{\text{infer}} \backslash \{o\}$  \tcp*{remove completed rollouts from inference}
        }
      }
    }
    $\theta_{t+1} \gets \text{Update}(\theta_t, Q_{\text{train}})$ \;
    $Q_{\text{train}} \gets \emptyset$ \;
    $t \gets t + 1$ \;
  }
  \Return{$\theta_t$}
}
\end{algorithm}




\subsubsection{Training Recipe}
\label{subsec:rl-train}
All policy optimization was conducted using the ASystem framework. We employ the AdamW optimizer with hyperparameters $\beta_1$ = 0.9, $\beta_2$ = 0.999, weight decay of $0.01$, and with the MoE router bias held fixed. 
\textbf{For the reasoning RL stage}, we implemented the proposed IcePop ($\alpha=0.5$, $\beta=5$) and C3PO++ algorithms. The training configuration used a learning rate of $2 \times 10^{-6}$, a KL coefficient of $0.0$, and a sampling temperature of $1.0$. Each training step utilized 480 unique prompts, with 8 rollouts sampled per prompt and a maximum length of 65,536 tokens. 
\textbf{For the general RL stage}, we utilized GRPO with a learning rate of $3 \times 10^{-6}$, a KL coefficient of $0.0$, and a sampling temperature of $1.0$. Each step in this stage consisted of 80 unique questions with 8 outputs each, and a maximum length of 32,768 tokens.
\subsection{Experiments and Analysis}
\label{subsec:rl-exp}
This section presents experiments to validate the effectiveness of our proposed methods: IcePop, which ensures stable policy optimization, and C3PO++, which enables efficient rollout generation.
\subsubsection{IcePop}\label{subsec:rl-exp-icepop}

\paragraph{Setup} To evaluate the effectiveness of IcePop, we conduct preliminary experiments on the Ring-mini-2.0\footnote{\href{https://huggingface.co/inclusionAI/Ring-mini-2.0}{https://huggingface.co/inclusionAI/Ring-mini-2.0}} model, which is a MoE model with 16.8B total parameters and 0.75B activated parameters. We compare three settings: (1) IcePop with $\alpha=0.5, \beta=5$, (2) TIS~\citep{yao2025offpolicy} with the officially recommended setting, which mitigates the training-inference mismatch issue with importance-sampling correction, and (3) Vanilla GRPO without the KL-term. For fair comparison, we use the same training dataset for all models.

\paragraph{Preliminary results on Ring-mini-2.0.} As shown in Figure \ref{fig:icepop_aime25}, we can see that IcePop consistently outperforms TIS on the challenging benchmark AIME25, with a large gain along the training process, and finally improves the base score (63\%) by over 14\%, and expands the performance gap with TIS by relative 6\%.

\begin{figure}[!htb]
    \centering
    \includegraphics[width=0.6\linewidth]{figures/RL/AIME25.pdf}
    \caption{The performance comparison on AIME25 (Avg@64). We evaluate all models using the same setting.}
    \label{fig:icepop_aime25}
\end{figure}


\paragraph{Experiments on Ring-1T.} As training progresses, we can see from Figure~\ref{fig:icepop_ring_1t} that the original GRPO suffers from training instability, as both the gradient norms and the probability discrepancy between the inference and training engines tend to increase rapidly. However, after applying IcePop, we can observe that the mismatch issue has been largely mitigated, stabilizing the RL training process. 


\begin{figure}[h!]
\centering
\begin{subfigure}[b]{0.45\textwidth}
\centering
\includegraphics[width=\textwidth]{figures/RL/ring_max_grad.pdf}
\end{subfigure}
\hspace{2mm}
\begin{subfigure}[b]{0.45\textwidth}
\centering
\includegraphics[width=\textwidth]{figures/RL/ring_max_logp.pdf}
\end{subfigure}
\begin{subfigure}[b]{0.45\textwidth}
\centering
\includegraphics[width=\textwidth]{figures/RL/ring_max_logp_diff_abs_mean.pdf}
\end{subfigure}
\hspace{2mm}
\begin{subfigure}[b]{0.45\textwidth}
\centering
\includegraphics[width=\textwidth]{figures/RL/ring_max_logp_diff_max.pdf}
\end{subfigure}
\caption{The training dynamics before and after applying IcePop.}
\label{fig:icepop_ring_1t}
\end{figure}

\subsubsection{C3PO++}\label{subsec:rl-exp-c3poplus}
We compare C3PO++ with the baseline setting that omits our budget-controlled rollout partition mechanism, assessing training efficiency and effectiveness in terms of training time, training reward, and benchmark performance. 
\begin{itemize}
    \item \textbf{Training Time.}~~As illustrated in Figure~\ref{fig:c3popp_time}, C3PO++ substantially reduces the time of the rollout phase, achieving an approximately 2.5 times speedup per step. Since rollout duration usually accounts for a large portion of training time in RL, the training optimization designed by C3PO++ yields about a 1.5 times speedup for the end-to-end phase per step, significantly boosting the training efficiency for reinforcement learning.
    \begin{figure}[!htb]
    \centering
    \begin{subfigure}[b]{0.45\textwidth}
    \centering
    \includegraphics[width=\textwidth]{figures/RL/c3po++_e2e.pdf}
    \end{subfigure}
    \hspace{2mm}
    \begin{subfigure}[b]{0.45\textwidth}
    \centering
    \includegraphics[width=\textwidth]{figures/RL/c3po++_rollout.pdf}
    \end{subfigure}
    \caption{Comparison of time cost between C3PO++ and the baseline.}
   \label{fig:c3popp_time}
   \end{figure}
   
    \item \textbf{Reward and Performance.}~~As shown in Figure~\ref{fig:c3popp_reward}, the reward curve of C3PO++ remains close to that of the baseline, suggesting that our optimization in rollout management maintains comparable training dynamics in the reinforcement learning process. On the representative reasoning benchmarks, C3PO++ achieves performance on par with the baseline, demonstrating its strength in producing competitive results.
    \begin{figure}[!htb]
    \centering
    \begin{subfigure}[b]{0.45\textwidth}
    \centering
    \includegraphics[width=\textwidth]{figures/RL/c3po++_reward.pdf}
    \end{subfigure}
    \hspace{2mm}
    \begin{subfigure}[b]{0.45\textwidth}
    \centering
    \includegraphics[width=\textwidth]{figures/RL/c3po++_perf.pdf}
    \end{subfigure}
    \caption{Comparison of reward and benchmark performance between C3PO++ and the baseline.}
   \label{fig:c3popp_reward}
   \end{figure}

\end{itemize}




\subsection{Large-Scale RL Infrastructure: ASystem}
\label{subsec:asys}


\begin{figure}[!hbt]
    \centering
    \includegraphics[width=0.85\linewidth]{figures/asystem.pdf}
    \caption{An overview of ASystem RL training framework.}
    \label{fig:asystem}
\end{figure}

Training \model{} with reinforcement learning requires a specialized infrastructure that can manage its unprecedented scale. The sheer size of the model, coupled with the inherent complexity of distributed RL workflows, poses unique challenges in memory management, state synchronization, and computational throughput. To this end, we developed ASystem, a high-performance RL framework whose components are co-designed with the requirements of \model{} in mind.

As illustrated in Figure~\ref{fig:asystem}, ASystem's architecture is built around a unified execution environment and includes the following key components, each engineered to address a specific bottleneck in the RL training for a trillion-parameter model:

\begin{itemize}
\item \textbf{Hybrid Runtime}: The core of ASystem, this runtime seamlessly integrates training and inference workloads. For \model{}, this means we can conduct massive parallel policy evaluation (inference) and model weight updates (training), eliminating the overhead of data transfer between separate systems and ensuring efficient utilization of thousands of GPUs.
\item \textbf{AMem}: AMem is a GPU memory management library designed to overcome the critical memory bottleneck in large-scale RL training, like that of our 1T model. It optimizes memory usage and data transfer, enabling larger batches, fewer OOM errors, and faster deployment with minimal code changes and no loss of accuracy.
\item \textbf{AState}: AState is a high-performance weight synchronization framework for RL. It efficiently addresses the challenge of distributing updated model parameters from trainers to inference actors using a zero-redundancy peer-to-peer mechanism, enabling synchronization of trillion-parameter models in under 10 seconds.
\item \textbf{ASandbox}:  A serverless environment for rapid scenario validation. By offering millisecond-scale cold start and high-throughput isolation, ASandbox accelerates evaluation of \model{} rollouts during large-scale RL training.
\end{itemize}


This foundational design, based on a SingleController + SPMD (Single Program, Multiple Data) architecture, delivers significant advantages for robust large-scale training. It provides plug-and-play support for training, inference, and reward model backends, facilitating independent debugging and development at scale. Crucially, by separating the control flow from the data flow, ASystem effectively mitigates the single-point data flow bottlenecks prevalent in mainstream SingleController frameworks. Furthermore, the system incorporates mechanisms for fast-fail reporting and automatic recovery from slow training and hangs, thereby enhancing overall training stability and efficiency for demanding workloads like our \model{} model.


\subsubsection{Hybrid Runtime: A Unified Training-Inference Execution Environment}

Hybrid Runtime is an integrated training-inference system designed for large-scale LLM reinforcement learning. It provides a high-performance, elastic, and scalable foundation by unifying efficient resource scheduling, linear scalability, comprehensive parallelism strategies, and a unified execution engine. The system is architected to support models of diverse architectures, scales, and training paradigms on large-scale clusters.

To bridge the gap between dynamic training and real-time inference in reinforcement learning (RL), we introduce \textbf{AState}, a high-speed framework for synchronizing weights between training and inference. AState provides a unified weight management API that supports diverse model architectures, deployment topologies, and pipeline paradigms without requiring framework modifications. At its core, a zero-redundancy peer-to-peer transmission mechanism delivers only necessary weight shards, enabling in-place updates on inference engines to eliminate costly data copies. This is complemented by a hardware–software co-design that optimizes data movement through NUMA topology and CPU-GPU affinity awareness, alongside a multi-transport communication layer (integrating RDMA, NCCL, and shared memory) that dynamically selects the optimal protocol based on data size and hardware topology. Consequently, AState achieves sub-second parameter updates, ensuring inference rollouts use the latest model and maintaining the critical training-inference alignment essential for stable policy optimization.

To enhance GPU memory efficiency, we introduce \textbf{AMem}, a memory and data transfer library optimized for RL workloads on GPU clusters. AMem enhances memory management efficiency through three key mechanisms: (1) Memory Switching, for the transparent release and resumption of training state, including NCCL communications and CUDA graphs; (2) Distributed Multi-path Transfer~\cite{shen2025flexlinkboostingnvlinkbandwidth}, which aggregates bandwidth across multiple channels; and (3) Unified Memory Pooling, for dynamic allocation across GPUs and nodes. By enabling larger batch sizes, reducing out-of-memory (OOM) errors, and accelerating system startup, AMem alleviates common bottlenecks in large-scale RL. The library is designed for transparency, requiring no model modifications and ensuring no impact on RL convergence, thereby providing robust infrastructure support for the Hybrid Runtime and AState components.

\subsubsection{ASandbox: An On-Demand Serverless Sandbox Engine}
ASandbox is a serverless sandbox engine for RL, providing rapid, isolated environments for tasks like code execution and terminal simulation. Integrated with Kubernetes and deployable as a standalone FaaS cluster, it executes RL tasks via function calls. It offers specialized sandboxes (e.g., math, code, STEM, terminal) supporting HTTP and MCP protocols. To ensure the consistent, stable feedback critical for RL training, it features: 1) Security: Kernel-level isolation via secure containers (runsc, kata); 2) Availability: Automatic node failure detection and isolation; 3) Speed: 100ms startup via image caching, cgroups, and fork; 4) Scalability: 5,000 QPS/200ms throughput via scheduling partitions.


\subsubsection{AReaL: A High-Performance RL Algorithm Framework}
ASystem is a unified, high-performance foundation for distributed reinforcement learning. Its reinforcement learning component, AReaL, is an open-source framework~\citep{fu2025areal} built to prioritize algorithm development by balancing ease of use with system flexibility. It offers both single-controller and SPMD interfaces through minimalist APIs and an extensible plugin mechanism, allowing researchers to focus on algorithmic innovation.

AReaL is characterized by several key features as follows:

\begin{itemize}
    \item \textbf{Asynchronous Multi-Stage Pipeline:} A fully decoupled architecture that concurrently executes trajectory generation, reward computation, and training. This overlap eliminates rollout long-tail issues and maximizes hardware utilization.
    \item \textbf{Efficient Data Management:} Intelligent data packing and sharding minimize padding and rebalancing overhead, reducing computational waste and training stalls.
    \item \textbf{Fault Tolerance:} The system features automated error detection, retry, and recovery mechanisms to ensure stability amidst hardware and software failures.
    \item \textbf{Massive Scalability:} By separating control and data planes, AReaL avoids the single-controller bottleneck, enabling seamless scaling across large clusters.
\end{itemize}





% \vspace{-0.2em}
\section{Analysis on Training Stability}  
\label{sec:stability}  

The RL fine-tuning of LLMs is prone to training instability due to \textit{training-inference mismatch} (see Appendix \ref{app:related_work_mismatch}). 
% Recent work has identified a key culprit: the \textit{training-inference mismatch} (${\trainerpi}_\theta \neq {\rolloutpi}_\theta$), where the policy distribution used for gradient computation ($\trainerpi_\theta$) diverges from the one used for data generation ($\rolloutpi_\theta$), even when using identical model parameters $\theta$ \citep{yao2025offpolicy, qi2025defeating, liu-li-2025, zheng2025stabilizing}. This discrepancy arises from numerical precision errors \citep{qi2025defeating} and subtle differences in implementation \citep{Team2025EveryAM, he2025nondeterminism}. As training progresses, this mismatch can be amplified if the RL algorithm cannot manage it appropriately, leading to catastrophic performance degradation.
In this section, we conduct an empirical study to dissect this issue and verify the stability of our DPPO algorithm. To formalize our analysis, we denote the parameters being optimized as $\theta$ and the parameters used for data generation as $\theta'$. We aim to answer three fundamental research questions:
\begin{enumerate}%[nosep]
    \item Given the extremely low learning rates (e.g., $10^{-6}$) common in LLM fine-tuning, is a trust region still necessary to ensure training stability?  
    \item Should the trust region be defined with respect to the original rollout distribution ($\rolloutpi_{\theta'}$) or a recomputed policy distribution ($\trainerpi_{\theta'}$)?  
    \item What specific types of policy updates are the primary drivers of training instability?  
\end{enumerate}

\textbf{Experimental Setting:}
Our experimental setup follows the sanity test proposed by \citet{qi2025defeating}. We fine-tune DeepSeek-R1-Distill-Qwen-1.5B~\citep{guo2025deepseekr1} on a curated set of 1,460 problems from the MATH dataset \citep{hendrycks2021measuring}. In this setting, a stable algorithm should theoretically converge to 100\% training accuracy, as all problems are known to be solvable by the initial model. 

We evaluate several algorithms, each representing a different approach to managing the policy update. The baselines include: \textbf{PG-IS} and its truncated variant \textbf{PG-TIS} (also known as CISPO \citep{chen2025minimax}), which use standard policy gradients with token-level importance sampling; \textbf{GRPO with Clip-Higher}, a PPO-like algorithm where clipping is based on the rollout policy ratio $r_t = \frac{\trainerpi_\theta}{\rolloutpi_{\theta'}}$ \citep{shao2024deepseekmath, liu2025understanding}; and \textbf{MiniRL \& MiniRL-TIS}, a PPO variant where clipping is based on a recomputed policy ratio $r_t = \frac{\trainerpi_\theta}{\trainerpi_{\theta'}}$ \citep{zheng2025stabilizing}. We compare these against \textbf{DPPO (Ours)}, our proposed method using either binary KL or TV divergence, with the trust region defined with respect to the rollout distribution $\rolloutpi_{\theta'}$. Detailed configurations for each algorithm are provided in the Appendix \ref{app:sanity_test}.

% We evaluate several algorithms, each representing a different approach to manage the policy update. We denote the parameters being optimized as $\theta$ and the parameters used for data generation as $\theta'$:  
% \begin{itemize}  
%     \item \textbf{PG-IS \& PG-TIS (CISPO):} Unconstrained policy gradient with token-level importance sampling. The TIS variant truncates the importance ratio, a method also known as CISPO \citep{chen2025minimax}.  
%     \item \textbf{GRPO with Clip-Higher:} A PPO-like algorithm where the ratio clipping, $r_t = \frac{\trainerpi_\theta}{\rolloutpi_{\theta'}}$, is based on the rollout policy \citep{shao2024deepseekmath, liu2025understanding}.  
%     \item \textbf{MiniRL \& MiniRL-TIS:} A PPO variant where the ratio clipping is based on a recomputed policy distribution, $r_t = \frac{\trainerpi_\theta}{\trainerpi_{\theta'}}$ \citep{zheng2025stabilizing}.  
%     \item \textbf{DPPO (Ours):} Our proposed method using either binary KL or TV divergence, with the trust region defined with respect to the rollout distribution $\rolloutpi_{\theta'}$.  
% \end{itemize}  
% Detailed configurations for each algorithm are provided in the Appendix.


% \vspace{-0.2cm}
\subsection{The Necessity of a Trust Region}  
  
Our first question addresses whether a trust region is redundant at low learning rates. \Cref{fig:sanity_test} provides a clear answer. The unconstrained methods, PG-IS and PG-TIS (CISPO), both suffer from an increasing training-inference mismatch, which culminates in a collapse of performance. In contrast, our DPPO variants, which enforce a principled trust region, maintain a stable, low level of mismatch throughout training and achieve near-perfect final rewards.  

\textbf{Takeaway 1:} A trust region is essential for stable training, even with very small learning rates. Without it, the training-inference mismatch accumulates and leads to collapse.  

\begin{figure}[h]
    \centering  
    \includegraphics[width=1.0\linewidth]{figs/which_trust_region.pdf}  
    \caption{Switching the stable DPPO-KL to a decoupled objective causes the mismatch to grow and performance to collapse, confirming that the trust region must be anchored to the rollout policy.}  
    % \vspace{-1.7em}
    \label{fig:which_trust_region}  
\end{figure}  

% \vspace{-0.2cm}
\subsection{The Correct Anchor for the Trust Region}  
\label{sec:correct_ancher_for_truct_region}
% \vspace{-0.1cm}

Next, we investigate to which distribution the trust region should be anchored. A common practice in open-source implementations \citep{sheng2024hybridflow, slime_github} is to use a \textit{decoupled} objective \citep{hilton2022batch}, where the trust region is enforced relative to a recomputed policy distribution ($\trainerpi_{\theta'}$) instead of the original behavior policy ($\rolloutpi_{\theta'}$). The MiniRL algorithm, for example, follows this design \citep{zheng2025stabilizing}.  
Our results show this choice is detrimental. As in \Cref{fig:sanity_test}, MiniRL fails to control the training-inference mismatch and its performance collapses, despite using a trust region. To confirm this, we created a decoupled version of our stable DPPO-KL algorithm. \Cref{fig:which_trust_region} shows that this single change corrupts the stable training process, causing the mismatch to grow and performance to collapse.

\textbf{Takeaway 2:} The trust region must be defined with respect to the original behavior policy ($\rolloutpi_{\theta'}$). Using a recomputed on-policy distribution as the anchor leads to instability. This finding aligns with the theoretical bound in \Cref{eq:llm-tv-bound} and offers a significant practical benefit: by removing the need for recomputation, we can reduce training costs by approximately 25\% \citep{qi2025defeating}.  
  
\begin{figure}[h]
    \centering  
    \includegraphics[width=1.0\linewidth]{figs/rewards_mask_fraction_combined.pdf}  
    \caption{Isolating the source of instability. The solid curves are training rewards, while the dashed lines are the percentage of \textit{bad updates}. Starting with the unstable PG-IS, applying a minimal mask that only blocks large-divergence bad updates on negative samples is sufficient to stabilize training, indicating these bad updates are the primary cause of training instability.}  
    \label{fig:ablation_mask_negative}  
    % \vspace{-1.4em}
\end{figure}

% \vspace{-0.2cm}
\subsection{Identifying the Source of Instability}  
% \vspace{-0.1cm}

Finally, we seek to pinpoint which specific policy updates are most responsible for the instability. Our methodology is to start with the unstable PG-IS algorithm, which applies no update masking, and introduce the most minimal mask necessary to restore stability. This allows us to isolate the most detrimental class of updates.  
Since updates on positively rewarded samples are typically safe, we focus on negative samples where the policy is penalized \citep{liu2025deepseek, ren2025learning_dynamics_LLM}. We design a simple mask that only blocks updates on negative samples where the probability of the sampled token is decreased by more than a threshold $\delta$:  \looseness=-1
$
M_t = 0 \; \text{if} \; \hat{A}_t < 0 \; \text{and} \; {\rolloutpi}_{\theta'}(y_t|s_t) - {\trainerpi}_{\theta}(y_t|s_t) \ge \delta. 
$
% \begin{align*}  
% M_t = 0 \; \text{if} \; \hat{A}_t < 0 \; \text{and} \; {\rolloutpi}_{\theta'}(y_t|s_t) - {\trainerpi}_{\theta}(y_t|s_t) \ge \delta.  
% \end{align*}  
As shown in \Cref{fig:ablation_mask_negative}, applying this minimal mask with $\delta=0.5$ is sufficient to stabilize the training. In contrast, a slightly looser mask ($\delta=0.8$) or one anchored to the recomputed distribution (``Mask-0.5-Recompute'') both fail to prevent the eventual collapse. We define \textit{bad updates} as those where this divergence exceeds 0.5 and plot their percentage over time. The plot reveals that only a very small fraction of updates are ``bad'' ($\leq 0.5\%$) yet they are the primary culprits behind training collapse. Furthermore, the percentage of these bad updates strongly correlates with reward fluctuation; as the fraction of bad updates rises, the reward curve becomes more erratic, reinforcing a causal link.\looseness=-1

\textbf{Takeaway 3:} The primary source of instability is a small subset of updates on negative samples that push the policy far outside the trust region. A likely reason is that aggressively penalizing a token the model deems probable can corrupt the LLM's internal knowledge and destabilize the learning process. This finding confirms the critical need for a trust region, particularly when handling negative feedback.  


% \vspace{-0.2cm}
\subsection{The Pitfalls of Truncated Importance Sampling}  
  
Our empirical results also reveal a surprising finding regarding Truncated Importance Sampling (TIS), a technique widely adopted to control the variance of policy gradient estimates \citep{yao2025offpolicy, chen2025minimax}. Contrary to its intended purpose, TIS consistently degrades training stability in our experiments. As illustrated in \Cref{fig:sanity_test}, the TIS-enabled variants (PG-TIS and MiniRL-TIS) suffer from premature collapse and significantly underperform their untruncated counterparts.  

We hypothesize that this detrimental effect stems from the same issue as PPO's ratio clipping: low-probability tokens, which naturally produce high-variance ratios, are the most likely to be truncated by TIS. While this does reduce variance, it systematically down-weights the gradient signal from these tokens, introducing a significant and harmful bias into the policy update. This suggests that naive truncation can be just as damaging as naive clipping.




% \vspace{-0.5em}
\section{Analysis on Training Efficiency}
\label{sec:training_efficiency}





Beyond training stability, the design of trust region is also critical for training \textit{efficiency}. As motivated in \Cref{sec:method_limitations}, PPO's ratio-clipping over-constrains the updates to low-probability tokens, which might be permitted by a divergence-based trust region. In this section, we aim to analyze how low-probability tokens affect the training dynamics, thus justifying the adoption of divergence-based trust region in our DPPO algorithm.

\begin{figure}[h]
    \centering
    \includegraphics[width=\linewidth]{figs/low_prob_sweep.pdf}
    \caption{Analysis of relaxing trust regions for low-probability tokens. (\textbf{Left}) Training reward curves. (\textbf{Middle}) Rollout probability of clipped tokens. (\textbf{Right}) Entropy of clipped tokens.}
    % \vspace{-1.2em}
    \label{fig:low_prob_sweep}
\end{figure}

\textbf{Experimental Setting:}  
We fine-tune Qwen3-1.7B-Base \citep{yang2025qwen3} on the DAPO dataset \citep{yu2025dapo}. We employ GRPO~\citep{guo2025deepseekr1, liu2025understanding}  with the Clip-Higher trick~\citep{yu2025dapo} as the baseline algorithm. 
% The probability ratio is defined as $r_t = \frac{{\trainerpi}_\theta(y_t|s_t)}{\rolloutpi_{\theta'}(y_t|s_t)}$.
We  then \textit{relax} trust regions by setting the clipping threshold $\epsilon$ in \Cref{eq:ppo-clip} as infinity for tokens with $\rolloutpi(y_t|s_t) < \alpha$, thus isolating the effect of low-probability tokens.\looseness=-1

The learning curves for varying values of $\alpha$ are presented in Figure~\ref{fig:low_prob_sweep}. Notably, relaxing the clipping constraint for tokens with $\rolloutpi(y_t|s_t) < 0.1$ yields a substantial improvement in training efficiency compared to the GRPO baseline ($\alpha=0$). This observation validates our hypothesis that the ratio-clipping mechanism in PPO over-constrains updates to low-probability tokens, thereby hindering overall learning progress.
The middle plot reveals that \textbf{clipped tokens are predominantly characterized by low probabilities} (typically below $0.15$ for the baseline in blue). As $\alpha$ increases, the probabilities of clipped tokens also rise, confirming that PPO's ratio-clipping is structurally biased against low-probability tokens. Furthermore, the right plot demonstrates that \textbf{clipped tokens frequently exhibit high entropy}. Consistent with \citet{wang2025beyond}, which posits that RL is driven primarily by high-entropy tokens in LLMs, our results suggest that relaxing constraints on these tokens enables more informative policy updates and thus achieves higher training efficiency (see Appendix \ref{app:clipped_tokens} for most frequent clipped tokens).\looseness=-1

Furthermore, we examine the effect of directional clip relaxation with a fixed $\alpha \! = \! 0.1$. We generalize the clip operation with asymmetric thresholds, denoted as $\operatorname{clip}(r_t, 1 \! - \! \epsilon_{\text{low}}, 1 \! + \! \epsilon_{\text{high}})$, where $\epsilon_{\text{low}} \! = \! 0.2$ and $\epsilon_{\text{high}} \! = \!0.28$ by default. We relax either one end (\textit{Relax-high} or \textit{Relax-low}) or both ends (\textit{Relax-both}). For example, Relax-high is implemented by $(\epsilon_{\text{low}}=0.2, \epsilon_{\text{high}}=\infty)$ for tokens with $\rolloutpi(y_t|s_t) < \alpha$.\looseness=-1


% \begin{figure}[t]
%     \centering  
%     \includegraphics[width=\linewidth]{figs/main-lora.pdf}  
%     \caption{Evolution of AIME24 and AIME25 scores during RL training with LoRA based on Qwen3-30B-A3B-Base.}  
%     \label{fig:main_lora}  
% \end{figure}



As illustrated in Figure~\ref{fig:clip_direction}, the direction of clip relaxation plays a critical role in the training efficiency and stability. Relax-high can be viewed as an extreme variant of the Clip-Higher trick \citep{yu2025dapo} applied only to low-probability tokens. While this approach maintains high entropy, it fails to yield significant gains in training efficiency. Conversely, Relax-low exhibits substantially faster initial learning\footnote{In contrast to the Clip-Higher intuition \citep{yu2025dapo}, we observe that ``Clip-Lower'' (relaxing $\epsilon_{\text{low}}$) for low-probability tokens is more vital for efficiency. This aligns with findings by \citet{pmlr-v235-tajwar24a} regarding the role of negative gradients in accelerating preference learning.}. However, this strategy eventually drops due to entropy collapse \citep{cui2025entropy}. Ultimately, we find that \textbf{Relax-both is the most effective strategy for achieving both efficient and stable training}, thereby validating the design of DPPO in relaxing both ends of the trust region.



\begin{figure}[h]
    \centering
    % \vspace{-0.5em}
    \includegraphics[width=\linewidth]{figs/clip_direction.pdf}
    \caption{Analysis of trust region relaxation direction. (\textbf{Left}) Training reward curves. (\textbf{Right}) Policy entropy.}
    % \vspace{-1.2em}
    \label{fig:clip_direction}
\end{figure}


\begin{figure*}[t]
    \centering  
    \includegraphics[width=\linewidth]{figs/main-base.pdf}  
    % \vspace{-1.8em}
    \caption{Evolution of AIME24 and AIME25 Avg@32 scores during RL training using Qwen3-30B-A3B-Base. 
    % The left two panels show AIME24 results, and the right two panels show AIME25 results. 
    The first and third panels correspond to the same experiment without rollout router replay (w/o R3), while the second and fourth panels correspond to the same experiment with rollout router replay (w/ R3).}
    \label{fig:main_base}  
\end{figure*}  


\begin{figure*}[t]
    \centering  
    \includegraphics[width=\linewidth]{figs/main-a3b_and_8b.pdf} 
    % \vspace{-1.8em}
    \caption{Evolution of AIME24 and AIME25 scores during RL training using Qwen3-30B-A3B (left) and Qwen3-8B-Base (right).}  
    \label{fig:main_a3b_and_8b}  
    % \vspace{-0.6em}
\end{figure*}  


% \vspace{-0.2cm}
\section{Scaling Experiments}
\label{sec:scaling_exp}


\textbf{Experimental Setting:} We conduct large-scale experiments to further validate our methods. We train on a filtered subset of DAPO-Math~\citep{yu2025dapo}, containing approximately 13k samples. Five model configurations (different base models and training techniques) are evaluated: (1) \textbf{MoE Base}: Qwen3-30B-A3B-Base~\citep{yang2025qwen3}; (2) \textbf{MoE Base w/ R3}: Qwen3-30B-A3B-Base with rollout router replay (R3)~\citep{ma2025stabilizing}; (3) \textbf{MoE Thinking}: Qwen3-30B-A3B; (4) \textbf{Dense Base}: Qwen3-8B-Base; (5) \textbf{MoE Base w/ LoRA}: Qwen3-30B-A3B-Base with LoRA~\citep{hu2022lora}. Baseline methods include \textbf{GRPO-ClipHigher}\citep{shao2024deepseekmath, liu2025understanding, yu2025dapo} and \textbf{CISPO}\citep{chen2025minimax, khatri2025art}. 
All methods use the behavior policy ($\rolloutpi_{\theta'}$) instead of recomputed policy distribution ($\trainerpi_{\theta'}$) to construct the trust region (i.e., for clipping or masking). We compare our proposed methods, \textbf{DPPO-Binary-KL} and \textbf{DDPO-Binary-TV}, against these baselines. More details are provided in Appendix~\ref{appendix:detailed_experimental_settings}.

\textbf{Main Results.}
We present online evaluation results on AIME24 and AIME25 \citep{AIME} during RL training in the following figures: \Cref{fig:main_base} (MoE Base with and without R3) and \Cref{fig:main_a3b_and_8b} (MoE Thinking and Dense Base). Results for MoE Base with LoRA are provided in Appendix~\ref{appendix:extended_main_results}.

Our proposed method consistently demonstrates superior \textbf{stability} and \textbf{efficiency} across all five large-scale experiments. Specifically, DPPO optimizes rewards at a significantly faster speed than the GRPO-ClipHigher baseline and achieves better converged performance, providing empirical validation for the motivations discussed in \Cref{sec:method_limitations}. While all baseline methods frequently exhibit training instability or catastrophic collapse (e.g., CISPO in MoE~Base without R3 and GRPO-ClipHigher in MoE~Thinking), our approach maintains a remarkably stable training process.

% Rollout router replay (R3)~\citep{ma2025stabilizing,zheng2025stabilizing, liu2025deepseek} is deemed to be a necessary technique for stabilizing RL training in MoE models. However, as shown in \Cref{fig:main_base}, both of our DPPO variants even \textit{without} R3 consistently outperform baselines \textit{with} R3, highlighting the effectiveness of DPPO in both training efficiency and stability. Additional detailed results and discussions are available in Appendix~\ref{appendix:extended_main_results}.

Rollout router replay (R3) is widely considered a necessary technique for stabilizing RL training in MoE models~\citep{ma2025stabilizing, zheng2025stabilizing, liu2025deepseek}. However, as illustrated in \Cref{fig:main_base}, our DPPO variants (\textit{without} R3) even consistently \textbf{outperform the R3-enhanced baselines}, which underscores the superior training efficiency and inherent stability of the DPPO framework. We provide additional detailed results and extended discussions in Appendix~\ref{appendix:extended_main_results}.


\textbf{Ablation on TV/KL Approximation.}
In the above scaling experiments, DPPO is implemented using the binary TV/KL approximation (Equations~\ref{eq:binary_tv} and \ref{eq:binary_kl}). To assess the impact of this simplification, we compare it against DPPO with the top-K (K=20) TV/KL (Equations~\ref{eq:topk_tv} and \ref{eq:topk_kl}) under the same setting as MoE Base. The results, presented in \Cref{fig:main_topk}, show that both approximations perform similarly and significantly outperform the baselines. This finding indicates that the easy-to-implement binary approximation is a sufficient and computationally efficient choice for scalable RL. We provide more detailed results in Appendix~\ref{appendix:ablation_topk_approximation}.


% \begin{figure}[t]
%     \centering  
%     \includegraphics[width=1.0\linewidth]{figs/main-lora.pdf}  
%     \caption{Evolution of AIME24 and AIME25 scores during RL training with LoRA using Qwen3-30B-A3B-Base.}  
%     \label{fig:main_lora}  
%     % \vspace{-1.4em}
% \end{figure}



\textbf{Generalization to Other Model Families and Tasks.}
We also conduct experiments on models from the Llama family \citep{touvron2023llama,wang2025octothinker} and on tasks beyond math reasoning \citep{liu2025gem}. The results, which are presented in Appendix ~\ref{appendix:extended_models_tasks}, show DPPO outperforms the baseline across most settings, highlighting its broad applicability.



\begin{figure}[t]
    \centering  
    \includegraphics[width=1.0\linewidth]{figs/main-topk.pdf}  
    \caption{Evolution of AIME24 and AIME25 scores for baselines and DPPO with binary/Top-K (K=20) TV/KL approximation under the same setting as MoE Base w/o R3.}  
    \label{fig:main_topk}  
    \vspace{-1.4em}
\end{figure}



% \vspace{-0.2cm}
\section{Conclusion}


In this work, we have presented a comprehensive rethinking of the trust region framework within the context of LLM fine-tuning. We derived policy improvement bounds specifically tailored to the finite-horizon, undiscounted setting of LLM generation, establishing a rigorous theoretical foundation for future trust-region research. Furthermore, through extensive empirical analysis, we investigated the trade-offs between training stability and efficiency, providing practical guidelines to optimize both.

Central to our contribution is the introduction of Divergence Proximal Policy Optimization (DPPO). We identified and addressed a critical structural flaw in the standard PPO algorithm: it over-constrains updates to low-probability tokens while under-constraining potentially catastrophic shifts in high-probability tokens. 
This implicit bias results in a sub-optimal training dynamic, particularly for the expansive, long-tailed vocabularies inherent to LLMs. 
By substituting heuristic ratio clipping with a more principled policy divergence, DPPO significantly enhances both efficiency and stability. To avoid huge memory footprint for computing an exact policy divergence, we introduced Binary and Top-K approximations, which capture essential divergence with negligible overhead. Our evaluations demonstrate that DPPO consistently outperforms existing methods like GRPO in both training efficiency and stability, offering a more robust foundation for the RL-based LLM fine-tuning.



\clearpage


\section*{Impact Statement}

This paper presents work whose goal is to advance the field of Machine
Learning. There are many potential societal consequences of our work, none
which we feel must be specifically highlighted here.

% In the unusual situation where you want a paper to appear in the
% references without citing it in the main text, use \nocite
% \nocite{langley00}

\bibliography{example_paper}
\bibliographystyle{icml2026}

%%%%%%%%%%%%%%%%%%%%%%%%%%%%%%%%%%%%%%%%%%%%%%%%%%%%%%%%%%%%%%%%%%%%%%%%%%%%%%%
%%%%%%%%%%%%%%%%%%%%%%%%%%%%%%%%%%%%%%%%%%%%%%%%%%%%%%%%%%%%%%%%%%%%%%%%%%%%%%%
% APPENDIX
%%%%%%%%%%%%%%%%%%%%%%%%%%%%%%%%%%%%%%%%%%%%%%%%%%%%%%%%%%%%%%%%%%%%%%%%%%%%%%%
%%%%%%%%%%%%%%%%%%%%%%%%%%%%%%%%%%%%%%%%%%%%%%%%%%%%%%%%%%%%%%%%%%%%%%%%%%%%%%%
\newpage
\appendix
\onecolumn

\section{Related Work}
\label{app:related_work}

\subsection{Extended Connections to Existing Work} \label{app:related_work_ratio_clipping}

In this work, we identify a structural flaw in PPO's ratio-clipping mechanism within the LLM regime: it over-penalizes low-probability tokens and under-penalizes high-probability ones, thereby impairing training efficiency and stability. Our proposed DPPO addresses this issue by directly constraining the policy divergence. 
This methodology aligns with the insights of \citet{wang2019trust, wang2020truly}, who observed similar exploration issues and proposed adaptive clipping based on KL divergence in traditional RL settings. However, in the context of LLMs, computing the exact divergence is prohibitive due to the huge memory footprint. To overcome this, we propose a binary divergence approximation, which empirically captures most of the benefits (see Appendix \ref{appendix:ablation_topk_approximation}). Furthermore, as demonstrated in \Cref{sec:stability} and \Cref{sec:training_efficiency}, the challenges of training stability and efficiency are exacerbated in LLMs by their expansive vocabularies, because low-probability tokens form a non-trivial portion of the entire distribution due to the long-tailed nature (see \Cref{fig:moe_prob_ratio_tv}). Finally, the training-inference mismatch inherent to the LLM era introduces additional algorithmic complexities, as further detailed in \Cref{sec:stability}.

\subsection{Training-inference Mismatch} \label{app:related_work_mismatch}
Recent work has identified a key culprit for training instability: the \textit{training-inference mismatch} (${\trainerpi}_\theta \neq {\rolloutpi}_\theta$), where the policy distribution used for gradient computation ($\trainerpi_\theta$) diverges from the one used for data generation ($\rolloutpi_\theta$), even when using identical model parameters $\theta$ \citep{yao2025offpolicy, qi2025defeating, liu-li-2025, zheng2025stabilizing}. This discrepancy arises from numerical precision errors \citep{qi2025defeating} and subtle differences in implementation \citep{Team2025EveryAM, he2025nondeterminism}. As training progresses, this mismatch can be amplified if the RL algorithm cannot manage it appropriately, leading to catastrophic performance degradation~\citep{qi2025defeating, liu-li-2025}.

Existing efforts to mitigate this issue primarily focus on correcting biased gradients through importance sampling. Building on this principle, techniques such as Truncated Importance Sampling (TIS)~\citep{yao2025offpolicy, zheng2025stabilizing} and Masked Importance Sampling \citep{liu-li-2025, team2025every} have been introduced at both the token and sequence levels. However, as suggested by \citet{qi2025defeating}, these methods often fail to achieve a satisfactory balance between training efficiency and stability. In contrast, our DPPO algorithm significantly enhances both aspects compared to these existing approaches.

Another line of research attempts to resolve the mismatch issue through higher precision \citep{qi2025defeating} or rigorous engineering alignment \citep{Team2025EveryAM, he2025nondeterminism, zhang2025deterministic}. While promising, these methods face limited applicability. For instance, aligning implementation details often requires specific training engines or model architectures, hindering broad adoption. Furthermore, in low-precision settings optimized for high-speed training, we must tolerate a significant training-inference mismatch. In such scenarios, a robust and fast algorithm like DPPO remains essential. Finally, our algorithmic design is orthogonal to these engineering-level optimizations and can be combined with them to achieve even greater performance gains.


\section{Trust Region in LLMs}
\label{app:llm_tr_proof}

\subsection{Proof of Performance Difference Identity}

\begin{proof}[Proof of \Cref{lem:llm_identity}]  
We begin by expressing the difference in expected returns by its definition:  
\begin{align*}  
\mathcal{J}(\trainerpi) - \mathcal{J}(\rolloutpi) 
&= \mathbb{E}_{y \sim \trainerpi}[R(y)] - \mathbb{E}_{y \sim \rolloutpi}[R(y)] \\
&= \sum_{y} \big( \trainerpi(y|x) - \rolloutpi(y|x) \big) R(y).
\end{align*}  
The core of the proof is to establish an identity for the difference in the probabilities of generating a sequence $y$, $\trainerpi(y|x) - \rolloutpi(y|x)$. We use the following telescoping sum identity, which can be verified by expanding the terms:  
\begin{align*}  
\trainerpi(y|x) - \rolloutpi(y|x) &= \sum_{t=1}^{T} \left( \prod_{k=1}^{t-1} \rolloutpi(y_k|s_k) \right) \Big( \trainerpi(y_t|s_t) - \rolloutpi(y_t|s_t) \Big) \left( \prod_{j=t+1}^{T} \trainerpi(y_j|s_j) \right).
\end{align*}  
Substituting this identity into the expression for the performance difference yields:  
\begin{align*}  
\mathcal{J}(\trainerpi) - \mathcal{J}(\rolloutpi) &= \sum_{y} R(y) \sum_{t=1}^{T}  \left( \prod_{k=1}^{t-1} \rolloutpi(y_k|s_k) \right) \Big( \trainerpi(y_t|s_t) - \rolloutpi(y_t|s_t) \Big) \left( \prod_{j=t+1}^{T} \trainerpi(y_j|s_j) \right) \\  
&= \sum_{y} \rolloutpi(y|x) R(y) \sum_{t=1}^{T} \left( \frac{\trainerpi(y_t|s_t)}{\rolloutpi(y_t|s_t)} - 1 \right) \left( \prod_{j=t+1}^{T} \frac{\trainerpi(y_j|s_j)}{\rolloutpi(y_j|s_j)} \right) \\ 
&= \mathbb{E}_{y \sim \rolloutpi} \left[ R(y) \sum_{t=1}^{T} \left( \frac{\trainerpi(y_t|s_t)}{\rolloutpi(y_t|s_t)} - 1 \right) \left( \prod_{j=t+1}^{T} \frac{\trainerpi(y_j|s_j)}{\rolloutpi(y_j|s_j)} \right)  \right] \\
% \end{align*}  
% This expression is exact. To derive the final form as stated in the theorem, we add and subtract a term inside the expectation, which corresponds to setting the future policy ratio term, $\frac{\trainerpi(y_{>t}|s_{t+1})}{\rolloutpi(y_{>t}|s_{t+1})}$, to 1.  
% \begin{align*}  
% \mathcal{J}(\trainerpi) - \mathcal{J}(\rolloutpi) 
&= \mathbb{E}_{y \sim \rolloutpi} \left[ R(y) \sum_{t=1}^{|y|} \left( \frac{\trainerpi(y_t|s_t)}{\rolloutpi(y_t|s_t)} - 1 \right) \right] \\  
& \quad - \mathbb{E}_{y \sim \rolloutpi} \left[ R(y) \sum_{t=1}^{|y|} \left( \frac{\trainerpi(y_t|s_t)}{\rolloutpi(y_t|s_t)} - 1 \right) \left( 1 -  \prod_{j=t+1}^{T} \frac{\trainerpi(y_j|s_j)}{\rolloutpi(y_j|s_j)}   \right) \right].  
\end{align*}  
By identifying the terms with the definitions in the theorem statement, we arrive at:  
\begin{equation*}  
\mathcal{J}(\trainerpi) - \mathcal{J}(\rolloutpi) = L_{\rolloutpi}'(\trainerpi) - \Delta(\rolloutpi, \trainerpi),  
\end{equation*}  
where  
\begin{align*}  
L_{\rolloutpi}'(\trainerpi) &= \mathbb{E}_{y \sim \rolloutpi} \left[ R(y) \sum_{t=1}^{|y|} \left( \frac{\trainerpi(y_t|s_t)}{\rolloutpi(y_t|s_t)} - 1 \right) \right], \\  
\Delta(\rolloutpi, \trainerpi) &= \mathbb{E}_{y \sim \rolloutpi} \left[ R(y) \sum_{t=1}^{|y|} \left( \frac{\trainerpi(y_t|s_t)}{\rolloutpi(y_t|s_t)} - 1 \right) \left( 1 - \prod_{j=t+1}^{T} \frac{\trainerpi(y_j|s_j)}{\rolloutpi(y_j|s_j)} \right) \right].  
\end{align*}  
This completes the proof.  
\end{proof}


\subsection{Proof of Policy Improvement Bound}

\begin{lemma}[Bound on Sequence-Level TV Divergence]  
\label{lem:sequence_tv_bound}  
Let $\rolloutpi$ and $\trainerpi$ be two policies that generate sequences of length $N$. Let ${\rolloutpi}_N(\cdot|s_1)$ and ${\trainerpi}_N(\cdot|s_1)$ denote the distributions over sequences $y=(y_1, \dots, y_N)$. The total variation (TV) divergence between these sequence distributions is bounded by the sum of the expected single-step TV divergences:  
\begin{equation*}  
D_{\mathrm{TV}}\big({\rolloutpi}_N(\cdot|s_1)  \|  {\trainerpi}_N(\cdot|s_1)\big) \le \sum_{t=1}^{N} \mathbb{E}_{s_t \sim \rolloutpi} \left[ D_{\mathrm{TV}}\big(\rolloutpi(\cdot|s_t) \| \trainerpi(\cdot|s_t)\big) \right],  
\end{equation*}  
where the expectation is over the state distribution induced by policy $\rolloutpi$.  
\end{lemma}  
  
\begin{proof}  
Let $P(y) = {\rolloutpi}_N(y|s_1)$ and $Q(y) = {\trainerpi}_N(y|s_1)$.  
\begin{equation*}  
    2 D_{\mathrm{TV}}(P \|  Q) = \sum_{y} |P(y) - Q(y)| = \sum_{y} \left| \prod_{t=1}^N \rolloutpi(y_t|s_t) - \prod_{t=1}^N \trainerpi(y_t|s_t) \right|.  
\end{equation*}  
We use the algebraic identity $a_1\dots a_N - b_1\dots b_N = \sum_{t=1}^N \left(\prod_{k=1}^{t-1} a_k\right) (a_t - b_t) \left(\prod_{j=t+1}^N b_j\right)$. Applying this to the policy probabilities and then using the triangle inequality, we get:  
\begin{align*}  
    2 D_{\mathrm{TV}}(P \|  Q) &\le \sum_{y} \sum_{t=1}^N \left(\prod_{k=1}^{t-1} \rolloutpi(y_k|s_k)\right) |\rolloutpi(y_t|s_t) - \trainerpi(y_t|s_t)| \left(\prod_{j=t+1}^N \trainerpi(y_j|s_j)\right) \\  
    &= \sum_{t=1}^N \sum_{y} \left(\prod_{k=1}^{t-1} \rolloutpi(y_k|s_k)\right) |\rolloutpi(y_t|s_t) - \trainerpi(y_t|s_t)| \left(\prod_{j=t+1}^N \trainerpi(y_j|s_j)\right).  
\end{align*}  
For each term in the outer sum over $t$, we can sum over the variables $y_j$ for $j>t$. Since $\sum_{y_j} \trainerpi(y_j|s_j) = 1$ for all $s_j$, the product of terms for $j>t$ sums to 1 when we integrate out $y_{t+1}, \dots, y_N$. This leaves:  
\begin{align*}  
    2 D_{\mathrm{TV}}(P \|  Q) &\le \sum_{t=1}^N \sum_{y_1, \dots, y_t} \left(\prod_{k=1}^{t-1} \rolloutpi(y_k|s_k)\right) |\rolloutpi(y_t|s_t) - \trainerpi(y_t|s_t)| \\  
    &= \sum_{t=1}^N \sum_{y_1, \dots, y_{t-1}} \left(\prod_{k=1}^{t-1} \rolloutpi(y_k|s_k)\right) \sum_{y_t} |\rolloutpi(y_t|s_t) - \trainerpi(y_t|s_t)|.  
\end{align*}  
The inner sum is $2 D_{\mathrm{TV}}(\rolloutpi(\cdot|s_t) \|  \trainerpi(\cdot|s_t))$. The outer sum over $y_1, \dots, y_{t-1}$ defines an expectation over states $s_t$ under policy $\rolloutpi$. Thus, we have:  
\begin{equation*}  
    2 D_{\mathrm{TV}}(P \|  Q) \le \sum_{t=1}^N \mathbb{E}_{s_t \sim \rolloutpi} \left[ 2 D_{\mathrm{TV}}\big(\rolloutpi(\cdot|s_t) \|  \trainerpi(\cdot|s_t)\big) \right].  
\end{equation*}  
Dividing by 2 yields the desired result.  
\end{proof}


\begin{proof}[Proof of \Cref{thm:llm_tr_bound}]  
From Lemma \ref{lem:llm_identity}, we start with the exact performance difference identity:  
\begin{equation*}  
\mathcal{J}(\trainerpi) - \mathcal{J}(\rolloutpi) = L_{\rolloutpi}'(\trainerpi) - \Delta(\rolloutpi, \trainerpi).  
\end{equation*}  
For brevity, we define $y_{\leq t} = \{x, y_1, \dots, y_t \}$ and $y_{>t} = \{y_{t+1}, y_{t+2}, \dots \}$, then we can rewrite $\Delta(\rolloutpi, \trainerpi)$ as:
\begin{equation*}
\Delta(\rolloutpi, \trainerpi) = \mathbb{E}_{y \sim \rolloutpi} \left[ R(y) \sum_{t=1}^{|y|} \left( \frac{\trainerpi(y_t|s_t)}{\rolloutpi(y_t|s_t)} - 1 \right) \left( 1 - \frac{\trainerpi(y_{>t}|s_{t+1})}{\rolloutpi(y_{>t}|s_{t+1})}  \right) \right].  
\end{equation*}
Our goal is to find an upper bound for the error term $\Delta(\rolloutpi, \trainerpi)$. We begin by bounding the reward by its maximum absolute value, $\xi = \max_{y} |R(y)|$.  
\begin{align}  
\label{eq:delta_bound_intermediate}
\begin{split}
\Delta(\rolloutpi, \trainerpi) &\le \xi \cdot \mathbb{E}_{y \sim \rolloutpi} \left[ \sum_{t=1}^{T} \left| \frac{\trainerpi(y_t|s_t)}{\rolloutpi(y_t|s_t)} - 1 \right| \cdot \left| 1 - \frac{\trainerpi(y_{>t}|s_{t+1})}{\rolloutpi(y_{>t}|s_{t+1})} \right| \right] \\  
&= \xi \cdot \sum_{t=1}^{T} \mathbb{E}_{y_{\le t} \sim \rolloutpi} \left[ \left| \frac{\trainerpi(y_t|s_t)}{\rolloutpi(y_t|s_t)} - 1 \right| \cdot \mathbb{E}_{y_{>t} \sim \rolloutpi(\cdot|s_{t+1})} \left[ \left| 1 - \frac{\trainerpi(y_{>t}|s_{t+1})}{\rolloutpi(y_{>t}|s_{t+1})} \right| \right] \right].  
\end{split}
\end{align}  
The inner expectation is exactly twice the TV divergence between the distributions over future trajectories:  
\begin{equation*}  
\mathbb{E}_{y_{>t} \sim \rolloutpi(\cdot|s_{t+1})} \left[ \left| 1 - \frac{\trainerpi(y_{>t}|s_{t+1})}{\rolloutpi(y_{>t}|s_{t+1})} \right| \right] = 2 D_{\mathrm{TV}}\big({\rolloutpi}_{>t}(\cdot|s_{t+1}) \|  {\trainerpi}_{>t}(\cdot|s_{t+1})\big).  
\end{equation*}  
Using \Cref{lem:sequence_tv_bound} on this sequence-level TV divergence (for a sequence of length $T-t$), we get:  
\begin{equation*}  
D_{\mathrm{TV}}\big({\rolloutpi}_{>t}(\cdot|s_{t+1}) \|  {\trainerpi}_{>t}(\cdot|s_{t+1})\big) \le \sum_{k=t+1}^{T} \mathbb{E}_{s_k \sim \rolloutpi(\cdot|s_{t+1})} \left[ D_{\mathrm{TV}}\big(\rolloutpi(\cdot|s_k) \|  \trainerpi(\cdot|s_k)\big) \right].  
\end{equation*}  
We bound each term in the sum by the maximum single-step TV divergence, $D_{\mathrm{TV}}^{\max}(\rolloutpi \|  \trainerpi) = \max_{s} D_{\mathrm{TV}}(\rolloutpi(\cdot|s) \|  \trainerpi(\cdot|s))$, which gives:  
\begin{equation*}  
D_{\mathrm{TV}}\big({\rolloutpi}_{>t}(\cdot|s_{t+1}) \|  {\trainerpi}_{>t}(\cdot|s_{t+1})\big) \le \sum_{k=t+1}^{T} D_{\mathrm{TV}}^{\max}(\rolloutpi \|  \trainerpi) = (T-t) D_{\mathrm{TV}}^{\max}(\rolloutpi \|  \trainerpi).  
\end{equation*}  
Substituting this back into the bound for $\Delta(\rolloutpi, \trainerpi)$:  
\begin{align}
\label{eq:delta_bound_final}
\begin{split}
\Delta(\rolloutpi, \trainerpi) &\le \xi \cdot \sum_{t=1}^{T} \mathbb{E}_{y_{\le t} \sim \rolloutpi} \left[ \left| \frac{\trainerpi(y_t|s_t)}{\rolloutpi(y_t|s_t)} - 1 \right| \cdot 2(T-t) D_{\mathrm{TV}}^{\max}(\rolloutpi \|  \trainerpi) \right] \\  
&= 2\xi \cdot D_{\mathrm{TV}}^{\max}(\rolloutpi \|  \trainerpi) \sum_{t=1}^{T} (T-t) \mathbb{E}_{s_t \sim \rolloutpi} \left[ \sum_{y_t} \rolloutpi(y_t|s_t) \left| \frac{\trainerpi(y_t|s_t)}{\rolloutpi(y_t|s_t)} - 1 \right| \right] \\
&= 2\xi \cdot D_{\mathrm{TV}}^{\max}(\rolloutpi \|  \trainerpi) \sum_{t=1}^{T} (T-t) \mathbb{E}_{s_t \sim \rolloutpi} \left[ 2 D_{\mathrm{TV}}(\rolloutpi(\cdot|s_t) \|  \trainerpi(\cdot|s_t)) \right] \\
&\le 2\xi \cdot D_{\mathrm{TV}}^{\max}(\rolloutpi \|  \trainerpi) \sum_{t=1}^{T} (T-t) \cdot \mathbb{E}_{s_t \sim \rolloutpi} \left[ 2 D_{\mathrm{TV}}^{\max}(\rolloutpi \|  \trainerpi) \right] \\  
&= 4\xi \cdot {D_{\mathrm{TV}}^{\max}(\rolloutpi \|  \trainerpi)}^2 \sum_{t=1}^{T} (T-t) \\
&= 2\xi T(T-1) \cdot {D_{\mathrm{TV}}^{\max}(\rolloutpi \|  \trainerpi)}^2.  
\end{split}
\end{align}  
Substituting this into the performance difference identity gives the desired result:  
\begin{equation*}  
\mathcal{J}(\trainerpi) - \mathcal{J}(\rolloutpi) \ge L_{\rolloutpi}'(\trainerpi) - 2\xi T(T-1) \cdot {D_{\mathrm{TV}}^{\max}(\rolloutpi \|  \trainerpi)}^2.  
\end{equation*}  
This completes the proof.  
\end{proof}

\subsection{A Tighter Policy Improvement Bound}  
\label{app:tighter_bound}  
  
The policy improvement bound derived in \Cref{eq:delta_bound_final} suffers from a quadratic dependence on the horizon length, $T^2$. This makes the bound excessively loose for typical LLM fine-tuning tasks where sequences can be very long. By leveraging the property that the total variation divergence is always bounded by one, i.e., $D_{\mathrm{TV}}(P \|  Q) \le 1$, we can derive an alternative bound that is only linear in $T$, offering a much tighter and more practical guarantee for long-horizon problems.  
  
We begin from the intermediate step in \Cref{eq:delta_bound_intermediate}:  
\begin{equation*}  
\Delta(\rolloutpi, \trainerpi) \le \xi \cdot \sum_{t=1}^{T} \mathbb{E}_{y_{\le t} \sim \rolloutpi} \left[ \left| \frac{\trainerpi(y_t|s_t)}{\rolloutpi(y_t|s_t)} - 1 \right| \cdot \mathbb{E}_{y_{>t} \sim \rolloutpi(\cdot|s_{t+1})} \left[ \left| 1 - \frac{\trainerpi(y_{>t}|s_{t+1})}{\rolloutpi(y_{>t}|s_{t+1})} \right| \right] \right].  
\end{equation*}  
The inner expectation is exactly twice the TV divergence between the future trajectory distributions, $2 D_{\mathrm{TV}}\big({\rolloutpi}_{>t}(\cdot|s_{t+1}) \|  {\trainerpi}_{>t}(\cdot|s_{t+1})\big)$. Instead of bounding this term with $2 (T-t) D_{\mathrm{TV}}^{\max}(\rolloutpi \|  \trainerpi)$, we now apply the simple upper bound of 2:  
\begin{align}  
\Delta(\rolloutpi, \trainerpi) &\le \xi \cdot \sum_{t=1}^{T} \mathbb{E}_{y_{\le t} \sim \rolloutpi} \left[ \left| \frac{\trainerpi(y_t|s_t)}{\rolloutpi(y_t|s_t)} - 1 \right| \cdot 2 D_{\mathrm{TV}}\big({\rolloutpi}_{>t}(\cdot|s_{t+1}) \|  {\trainerpi}_{>t}(\cdot|s_{t+1})\big) \right] \nonumber \\  
&\le 2\xi \cdot \sum_{t=1}^{T} \mathbb{E}_{y_{\le t} \sim \rolloutpi} \left[ \left| \frac{\trainerpi(y_t|s_t)}{\rolloutpi(y_t|s_t)} - 1 \right| \right] \label{eq:tighter_bound_step1} \\  
&= 2\xi \cdot \sum_{t=1}^{T} \mathbb{E}_{s_t \sim \rho_t^{\rolloutpi}} \mathbb{E}_{y_t \sim \rolloutpi(\cdot|s_t)} \left[ \left| \frac{\trainerpi(y_t|s_t)}{\rolloutpi(y_t|s_t)} - 1 \right| \right] \nonumber \\  
&= 2\xi \cdot \sum_{t=1}^{T} \mathbb{E}_{s_t \sim \rho_t^{\rolloutpi}} \left[ 2 D_{\mathrm{TV}}(\rolloutpi(\cdot|s_t) \|  \trainerpi(\cdot|s_t)) \right] \nonumber \\  
&= 4\xi \cdot \mathbb{E}_{y \sim \rolloutpi} \left[ \sum_{t=1}^{|y|} D_{\mathrm{TV}}(\rolloutpi(\cdot|s_t) \|  \trainerpi(\cdot|s_t)) \right]. \label{eq:tighter_bound_final}  
\end{align}  
This provides a bound that is linear in the expected sum of single-step divergences. By combining this with our original quadratic bound from \Cref{eq:delta_bound_final}, we can form a tighter, composite bound by taking the minimum of the two:  
\begin{align*}  
\mathcal{J}(\trainerpi) - \mathcal{J}(\rolloutpi) &\ge L_{\rolloutpi}'(\trainerpi) - \Delta(\rolloutpi, \trainerpi) \\  
&\ge L_{\rolloutpi}'(\trainerpi) - \min \left( 2\xi T(T-1) \cdot {D_{\mathrm{TV}}^{\max}}^2, 4\xi \cdot \mathbb{E}_{y \sim \rolloutpi} \left[ \sum_{t=1}^{|y|} D_{\mathrm{TV}}(\rolloutpi(\cdot|s_t) \|  \trainerpi(\cdot|s_t)) \right] \right).  
\end{align*}  
This composite bound provides a more robust guarantee on policy improvement, leveraging the quadratic bound for infinitesimal updates and the linear bound for larger updates or longer horizons.

\subsection{Comparing Surrogate Objectives with Classical RL}  
\label{app:compare_classical_rl}  
  
At first glance, the surrogate objective for the LLM regime in \Cref{eq:llm_surrogate} appears distinct from the classical RL surrogate in \Cref{eq:surrogate}. The former is an expectation over full trajectories $y$ weighted by the reward $R(y)$, while the latter is an expectation over state-action pairs $(s,a)$ weighted by the advantage $A^{\rolloutpi}(s,a)$. However, we will now show that their gradients with respect to the policy parameters $\theta$ are fundamentally analogous, confirming that our LLM-specific formulation is a valid adaptation of the standard policy gradient theorem.  
  
Let the policy $\trainerpi$ be parameterized by $\theta$. We will use the identity $\nabla_\theta \pi_\theta(a|s) = \pi_\theta(a|s) \nabla_\theta \log \pi_\theta(a|s)$.  
  
\textbf{Gradient of the Classical Surrogate Objective.}  
We begin with the classical surrogate objective from \Cref{eq:surrogate}:  
\begin{equation*}  
L_{\rolloutpi}({\trainerpi}_\theta) = \frac{1}{1 - \gamma}  \mathbb{E}_{s \sim \rho^{\rolloutpi},\,a \sim \rolloutpi(a|s)} \left[ \frac{\trainerpi_\theta(a| s)}{\rolloutpi(a| s)} A^{\rolloutpi}(s,a) \right].  
\end{equation*}  
Taking the gradient with respect to $\theta$ and moving it inside the expectation, we get:  
\begin{align}  
\label{eq:grad_classical}
\begin{split}
\nabla_\theta L_{\rolloutpi}({\trainerpi}_\theta) &= \frac{1}{1 - \gamma} \mathbb{E}_{s \sim \rho^{\rolloutpi},\,a \sim \rolloutpi(a|s)} \left[ \frac{\nabla_\theta \trainerpi_\theta(a| s)}{\rolloutpi(a| s)} A^{\rolloutpi}(s,a) \right] \\  
&= \frac{1}{1 - \gamma} \mathbb{E}_{s \sim \rho^{\rolloutpi},\,a \sim \rolloutpi(a|s)} \left[ \frac{\trainerpi_\theta(a| s)}{\rolloutpi(a| s)} \nabla_\theta \log {\trainerpi}_\theta(a| s) A^{\rolloutpi}(s,a) \right].  
\end{split}
\end{align}  
\textbf{Gradient of the LLM Surrogate Objective.}  
Next, we consider our LLM-specific surrogate from \Cref{eq:llm_surrogate}:  
\begin{equation*}  
L_{\rolloutpi}'({\trainerpi}_\theta) = \mathbb{E}_{y \sim \rolloutpi} \left[ R(y) \sum_{t=1}^{|y|} \left( \frac{\trainerpi_\theta(y_t|s_t)}{\rolloutpi(y_t|s_t)} - 1 \right) \right].  
\end{equation*}  
Taking the gradient with respect to $\theta$ and noting that the $-1$ term has a zero gradient:  
\begin{align*}  
\nabla_\theta L_{\rolloutpi}'({\trainerpi}_\theta) &= \mathbb{E}_{y \sim \rolloutpi} \left[ R(y) \sum_{t=1}^{|y|} \frac{\nabla_\theta \trainerpi_\theta(y_t|s_t)}{\rolloutpi(y_t|s_t)} \right] \\  
&= \mathbb{E}_{y \sim \rolloutpi} \left[ \sum_{t=1}^{|y|} \frac{\trainerpi_\theta(y_t|s_t) }{\rolloutpi(y_t|s_t)} \nabla_\theta \log {\trainerpi}_\theta(y_t|s_t) R(y) \right].  
\end{align*}  
If we define a sequence-level advantage as $A^{\rolloutpi}(s_t, y_t) = R(y) - V(x)$, where $V(x)$ is a baseline value function for the prompt, the gradient becomes:  
\begin{equation} 
\label{eq:grad_llm_surrogate}
\nabla_\theta L_{\rolloutpi}'({\trainerpi}_\theta) = \mathbb{E}_{y \sim \rolloutpi} \left[ \sum_{t=1}^{|y|}  \frac{\trainerpi_\theta(y_t|s_t) }{\rolloutpi(y_t|s_t)} \nabla_\theta \log {\trainerpi}_\theta(y_t|s_t) A^{\rolloutpi}(s_t, y_t) \right].  
\end{equation}  
This form is directly analogous to the classical policy gradient in \Cref{eq:grad_classical}, where the sum over timesteps in a trajectory replaces the expectation over the state distribution $\rho^{\rolloutpi}$. Thus, our LLM surrogate objective is a theoretically sound adaptation of the classical trust region framework to the undiscounted, sequence-reward setting.



\section{Approximations as Lower Bounds of True Divergence}  
\label{app:divergence_lower_bounds}  
  
In this section, we provide a formal justification for our Binary and Top-K divergence approximations. We demonstrate that both are principled lower bounds on the true divergence and explicitly state the conditions under which these approximations become exact.  
  
Let $\mathcal{C} = \{C_1, \dots, C_m\}$ be any partition of the vocabulary $\mathcal{A}$. Our Binary and Top-K approximations correspond to specific choices of this partition. We will show that the divergence computed on the partitioned space is a lower bound on the true divergence.  
  
\subsection{Total Variation Divergence}  
  
The true TV divergence is $D_{\mathrm{TV}}(\rolloutpi \|  \trainerpi) = \frac{1}{2} \sum_{a \in \mathcal{A}} |\rolloutpi(a|s_t) - \trainerpi(a|s_t)|$. The divergence on a partitioned space $\mathcal{C}$ is $D_{\mathrm{TV}}^{\mathcal{C}} = \frac{1}{2} \sum_{j=1}^m |\rolloutpi(C_j|s_t) - \trainerpi(C_j|s_t)|$.  
  
\textbf{Proof of Lower Bound.}  
By definition, $|\rolloutpi(C_j|s_t) - \trainerpi(C_j|s_t)| = |\sum_{a \in C_j} (\rolloutpi(a|s_t) - \trainerpi(a|s_t))|$. The triangle inequality states that the absolute value of a sum is less than or equal to the sum of the absolute values. Applying this, we get $|\sum_{a \in C_j} (\rolloutpi(a|s_t) - \trainerpi(a|s_t))| \le \sum_{a \in C_j} |\rolloutpi(a|s_t) - \trainerpi(a|s_t)|$. Summing over all partitions $j$:  
\begin{align*}  
D_{\mathrm{TV}}^{\mathcal{C}} &= \frac{1}{2} \sum_{j=1}^m \left| \sum_{a \in C_j} (\rolloutpi(a|s_t) - \trainerpi(a|s_t)) \right| \\  
&\le \frac{1}{2} \sum_{j=1}^m \sum_{a \in C_j} |\rolloutpi(a|s_t) - \trainerpi(a|s_t)| = D_{\mathrm{TV}}(\rolloutpi \|  \trainerpi).  
\end{align*}  
Thus, $D_{\mathrm{TV}}(\rolloutpi \|  \trainerpi) \ge D_{\mathrm{TV}}^{\mathcal{C}}$. This holds for both Binary and Top-K partitions.  
  
\textbf{Analysis of the Approximation Gap.}  
The gap between the true and approximated TV divergence is the sum of the gaps within each partition. For any partition $C_j$, the gap is $\frac{1}{2} \left( \sum_{a \in C_j} |\rolloutpi(a|s_t) - \trainerpi(a|s_t)| - \left|\sum_{a \in C_j} (\rolloutpi(a|s_t) - \trainerpi(a|s_t))\right| \right)$. This gap is bounded by the total probability mass of the partition:  
\begin{equation*}  
\text{Gap}(C_j) \le \frac{1}{2} \sum_{a \in C_j} (\rolloutpi(a|s_t) + \trainerpi(a|s_t)) = \frac{1}{2} (\rolloutpi(C_j|s_t) + \trainerpi(C_j|s_t)).  
\end{equation*}  
For the Top-K approximation, the only partition with a potential gap is the "other" category, which contains the tail of the distribution. The total probability mass of this tail, $\rolloutpi(C_{\text{other}}|s_t)$, is typically very small. Therefore, the approximation gap is also small, justifying Top-K TV as a high-fidelity approximation.  
  
\textbf{Equality Condition.}  
Equality $D_{\mathrm{TV}} = D_{\mathrm{TV}}^{\mathcal{C}}$ holds if the gap is zero for all partitions. This occurs when $\rolloutpi(a|s_t) - \trainerpi(a|s_t)$ has the same sign for all tokens $a$ within each partition $C_j$.  
  
\subsection{KL Divergence}  
  
The true KL divergence is $D_{\mathrm{KL}}(\rolloutpi \| \trainerpi) = \sum_{a \in \mathcal{A}} \rolloutpi(a|s_t) \log \frac{\rolloutpi(a|s_t)}{\trainerpi(a|s_t)}$. The divergence on the partitioned space is $D_{\mathrm{KL}}^{\mathcal{C}} = \sum_{j=1}^m \rolloutpi(C_j|s_t) \log \frac{\rolloutpi(C_j|s_t)}{\trainerpi(C_j|s_t)}$.  
  
\textbf{Proof of Lower Bound.}  
The proof relies on the log-sum inequality, which states that for any two sets of non-negative numbers $\{x_1, \dots, x_n\}$ and $\{y_1, \dots, y_n\}$:  
\begin{equation*}  
\sum_{i=1}^n x_i \log \frac{x_i}{y_i} \ge \left(\sum_{i=1}^n x_i\right) \log \frac{\sum_{i=1}^n x_i}{\sum_{i=1}^n y_i}.  
\end{equation*}  
We apply this inequality to each partition $C_j$ in our vocabulary, setting $x_a = \rolloutpi(a|s_t)$ and $y_a = \trainerpi(a|s_t)$:  
\begin{align*}  
\sum_{a \in C_j} \rolloutpi(a|s_t) \log \frac{\rolloutpi(a|s_t)}{\trainerpi(a|s_t)} &\ge \left(\sum_{a \in C_j} \rolloutpi(a|s_t)\right) \log \frac{\sum_{a \in C_j} \rolloutpi(a|s_t)}{\sum_{a \in C_j} \trainerpi(a|s_t)} \\  
&= \rolloutpi(C_j|s_t) \log \frac{\rolloutpi(C_j|s_t)}{\trainerpi(C_j|s_t)}.  
\end{align*}  
Summing over all partitions $j$ gives the desired result:  
\begin{align*}  
D_{\mathrm{KL}}(\rolloutpi \| \trainerpi) &= \sum_{j=1}^m \sum_{a \in C_j} \rolloutpi(a|s_t) \log \frac{\rolloutpi(a|s_t)}{\trainerpi(a|s_t)} \\  
&\ge \sum_{j=1}^m \rolloutpi(C_j|s_t) \log \frac{\rolloutpi(C_j|s_t)}{\trainerpi(C_j|s_t)} = D_{\mathrm{KL}}^{\mathcal{C}}.  
\end{align*}  
  
\textbf{Equality Condition.}  
The log-sum inequality holds with equality if and only if the ratio $\frac{x_i}{y_i}$ is constant for all $i$. In our context, this means that for each partition $C_j$, the ratio $\frac{\rolloutpi(a|s_t)}{\trainerpi(a|s_t)}$ must be constant for all tokens $a \in C_j$. For both Binary and Top-K approximations, this implies the policy update must scale the probabilities of all tokens within the "other" category by a uniform factor.


\section{More Details for Stability Analysis}
\label{app:sanity_test}

Our experimental setup strictly follows the sanity test established in \citet{qi2025defeating}. Each policy iteration begins by sampling a batch of 64 questions. For each question, we generate 8 responses (rollouts) using a maximum context length of 8,000. The collected data is then used to perform 4 gradient steps. All experiments are conducted using the VeRL framework \citep{sheng2024hybridflow} together with the ODC optimization \citep{wan2026revisiting}, and models are trained in BFloat16 precision to better expose potential numerical instabilities between algorithms. For the evaluation on AIME, we sample 32 responses for each test question to ensure a robust assessment.

\subsection{Algorithmic Details for Stability Analysis}  
\label{app:sanity_test_details}  
  
In this section, we provide the specific policy gradient formulations for each algorithm evaluated in our stability analysis (\Cref{sec:stability}). To facilitate a direct comparison, we show how each algorithm's gradient update can be interpreted through the lens of a single, unified framework.  
  
\textbf{A Unified Policy Gradient Formulation.}  
The policy gradient for the algorithms we tested can be generalized into the following form, where the gradient of the objective $L(\theta)$ is expressed as:  
\begin{equation}  
\label{eq:unified_grad}  
\nabla_\theta L(\theta) = \mathbb{E}_{y \sim \rolloutpi_{\theta'}} \left[ \sum_{t=1}^{|y|} M_t \cdot \min \left( \frac{\trainerpi_\theta(y_t|s_t)}{\rolloutpi_{\theta'}(y_t|s_t)}, C \right) \cdot \hat{A}_t \cdot \nabla_\theta \log {\trainerpi}_\theta(y_t|s_t) \right].  
\end{equation}  
In this formulation, $\hat{A}_t$ is the advantage, estimated following the GRPO method but without standard deviation normalization \citep{shao2024deepseekmath, liu2025understanding}. The algorithms differ primarily in their definition of the binary mask $M_t$ and the clipping bound $C$.  
  
\begin{itemize}  
    \item For \textbf{PG-IS}, we have $M_t = 1$ and $C = \infty$.  
    \item For \textbf{PG-TIS (CISPO)}, we have $M_t = 1$ and $C = 3$.  
    \item For \textbf{GRPO}, the mask $M_t$ is the PPO-style clipping mask, and $C = \infty$.  
    \item For \textbf{MiniRL}, the mask $M_t$ is also a PPO-style clipping mask but is conditioned on a recomputed policy ratio. For this algorithm, $C = \infty$.  
    \item For \textbf{MiniRL-TIS}, the mask $M_t$ is the same as in MiniRL, but with $C = 3$.  
    \item For \textbf{DPPO (Ours)}, the mask $M_t$ is conditioned on the policy divergence, and $C = \infty$.  
\end{itemize}  
  
\textbf{Mask Definitions.}  
The specific forms of the masks are as follows:  
\begin{itemize}  
    \item For \textbf{GRPO}, the mask uses the rollout ratio $r_t = \frac{\trainerpi_\theta(y_t|s_t)}{\rolloutpi_{\theta'}(y_t|s_t)}$ and experimental hyperparameters $\epsilon_\text{high}=0.28, \epsilon_\text{low}=0.2$:  
    \begin{equation*}  
    M_t =  
        \begin{cases}  
            0, & \text{if } (\hat{A}_t > 0 \text{ and } r_t > 1 + \epsilon_\text{high}) \text{ or } (\hat{A}_t < 0 \text{ and } r_t < 1 - \epsilon_\text{low}) \\  
            1, & \text{otherwise}.  
        \end{cases}  
    \end{equation*}  
  
    \item For \textbf{MiniRL} and \textbf{MiniRL-TIS}, the mask is structurally identical to GRPO's and uses the same hyperparameters, but it is conditioned on the recomputed ratio $r'_t = \frac{\trainerpi_\theta(y_t|s_t)}{\trainerpi_{\theta'}(y_t|s_t)}$:  
    \begin{equation*}  
    M_t =  
        \begin{cases}  
            0, & \text{if } (\hat{A}_t > 0 \text{ and } r'_t > 1 + \epsilon_\text{high}) \text{ or } (\hat{A}_t < 0 \text{ and } r'_t < 1 - \epsilon_\text{low}) \\  
            1, & \text{otherwise}.  
        \end{cases}  
    \end{equation*}  
  
    \item For \textbf{DPPO}, our mask is conditioned on the policy divergence $D_t$:  
    \begin{equation*}  
    M_t =  
        \begin{cases}  
            0, & \text{if } (\hat{A}_t > 0 \text{ and } r_t > 1 \text{ and } D_t > \delta) \text{ or } (\hat{A}_t < 0 \text{ and } r_t < 1 \text{ and } D_t > \delta) \\  
            1, & \text{otherwise}.  
        \end{cases}  
    \end{equation*}  
    In our experiments, we set the divergence threshold $\delta=0.15$ for TV divergence and $\delta=0.05$ for KL divergence.  
\end{itemize}

% \subsection{The Unexpected Harm of Truncated Importance Sampling}  
  
% Our results also reveal a surprising finding regarding Truncated Importance Sampling (TIS), a technique widely adopted to control the variance of policy gradient estimates \citep{yao2025offpolicy, chen2025minimax}. Contrary to its intended purpose, TIS consistently degrades training stability in our experiments. As shown in \Cref{fig:sanity_test}, the TIS-enabled variants (PG-TIS (CISPO) and MiniRL-TIS) collapse significantly earlier and perform much worse than their untruncated counterparts.  
  
% We hypothesize that this detrimental effect stems from the same core issue as PPO's ratio clipping: low-probability tokens, which naturally produce high-variance ratios, are the most likely to be truncated by TIS. While this does reduce variance, it systematically down-weights the gradient signal from these tokens, introducing a significant and harmful bias into the policy update. This suggests that naive truncation can be just as damaging as naive clipping.


\section{Characterizing Clipped Tokens}  
\label{app:clipped_tokens}  
  
To understand the practical consequences of ratio clipping, we analyzed which tokens are most frequently penalized by a PPO-style algorithm. We trained a Qwen3-4B-Base model on the DAPO dataset with GRPO and, at training step 50, collected two sets of tokens:  
\begin{itemize}  
    \item \textbf{Clipped Positive Tokens:} From samples with $\hat{A}_t > 0$, tokens whose updates were blocked due to a high ratio ($r_t > 1.28$).  
    \item \textbf{Clipped Negative Tokens:} From samples with $\hat{A}_t < 0$, tokens whose updates were blocked due to a low ratio ($r_t < 0.8$).  
\end{itemize}  
  
The 50 most frequent tokens in each category reveal a striking pattern. Far from being random noise, the clipped tokens are often critical for task performance. The lists for both positive and negative samples are dominated by two key categories:  
\begin{enumerate}  
    \item \textbf{Numerical and Mathematical Tokens:} A significant portion of the clipped tokens are numbers (e.g., `\textcolor{red}{1}', `\textcolor{red}{4}') and mathematical symbols (e.g., `\textcolor{red}{+}', `\textcolor{red}{=}', `\textcolor{red}{div}').  
    \item \textbf{Reasoning and Structural Words:} The list also includes many words essential for logical exposition, such as `\textcolor{blue}{Wait}', `\textcolor{blue}{Next}', `\textcolor{blue}{Thus}', and `\textcolor{blue}{Since}'.  
\end{enumerate}
  
These findings highlight a fundamental flaw in ratio-based clipping. For positive samples, it blocks beneficial updates to tokens that are integral to constructing correct solutions. For negative samples, it blocks the necessary suppression of these same tokens when they are part of an incorrect reasoning path. By systematically interfering with the learning signal for these high-utility tokens, the algorithm inadvertently slows learning, stifles exploration, and hinders the model's ability to refine its problem-solving capabilities.


\begin{AIbox}{The 50 most frequently clipped tokens from \textbf{positively-rewarded} samples.}
\begin{lstlisting}
' the', ' \\(', '<@\textcolor{red}{1}@>', '<@\textcolor{blue}{Let}@>', ' in', ' ', ',', 'We', ' <@\textcolor{red}{+}@>', ' \\', ' numbers', ':\n\n', '<@\textcolor{blue}{Wait}@>', '<@\textcolor{red}{4}@>', '<@\textcolor{red}{6}@>', ' Identify', '(', '<@\textcolor{blue}{Next}@>', ' from', ')', ' k', ' <@\textcolor{red}{-}@>', '<@\textcolor{blue}{Since}@>', ' solve', '\\[', ' how', ' ->', ' to', ' are', '<@\textcolor{red}{Sub}@>', 'I', '):\n', '  \n\n', ' spiral', ' <@\textcolor{blue}{Instead}@>', ' this', '<@\textcolor{blue}{If}@>', '<@\textcolor{red}{div}@>', ' Conditions', ' vector', ' have', ' <@\textcolor{red}{=}@>', ' feasible', 'Or', ' inconsistency', ' express', '_{', ' increase', ' exact', ' consider'
\end{lstlisting}
\end{AIbox}



\begin{AIbox}{The 50 most frequently clipped tokens from \textbf{negatively-rewarded} samples.}
\begin{lstlisting}
' \\(', ' the', ',', ' a', ' \\', ' ', '<@\textcolor{red}{2}@>', '<@\textcolor{red}{1}@>', ':\n\n', '<@\textcolor{red}{0}@>', '<@\textcolor{red}{3}@>', ' and', ' (', ' that', '<@\textcolor{red}{-}@>', ' to', '<@\textcolor{red}{5}@>', ' of', '<@\textcolor{blue}{However}@>', '\\', ' is', ' <@\textcolor{red}{=}@>', '<@\textcolor{red}{4}@>', ' in', ' for', ' all', ' we', 'We', ')', '.\n\n', ' our', '.', ':\n', ' <@\textcolor{blue}{but}@>', ' with', '<@\textcolor{blue}{So}@>', ' both', 'From', ' <@\textcolor{blue}{Let}@>', ' this', '<@\textcolor{blue}{Thus}@>', '<@\textcolor{blue}{Wait}@>', ' if', ' <@\textcolor{red}{-}@>', ' <@\textcolor{red}{+}@>', '^', ' only', ' at', '<@\textcolor{blue}{Since}@>', ' integer'
\end{lstlisting}
\end{AIbox}



% \begin{AIbox}{The 50 most frequently clipped tokens from \textbf{positively-rewarded} samples.}
% \begin{lstlisting}
% ' the', ' \\(', ' \\', '\\', ',', ' <@\textcolor{red}{+}@>', '.\n\n', '<@\textcolor{red}{2}@>', ':\n\n', '<@\textcolor{red}{1}@>', ' (', ' a', ' ', '<@\textcolor{blue}{Given}@>', '<@\textcolor{red}{3}@>', '<@\textcolor{red}{4}@>', ' in', '<@\textcolor{blue}{So}@>', ' and', '<@\textcolor{blue}{Wait}@>', ' <@\textcolor{red}{=}@>', ' from', '.', '<@\textcolor{red}{5}@>', '<@\textcolor{red}{6}@>', ' all', ' to', '<@\textcolor{blue}{Now}@>', ' C', '<@\textcolor{red}{7}@>', ' point', ' invalid', '<@\textcolor{red}{8}@>', ' <@\textcolor{blue}{Since}@>', ' I', ' it', ' This', ' with', '<@\textcolor{red}{frac}@>', ' of', ' not', ' is', ' correct', '}', 'We', ' we', ' x', ' differences', ')', ' step'
% \end{lstlisting}
% \end{AIbox}


% \begin{AIbox}{The 50 most frequently clipped tokens from \textbf{negatively-rewarded} samples.}
% \begin{lstlisting}
% ' the', ' \\(', ' \\', ',', ' ', '<@\textcolor{red}{1}@>', ' a', '<@\textcolor{red}{2}@>', ' is', '.\n\n', ':\n\n', ' we', ':\n', ' and', '<@\textcolor{blue}{Wait}@>', ' in', '<@\textcolor{blue}{Given}@>', ' <@\textcolor{red}{=}@>', '<@\textcolor{red}{0}@>', '<@\textcolor{red}{3}@>', '\\', ' for', ' of', '<@\textcolor{blue}{So}@>', 'This', ' each', ' this', '<@\textcolor{blue}{But}@>', ' can', ' to', '{', ' <@\textcolor{red}{+}@>', ' all', ' with', '<@\textcolor{red}{4}@>', ' (', ')', ' are', '<@\textcolor{red}{9}@>', '<@\textcolor{blue}{Since}@>', '.', '<@\textcolor{blue}{Let}@>', ' find', ' if', ' <@\textcolor{blue}{but}@>', '<@\textcolor{blue}{Now}@>', ' The', ':', ' <@\textcolor{blue}{given}@>', ' it'
% \end{lstlisting}
% \end{AIbox}


\section{More Details for Scaling Experiments}
\label{appendix:detailed_experimental_settings}

In this section, we provide detailed training and evaluation settings of the \textbf{scaling experiments} in \Cref{sec:scaling_exp}. 

\textbf{Training Settings.}
We conduct experiments using the VeRL framework~\citep{sheng2024hybridflow} on NVIDIA H Series GPUs. All methods follow the hyperparameter configurations detailed in \Cref{tab:detailed_experimental_settings}. 
Rollout router replay (R3) \citep{ma2025stabilizing} records the routed experts used in the inference engine and replays them in the training engine, which mitigates the training-inference mismatch and stabilizes RL training for MoE models. We only use R3 in the MoE Base w/ R3 experiment and do not use it in all other experiments.
For experiments that utilize LoRA, as suggested by \citet{schulman2025lora}, we employ a larger learning rate of $1\times 10^{-5}$. For the MoE Base w/ LoRA experiment, we set \texttt{lora\_rank=32} and \texttt{lora\_alpha=64}.

As suggested in \Cref{sec:correct_ancher_for_truct_region}, for all methods, we use the behavior policy ($\rolloutpi_{\theta'}$) instead of recomputed policy distribution ($\trainerpi_{\theta'}$) to construct the trust region (i.e., for clipping or masking). Under the unified policy gradient formulation (Equation~\ref{eq:unified_grad}), the method-specific hyperparameters ($C=5$ by default) are configured as follows:

\begin{itemize}  
    \item For \textbf{GRPO-ClipHigher}, we have 
    \begin{equation*}  
    M_t =  
        \begin{cases}  
            0, & \text{if } (\hat{A}_t > 0 \text{ and } r_t > 1 + \epsilon_\text{high}) \text{ or } (\hat{A}_t < 0 \text{ and } r_t < 1 - \epsilon_\text{low}) \\  
            1, & \text{otherwise}.  
        \end{cases}  
    \end{equation*} 
    where $\epsilon_\text{high}=0.27$ and $\epsilon_\text{low}=0.2$, which follows the hyperparameters used in \citet{zheng2025stabilizing}.

    \item For \textbf{CISPO}, we have $M_t = 1$.

    \item For \textbf{DPPO-Binary-KL} and \textbf{DPPO-Binary-TV}, we have 
    \begin{equation*}  
    M_t =  
        \begin{cases}  
            0, & \text{if } (\hat{A}_t > 0 \text{ and } r_t > 1 \text{ and } D_t > \delta) \text{ or } (\hat{A}_t < 0 \text{ and } r_t < 1 \text{ and } D_t > \delta) \\  
            1, & \text{otherwise}.  
        \end{cases}  
    \end{equation*} 
    where $D_t$ is binary approximation of KL or TV as defined in \Cref{sec:method_binary}. For \textbf{DPPO-Binary-KL}, $\delta=0.05$ for all scaling experiments. For \textbf{DPPO-Binary-TV}, we use $\delta=0.15$ for MoE Base w/ LoRA experiment and $\delta=0.2$ for all other scaling experiments.
\end{itemize}


\begin{table}[h]
    % \vspace{-.4cm}
    % \fontsize{7.5}{9}\selectfont
    % \vspace{-0.9cm}
    % \tabcolsep 2.0pt
    \renewcommand{\arraystretch}{1.0}
    \caption{Detailed RL training hyperparameters of scaling experiments.}
    % \vspace{-0.25cm}
    \label{tab:detailed_experimental_settings}
    \centering
    % \input{tables/qwen3_4b_without_condition_y}
\begin{tabular}{l|cccccc}
\toprule
\textbf{Hyperparameters} & MoE Base & MoE Base w/ R3 & MoE Thinking & Dense Base & MoE Base w/ LoRA \\
\midrule
\texttt{max\_prompt\_length} & 1024 & 1024 & 1024 & 1024 & 1024 \\ 
\texttt{max\_response\_length} & 16384 & 16384 & 16384 & 8000 & 8000 \\ 
\texttt{train\_batch\_size} & 256 & 256 & 256 & 128 & 128 \\ 
\texttt{ppo\_mini\_batch\_size} & 32 & 32 & 32 & 32 & 16\\ 
\texttt{optim.lr} & 1e-6 & 1e-6 & 1e-6 & 1e-6 & 1e-5 \\ 
% \texttt{tis\_imp\_ratio\_cap} & 5 & 5 & 5 & 5 & 5 \\ 
\texttt{rollout.temperature} & 1.0 & 1.0  & 1.0 & 1.0 & 1.0 \\ 
\texttt{rollout.n} & 16 & 16 & 16 & 8 & 8\\ 
\midrule
\textbf{Detailed Results} & \Cref{fig:appendix_base_woR3} & \Cref{fig:appendix_base_wR3} & \Cref{fig:appendix_a3b} & \Cref{fig:appendix_8b} & \Cref{fig:appendix_lora} \\
\bottomrule
\end{tabular}
    % \vspace{-.3cm}
\end{table}

\textbf{Evaluation Settings.}
We perform online evaluation for each method and experimental configuration, monitoring AIME24 and AIME25 scores throughout RL training. Evaluations are conducted every 5 training steps for MoE Base, MoE Base w/ R3, and MoE Thinking, and every 10 steps for Dense Base and MoE Base w/ LoRA.

Across all scaling experiments, we use consistent sampling parameters: \texttt{temperature=0.7}, \texttt{top\_p=0.95}, and \texttt{n=32}. The \texttt{n=32} setting indicates that each question from AIME24 and AIME25 is sampled 32 times, and we report the average scores. The \texttt{max\_response\_length} remains identical to that used during training rollouts.


\section{More Empirical Results}

\subsection{Extended Main Results}
\label{appendix:extended_main_results}

In addition to the results provided in \Cref{sec:scaling_exp}, here we provide more detailed results of the five scaling experiments: \Cref{fig:appendix_base_woR3} for MoE Base w/o R3, \Cref{fig:appendix_base_wR3} for MoE Base w/ R3, \Cref{fig:appendix_a3b} for MoE Thinking, \Cref{fig:appendix_8b} for Dense Base, \Cref{fig:appendix_lora} for MoE Base w/ LoRA. We record the following metrics throughout the RL training: training rewards (denoted as ``\textbf{Rewards}''), \textbf{AIME 2024} Avg@32 scores, \textbf{AIME 2025} Avg@32 scores, mean of $\vert\rolloutpi_{\theta'} - \trainerpi_{\theta'}\vert$ (denoted as ``\textbf{Mean of $\vert \pi-\mu \vert$}''), mean of the response length (denoted as ``\textbf{Response Length}''), and mean of token entropy (denoted as ``\textbf{Entropy}''). 
For clearer visualization, all metrics except AIME24 and AIME25 are smoothed using a Gaussian filter with standard deviation $\sigma=2$. The original unsmoothed curves are shown in the background as shaded regions.


\begin{figure}
    \centering  
    \includegraphics[width=\linewidth]{figs/appendix-base_woR3.pdf}  
    \caption{Evolution of metrics for \textbf{MoE Base w/o R3} experiment (based on Qwen3-30B-A3B-Base, without rollout router replay).}  
    \label{fig:appendix_base_woR3}  
\end{figure}

\begin{figure}
    \centering  
    \includegraphics[width=\linewidth]{figs/appendix-base_wR3.pdf}  
    \caption{Evolution of metrics for \textbf{MoE Base w/ R3} experiment (based on Qwen3-30B-A3B-Base, with rollout router replay).}  
    \label{fig:appendix_base_wR3}  
\end{figure}

Overall, across the five experiments, our method DPPO demonstrates consistent and robust improvements in training rewards, highlighting its \textit{stability} and \textit{efficiency}. On both AIME~24 and AIME~25 benchmarks, DPPO exhibits a clear, stable upward trend during training and maintains superior performance after convergence. The stability of our approach is evidenced by learning curves that generally show less fluctuation compared to baseline methods. Its efficiency is reflected in the rapid increase of training rewards and the strong final performance.

DPPO variants consistently demonstrate healthy training dynamics. The training-inference mismatch (measured by the mean absolute deviation $\vert \pi - \mu\vert$) and policy entropy remain within a stable, proper region throughout RL training. DPPO also effectively increases the generated response length across all scaling experiments, except for MoE Thinking. We note that the model Qwen3-30B-A3B already produces extremely long responses; as our training enforces a maximum length of approximately 16k tokens, RL training naturally shortens responses to fit this constraint.

In contrast, the GRPO-ClipHigher baseline, which relies on the ratio clipping mechanism of PPO, shows lower stability than DPPO and achieves inferior final performance in all five large-scale experiments. For example, in MoE Base w/o R3 (see \Cref{fig:appendix_base_woR3}), GRPO-ClipHigher, though more stable than CISPO, improves more slowly and converges to lower training rewards and AIME scores than DPPO. In MoE Thinking (see \Cref{fig:appendix_a3b}), GRPO-ClipHigher suffers a significant training collapse. Notably, GRPO-ClipHigher consistently leads to excessively high entropy in all large-scale experiments, a phenomenon not observed with other methods.

The CISPO baseline, which retains gradients for all tokens, is generally less stable and prone to collapse in certain settings. For instance, in MoE~Base~w/o~R3 (see \Cref{fig:appendix_base_woR3}), CISPO experiences a sudden and severe collapse leading to complete failure. In Dense Base (see \Cref{fig:appendix_8b}), CISPO shows a degenerative trend, particularly on AIME25. In MoE Base w/ LoRA (see \Cref{fig:appendix_lora}), the AIME24 scores, mean of $\vert \pi - \mu\vert$, and response length exhibit noticeable fluctuations, further indicating instability.

We also analyze the effect of rollout router replay (R3). Remarkably, DPPO variants \textit{without} R3 already outperform baselines that use R3, underscoring the importance of a proper masking mechanism in RL training (see \Cref{fig:appendix_base_woR3,fig:appendix_base_wR3}). Furthermore, incorporating R3 yields additional gains for DPPO, suggesting that the benefits of R3 and DPPO are largely orthogonal. This implies that DPPO provides a robust foundation for LLM RL fine-tuning, capable of further improvement even when training-inference mismatch is mitigated by other techniques.


\begin{figure}
    \centering  
    \includegraphics[width=\linewidth]{figs/appendix-a3b.pdf}  
    \caption{Evolution of metrics for \textbf{MoE Thinking} experiment (based on Qwen3-30B-A3B).}  
    \label{fig:appendix_a3b}  
\end{figure}

\begin{figure}
    \centering  
    \includegraphics[width=\linewidth]{figs/appendix-8b.pdf}  
    \caption{Evolution of metrics for \textbf{Dense Base} experiment (based on Qwen3-8B-Base).}  
    \label{fig:appendix_8b}  
\end{figure}

\begin{figure}
    \centering  
    \includegraphics[width=\linewidth]{figs/appendix-lora.pdf}  
    \caption{Evolution of metrics for \textbf{MoE Base w/ LoRA} experiment (based on Qwen3-30B-A3B-Base, with LoRA).}  
    \label{fig:appendix_lora}  
\end{figure}




\subsection{Ablation on TV/KL Approximation}
\label{appendix:ablation_topk_approximation}

In the scaling experiments, we compared DPPO variants using binary TV/KL approximations (Equations~\ref{eq:binary_tv} and \ref{eq:binary_kl}) against several baselines. To further investigate the approximation strategy, we experiment with DPPO variants with top-K TV/KL approximations (Equations~\ref{eq:topk_tv} and \ref{eq:topk_kl}), where we set $K=20$; these variants are denoted as \textbf{DPPO-TopK-TV} and \textbf{DPPO-TopK-KL}. The choice $K=20$ is limited by vLLM \citep{vllm}, which supports returning log probabilities for at most 20 candidate tokens per step. We strictly replicate the experimental setting of MoE Base w/o R3. As in the main scaling experiments, for \textbf{DPPO-Binary-TV} and \textbf{DPPO-TopK-TV} we set the clip threshold $\delta=0.2$, while for \textbf{DPPO-Binary-KL} and \textbf{DPPO-TopK-KL} we set $\delta=0.05$.


\begin{figure}
    \centering  
    \includegraphics[width=\linewidth]{figs/appendix-topk.pdf}  
    \caption{Evolution of metrics for baselines, DPPO with binary TV/KL approximation, and DPPO with Top-K (K=20) approximation under the same setting as MoE Base w/o R3.}  
    \label{fig:appendix_topk}  
\end{figure}


As presented in \Cref{fig:appendix_topk}, introducing the top-K approximation does not yield significant performance gains, indicating that the simpler binary approximation already provides a sufficient and efficient proxy for constructing the trust region. This finding is encouraging, suggesting that DPPO with binary TV/KL remains highly scalable without sacrificing effectiveness.


\label{appendix:more_settings}
\begin{figure*}[h]
    \centering
    \includegraphics[width=\linewidth]{figs/more_settings.pdf}
    \caption{Learning curve comparison of using ratio (PPO-Ratio) and TV divergence (DPPO-Binary-TV) for the trust region clipping.}
    \label{fig:more_settings}
\end{figure*}

\subsection{Extended Results for Different Model $\times$ Task Combinations}
\label{appendix:extended_models_tasks}


Besides experimental results presented in \Cref{sec:scaling_exp}, we evaluate DPPO on more model $\times$ task settings to validate its advantage over the GRPO baseline. The settings we considered include:

\begin{enumerate}
    \item \textbf{Different model family}. Training on a new model different from the Qwen family, OctoThinker-3B-Hybrid-Base~\citep{wang2025octothinker}, on the standard math reasoning dataset~\citep{hendrycks2021measuring}.

    \item \textbf{Abstract reasoning and induction}. Training the Qwen3-1.7B-Base model on abstract reasoning task (Arc1D) and induction task (Acre) from the Gem library~\citep{liu2025gem}.

    \item \textbf{Multi-turn reasoning}. Training the Qwen3-1.7B-Base model on the multi-turn reasoning environment (Sudoku-v0-easy) from Gem the library~\citep{liu2025gem}.
\end{enumerate}

The training is conducted using Oat~\citep{liu2025oat} with their example scripts (thereby the standard hyper-parameters) for math RL and Gem RL. For the TV divergence clipping, we use a threshold of $\delta=0.2$.
\Cref{fig:more_settings} shows the comparison between the TV variant of DPPO and the vanilla ratio-based PPO, both based on the GRPO algorithmic framework with the only difference being the trust region masking strategy. We can observe DPPO improves the efficiency (and sometimes asymptotic performance) over the baseline across different settings, validating its general effectiveness.

%%%%%%%%%









\end{document}

% This document was modified from the file originally made available by
% Pat Langley and Andrea Danyluk for ICML-2K. This version was created
% by Iain Murray in 2018, and modified by Alexandre Bouchard in
% 2019 and 2021 and by Csaba Szepesvari, Gang Niu and Sivan Sabato in 2022.
% Modified again in 2023 and 2024 by Sivan Sabato and Jonathan Scarlett.
% Previous contributors include Dan Roy, Lise Getoor and Tobias
% Scheffer, which was slightly modified from the 2010 version by
% Thorsten Joachims & Johannes Fuernkranz, slightly modified from the
% 2009 version by Kiri Wagstaff and Sam Roweis's 2008 version, which is
% slightly modified from Prasad Tadepalli's 2007 version which is a
% lightly changed version of the previous year's version by Andrew
% Moore, which was in turn edited from those of Kristian Kersting and
% Codrina Lauth. Alex Smola contributed to the algorithmic style files.

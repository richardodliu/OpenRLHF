\documentclass[11pt, letterpaper, logo, onecolumn, copyright, numbering]{minimax}
\usepackage{etoolbox}

\usepackage[authoryear, sort&compress, round]{natbib}

\usepackage[inkscapeformat=png]{svg}


\usepackage[most, breakable, skins]{tcolorbox}
\usepackage{academicons}

\tcbuselibrary{skins}
\usepackage{lipsum}
\usepackage{tabularx}
\usepackage{afterpage}
\usepackage{booktabs}
\usepackage{subcaption}
\usepackage{makecell}
\usepackage{multirow}
\usepackage{multicol} 
\usepackage{array}
\usepackage{float}
\usepackage{listings, listings-rust}
\usepackage{fontawesome5}
\usepackage{amssymb,graphicx}
\usepackage[dvipsnames]{xcolor}
\usepackage{hyperref}
\usepackage{cleveref}
\usepackage{longtable}
\usepackage{graphicx}
\usepackage{pdflscape}
\usepackage{adjustbox}
\usepackage{tikz}
\usetikzlibrary{calc,positioning,chains,shapes,arrows,fit,decorations.pathmorphing,patterns,fadings,shadows,patterns.meta,arrows.meta}
\usepackage{wrapfig}
\usepackage{dialogue}
\usepackage{algorithm}
\usepackage{algorithmic}
\usepackage{colortbl}
\usepackage{mdframed}


\usepackage{listings}

\usepackage{CJKutf8}
\usepackage{tcolorbox}

\usepackage[dvipsnames]{xcolor}
\usepackage{multicol}

\usepackage{caption}
\captionsetup{justification=justified, singlelinecheck=true}

\input{math_commands.tex}

\input{showcase_format}

\theoremstyle{plain}
\newtheorem{theorem}{Theorem}[section]
\newtheorem{proposition}[theorem]{Proposition}
\newtheorem{lemma}[theorem]{Lemma}
\newtheorem{corollary}[theorem]{Corollary}
\theoremstyle{definition}
\newtheorem{definition}[theorem]{Definition}
\newtheorem{assumption}[theorem]{Assumption}
\theoremstyle{remark}
\newtheorem{remark}[theorem]{Remark}

\usepackage{CJKutf8}

\lstset{
basicstyle=\footnotesize\ttfamily,
columns=flexible,
frame=single,
xleftmargin=1em,
breaklines=true,
breakindent=0em
}

% colours
\definecolor{medgray55}{gray}{0.55}
\definecolor{medgray}{gray}{0.7}
\definecolor{litegray}{gray}{0.9}
\definecolor{gblue}{RGB}{210, 227, 252}
\definecolor{gred}{RGB}{250, 210, 207}
\definecolor{gyellow}{RGB}{254, 239, 195}
\definecolor{ggreen}{RGB}{206, 234, 214}
\definecolor{gorange}{RGB}{254, 223, 200}

\definecolor{gblue9}{RGB}{23, 78, 166}
\definecolor{gred9}{RGB}{165, 14, 14}
\definecolor{gyellow9}{RGB}{227, 116, 0}
\definecolor{ggreen9}{RGB}{13, 101, 45}
\definecolor{gorange9}{RGB}{176, 96, 0}

\definecolor{myblue}{rgb}{0,0,1}
\definecolor{myred}{rgb}{1,0,0}
\definecolor{mylightgray}{gray}{0.95}

\definecolor{highlightblue}{HTML}{185ABC}

\makeatletter

\renewcommand\paragraph{\@startsection{paragraph}{4}{\z@}%
            {-2.5ex\@plus -1ex \@minus -.25ex}%
            {1.25ex \@plus .25ex}%
            {\itshape\normalsize\bfseries}}
\makeatother
\setcounter{secnumdepth}{4} % how many sectioning levels to assign numbers to
\setcounter{tocdepth}{4}    % how many sectioning levels to show in ToC


\newcolumntype{L}[1]{>{\raggedright\let\newline\\\arraybackslash\hspace{0pt}}m{#1}}
\newcolumntype{C}[1]{>{\centering}m{#1}}

\newcolumntype{R}[1]{>{\raggedleft\let\newline\\\arraybackslash\hspace{0pt}}m{#1}}

\definecolor{ao}{rgb}{0.0, 0.0, 1.0}

\newcommand{\sota}[1]{\textcolor{ao}{$\textbf{#1}$}}
\newcommand{\red}[1]{\textcolor{red}{{#1}}}
\newcommand{\green}[1]{\textcolor{green}{{#1}}}
\newcommand\rowincludegraphics[2][]{\raisebox{-0.55\height}{\includegraphics[#1]{#2}}}
\newcommand\vcent[1]{\vcenter{\hbox{#1}}}
\newcommand{\vvr}[1]{\textcolor{red}{[Vinay TODO: #1]}}
\newcommand{\ov}[1]{\textcolor{red}{[OV: #1]}}
\newcommand{\todo}[1]{\textcolor{red}{[TODO: #1]}}
\newcommand\loudspeaker[1][3]{\ensuremath{\vcent{\rule{.6ex}{.6ex}}\kern-.5ex%
  \vcent{\scalebox{.6}[1]{\rotatebox[origin=center]{90}{$\blacktriangle$}}}%
  \ifnum#1>0\relax\kern.05ex\vcent{\scalebox{.4}{\ttfamily)}}%
  \ifnum#1>1\relax\kern-.4ex\vcent{\scalebox{.56}{\ttfamily)}}%
  \ifnum#1>2\relax\kern-.55ex\vcent{\scalebox{.7}{\ttfamily)}}%
  \fi\fi\fi}%
}

\definecolor{green}{rgb}{0.9,0.9,0.9}
\newcommand{\std}[1]{{\tiny\(\pm\)#1}}
\newcommand{\etal}{\textit{et al.}} 

\newcommand{\yr}[1]{\textcolor{red}{uno:#1}}
\newcommand{\jiangwei}[1]{\textcolor{violet}{jiangwei:#1}}
\newcommand{\meish}[1]{\textcolor{blue}{#1}}
\newcommand{\panyq}[1]{\textcolor{violet}{#1}}
\newcommand{\huangyb}[1]{\textcolor{purple}{#1}}
\newcommand{\guixianren}[1]{\textcolor[rgb]{0.545,0.545,0}{[Guixianren: #1]}}
\newcommand{\yuban}[1]{\textcolor[rgb]{1.0,0.75,0.8}{[yuban: #1]}}
\newcommand{\puwang}[1]{\textcolor[rgb]{0.5,0.3,0.9}{[puwang: #1]}}
\newcommand{\jianxin}[1]{\textcolor[rgb]{0.8,0.75,1.0}{[jianxin: #1]}}
\newcommand{\yize}[1]{\textcolor[rgb]{1.0,0.75,0.8}{[yize: #1]}}
\newcommand{\pascal}[1]{\textcolor[rgb]{1.0,0.75,0.8}{[pascal: #1]}}
\newcommand{\ray}[1]{\textcolor[rgb]{0.63, 1.0, 0.4}{[Ray: #1]}}
\newcommand{\qinggangying}[1]{\textcolor[rgb]{0.3,0.8,1.0}{[qinggangying: #1]}}
\newcommand{\donghuang}[1]{\textcolor[rgb]{1.0,0.5,0.5}{[donghuang: #1]}}
\newcommand{\ami}[1]{\textcolor[rgb]{0.222, 0.715, 0.015}{[Ami: #1]}}
\newcommand{\sskip}[1]{\textcolor[rgb]{0.672, 0.061, 0.17}{[Skip: #1]}}
\newcommand{\xiaohui}[1]{\textcolor[rgb]{0.63, 0.7, 0.1}{[Xiaohui: #1]}}
\newcommand{\schon}[1]{\textcolor{brown}{[Schon: #1]}}
\newcommand{\olive}[1]{\textcolor{brown}{[Olive: #1]}}
\newcommand{\lecun}[1]{\textcolor{red}{[Lecun: #1]}}
\newcommand{\kawendixu}[1]{\textcolor{red}{[Kawendixu: #1]}}
\newcommand{\io}[1]{\textcolor[rgb]{0.33, 0.33, 0.33}{[IO: #1]}}
\newcommand{\pengyu}[1]{\textcolor[rgb]{1.0,0.6,0.0}{[Pengyu: #1]}}
\newcommand{\jh}[1]{\textcolor{magenta}{[Xiaoxian: #1]}}

\newcommand{\method}{CISPO}


\makeatletter
\renewcommand\subparagraph{%
 \@startsection {subparagraph}{5}{\z@ }{3.25ex \@plus 1ex
 \@minus .2ex}{-1em}{\normalfont \normalsize \bfseries }}%
\makeatother

\newcommand{\modelname}{MiniMax-01}
\newcommand{\modelnameflash}{MiniMax-01}

\bibliographystyle{plainnat}


\let\cite\citep

\title{MiniMax-M1: Scaling Test-Time Compute Efficiently with Lightning Attention}


\reportnumber{}


\renewcommand{\today}{}


\author[*,1]{MiniMax\footnote{Please send correspondence to model@minimax.io.}}


\begin{abstract}
We introduce MiniMax-M1, the world's first open-weight, large-scale hybrid-attention reasoning model. MiniMax-M1 is powered by a hybrid Mixture-of-Experts (MoE) architecture combined with a lightning attention mechanism. The model is developed based on our previous MiniMax-Text-01 model~\citep{minimax2025minimax01}, which contains a total of 456 billion parameters with 45.9 billion parameters activated per token. The M1 model natively supports a context length of 1 million tokens, 8x the context size of DeepSeek R1. Furthermore, the lightning attention mechanism in MiniMax-M1 enables efficient scaling of test-time compute -- For example, compared to DeepSeek R1, M1 consumes 25\% of the FLOPs at a generation length of 100K tokens. These properties make M1 particularly suitable for complex tasks that require processing long inputs and thinking extensively.
MiniMax-M1 is trained using large-scale reinforcement learning (RL) on diverse problems ranging from traditional mathematical reasoning to sandbox-based, real-world software engineering environments. 
In addition to the inherent efficiency advantage of lightning attention for RL training, we propose \method{}, a novel RL algorithm to further enhance RL efficiency. \method{} clips importance sampling weights rather than token updates, outperforming other competitive RL variants.
Combining hybrid-attention and \method{} enables MiniMax-M1's full RL training on 512 H800 GPUs to complete in only three weeks, with a rental cost of just \$534,700.
We release two versions of MiniMax-M1 models with 40K and 80K thinking budgets respectively, where the 40K model represents an intermediate phase of the 80K training.
Experiments on standard benchmarks show that our models are comparable or superior to strong open-weight models such as the original DeepSeek-R1 and Qwen3-235B, with particular strengths in complex software engineering, tool utilization, and long-context tasks. 
Through efficient scaling of test-time compute, MiniMax-M1 serves as a strong foundation for next-generation language model agents to reason and tackle real-world challenges. We publicly release MiniMax-M1 at \href{https://github.com/MiniMax-AI/MiniMax-M1}{https://github.com/MiniMax-AI/MiniMax-M1}. 

\end{abstract}

\begin{document}




\maketitle


\label{sec:intro}
\begin{figure*}[h]
\centering
\includegraphics[width=0.99\textwidth]{figures/main_results.pdf}
\caption{Performance comparison of \model{} and existing open-weights and close-weights$^{\dagger}$ models across benchmarks.}
\label{fig:ring-lite-performance}
\end{figure*}

\section{Introduction}
\label{sec:intro}

Artificial intelligence is undergoing a pivotal transition: Large Language Models (LLMs) are advancing beyond static corpora of human knowledge, becoming dynamic processors that transform information into actionable insights and understanding~\citep{kimiteam2025kimik2openagentic,deepseekai2025deepseekr1}. This progression towards more general intelligence is empirically validated by their core capability – complex, adaptive problem-solving. 
Recent breakthroughs in solving high-difficulty human competition problems provide concrete evidence of significantly advanced reasoning abilities in large language models. For instance, models~\citep{gpt5, qwen3max} have achieved 100\% accuracy on the AIME-2025~\citep{aime} and HMMT-2025~\footnote{https://www.hmmt.org/www/archive/problems}, and reached medal-level performance at the International Mathematical Olympiad (IMO)~\citep{gpt5}—a hallmark of sophisticated human intellect. This evolution beyond static knowledge repositories is driven by training on trillions of tokens across diverse domains, coupled with reinforcement learning-optimized reasoning techniques~\citep{openai2024openaio1card, deepseekai2025deepseekr1} that enable models to dynamically scale their capabilities with thinking effort, pointing toward higher levels of general intelligence.

While related work~\citep{deepseekai2025deepseekr1,glm46} has made valuable contributions to the open-source community, the frontier of trillion-parameter thinking models remains uncharted territory. Scaling to this level introduces formidable challenges, such as severe training instability and prohibitive computational costs. In this work, we introduce \model{}, a novel Mixture-of-Experts (MoE) thinking model scaled to unprecedented size—and demonstrate breakthrough methodologies for efficient trillion-parameter training. By solving fundamental stability and efficiency challenges at this scale, we enable robust large-scale reasoning training while providing extensive implementation insights.


Our \model{}, the first open-source reasoning model with one trillion total parameters, is built upon the Ling 2.0~\citep{lingv2} architecture and trained from the Ling-1T-base. 
With approximately 50 billion activated parameters per token, \model{} achieves state-of-the-art performance across multiple challenging benchmarks—despite relying solely on natural language reasoning capabilities. It significantly outperforms existing open-source models, achieving scores of 93.4 on AIME-2025, 86.72 on HMMT-2025, 2088 on CodeForces, and 55.94 on ARC-AGI-v1. Remarkably, in the IMO-2025 evaluation within AWorld~\footnote{https://github.com/inclusionAI/AWorld}, \model{} achieved a silver medal-level result by correctly solving four problems and partially proving Problem 2, all within a single submission, and without relying on code generation or external symbolic solvers.
Realizing this breakthrough required addressing fundamental challenges in trillion-scale RL training. We pioneered three interconnected innovations:
\begin{itemize}
    \item \textbf{IcePop} eliminates catastrophic training-inference misalignment in RL training by clipping excessive-discrepancy tokens. This selective correction pops out unstable contributions while preserving efficient updates, thereby stabilizing training without slowing inference.
    \item \textbf{C3PO++} introduces a budget-controlled rollout scheduling mechanism that eliminates rollout-stage bottlenecks. Thus, it avoids inefficient single-pass processing of oversized sequences, reducing computational overhead while enabling their efficient reuse through batched continuation.
    \item \textbf{ASystem} is a high-performance reinforcement learning (RL) framework designed for large-scale asynchronous training. It adopts a SingleController + SPMD (Single Program, Multiple Data) to enable fully asynchronous operations, multi-phase masking acceleration, and efficient data packing/sharding. 
\end{itemize}

The structure of this paper is organized as follows: Section~\ref{sec:method} describes our comprehensive training methodology, which includes Long Chain-of-Thought Supervised Fine-Tuning (Long-CoT SFT) and large-scale reinforcement learning (RL), including our key algorithmic contributions, IcePop and C3PO++, as well as the underlying training framework, Asystem. Finally, Section~\ref{sec:eval} presents a thorough evaluation of our model's performance against leading open-weights and closed-weights models on established benchmarks.


\section{Preparation for Scalable RL: Continual Pretraining and SFT}

In this work, we focus on scaling up reinforcement learning to enhance reasoning capabilities of Minimax-Text-01. To facilitate scalable RL training, we first carry out continual pretraining of our base model to strengthen its intrinsic reasoning abilities. Subsequently, we perform a cold-start supervised fine-tuning (SFT) stage to inject specific reasoning patterns to the model, thereby providing a stronger foundation for the subsequent RL phase.


\subsection{Continual Pre-Training: Foundation for RL Scaling}

To enhance the reasoning and long context capabilities of the foundation model while ensuring diversity, we continue training the MiniMax-Text-01 model with additional 7.5T tokens with optimized data quality and mixture.

\noindent\textbf{Training Data.}
We refine our pretraining Web and PDF parsing mechanisms and enhance our heuristic cleaning rules to ensure a high recall rate for mathematical and code-related data. We prioritize the extraction of natural Question-Answer (QA) pairs from a diverse range of sources, including webpages, forums, and textbooks, while strictly avoiding the use of synthetic data. Additionally, we conduct semantic deduplication on the QA data to maintain its diversity and uniqueness. Furthermore, we increase the proportion of STEM (Science, Technology, Engineering, and Mathematics), code, book, and reasoning-related data to 70\%. This significantly enhances the foundation model's ability to handle complex tasks without compromising its other general capabilities.

\noindent\textbf{Training Recipe.}
We decrease the coefficient of the MoE auxiliary loss and adjust the parallel training strategy to support a larger training micro batch size, which mitigates the detrimental effects of the auxiliary loss on overall model performance. Based on MiniMax-Text-01, we continue training with a constant learning rate of 8e-5 for 2.5T tokens, followed by a decay schedule over 5T tokens down to 8e-6.

\noindent\textbf{Long Context Extension.}
For a hybrid-lightning architecture model with higher convergence complexity, we have observed that excessively aggressive extensions of the training length can lead to a sudden gradient explosion that may occur during the training process. This makes the optimization process extremely challenging. We attribute this to the parameter optimization of the earlier layers not keeping up with the changes in the later layers -- For lightning attention, the earlier and later layers have different decay rates, which makes the earlier layers focus more on local information. We alleviate this issue by adapting a smoother extension of context length across four stages, starting from a 32K context window length and ultimately extending the training context to 1M tokens.  


\subsection{Supervised Fine-Tuning: Focused Alignment for Efficient RL}

After continual pretraining, we conduct Supervised Fine-Tuning (SFT) to instill desired behaviors like reflection-based Chain-of-Thought (CoT) reasoning using high-quality examples, creating a strong starting point for more efficient and stable RL in the next stage. Specifically, we curate data samples with long CoT responses. These data samples cover diverse domains such as math, coding, STEM, writing, QA, and multi-turn chat.  Math and coding samples account for around 60\% of all the data.


\section{Efficient RL Scaling: Algorithms and Lightning Attention} 
As shown in Figure~\ref{fig:perf-bars-flops} (Right), the M1 architecture demonstrates a clear efficiency advantage during inference. This naturally facilitates efficient RL scaling where increasingly longer responses are generated. However, as pioneers in scaling up RL with this hybrid architecture, we encounter unique challenges during the process, and the RL procedure can become unstable or even fail due to various issues.
To address these difficulties, we develop targeted solutions that enable us to successfully scale up RL training for M1. In addition, we propose a new RL algorithm that achieves greater RL efficiency compared to existing methods.
These dual contributions yield an efficient and scalable RL framework for training M1, where the complete training cycle requires 3 weeks on 512 H800 GPUs—equivalent to a rental cost of approximately \$0.53M USD.
In this section, we first provide general context on RL and present our novel RL algorithm, and then describe the specific challenges we face with the hybrid architecture, along with the solutions we devise to overcome them.


\subsection{Efficient RL Scaling with \method{}}
\label{sec:method}

\noindent\textbf{Background.}
For questions $q$ from a dataset $\mathcal{D}$, we denote $\pi$ as the policy model parameterized by $\theta$, and $o$ as the response generated by the policy.
PPO~\citep{schulman2017proximal} adopts the following objective to optimize the policy to maximize the expected return, and a clipping operation is applied to stabilize training: 
\begin{equation}
\begin{aligned}
\mathcal{J}_{\text{PPO}}(\theta) &= \mathbb{E}_{q \sim \mathcal{D}, o_i \sim \pi_{\theta_{\text{old}}}(\cdot|q)} \\
& \left[
     \frac{1}{|o_i|}\sum_{t=1}^{|o_i|} \min\left( r_{i,t}(\theta) \hat{A}_{i,t}, \text{clip}\big(r_{i,t}(\theta), 1 - \epsilon, 1 + \epsilon\big) \hat{A}_{i,t}\right) - \beta D_{KL}(\pi_{\theta} || \pi_{\text{ref}})
\right],
\end{aligned}
\label{eq:grpo_objective} 
\end{equation}
where $r_{i,t}(\theta) = \frac{\pi_\theta(o_{i,t} \mid q, o_{i,<t})}{\pi_{\theta_{\text{old}}}(o_{i,t} \mid q, o_{i,<t})}$ is the importance sampling (IS) weight, which is used to correct the distribution during off-policy updates, because we use $\pi_{\theta_{\text{old}}}$ to collect trajectories to update the policy via multiple steps in a minibatch manner. While PPO requires a separate value model to compute the advantage $\hat{A}_{i,t}$, GRPO~\citep{shao2024deepseekmath} eliminates the value model and defines the advantage as the output reward relative to other responses in the group:
\begin{equation}
\hat{A}_{i,t} = \frac{R_i - \text{mean}(\{R_j\}_{j=1}^G)}{\text{std}(\{R_j\}_{j=1}^G)}, 
\end{equation}
where $R_i$ is the reward of the response, and $G$ responses $\{o_i\}^G_{i=1}$ are sampled for each question. The reward is either from rule-based verifiers such as in mathematical problem solving, or from a reward model. 

\noindent\textbf{Issues of Token Clipping.}
In our initial experiments with the hybrid architecture under the zero-RL setting, we observed that the GRPO algorithm adversely affected training performance and failed to effectively promote the emergence of long CoT reasoning behaviors. Through a series of controlled ablation studies, we ultimately identified the undesirable clipping operation in the original PPO/GRPO loss as the primary factor contributing to degraded learning performance.
Specifically, we found that tokens associated with reflective behaviors (e.g., \texttt{However}, \texttt{Recheck}, \texttt{Wait}, \texttt{Aha}), which often serve as ``forks'' in reasoning paths, were typically rare and assigned low probabilities by our base model. During policy updates, these tokens were likely to exhibit high $r_{i,t}$ values. As a result, these tokens were clipped out after the first on-policy update, preventing them from contributing to subsequent off-policy gradient updates. This issue was particularly pronounced in our hybrid-architecture model and further hindered the scalability of reinforcement learning.
These low-probability tokens, however, are often crucial for stabilizing entropy~\citep{cui2025entropymechanismreinforcementlearning} and facilitating scalable RL~\citep{wang20258020rulehighentropyminority}. Although DAPO attempts to mitigate this issue by increasing the upper clipping bound~\citep{yu2025dapoopensourcellmreinforcement}, we found this approach to be less effective in our setup, which involved 16 rounds of off-policy updates per generation batch.

\begin{figure}[!t]
    \centering
    \includegraphics[width=0.5\textwidth]
    {figures/acc_compare_cropped.pdf}
\caption{Comparison of GRPO, DAPO, and our proposed \method{} on AIME 2024, based on Qwen2.5-32B-base. \method{} outperforms both GRPO and DAPO in terms of performance at the same number of training steps, and achieves comparable performance to DAPO using 50\% of the training steps.}
\label{fig:method-result}
\end{figure}

\noindent\textbf{The \method{} Algorithm.}
In response, we propose a new algorithm that explicitly avoids dropping tokens, even those associated with large updates, while inherently maintaining entropy within a reasonable range to ensure stable exploration. First, recall that the vanilla REINFORCE objective with corrected distribution for offline updates is:

\begin{equation}
    \begin{aligned}
    \mathcal{J}_{\text{REINFORCE}}(\theta) &= \mathbb{E}_{(q,a) \sim \mathcal{D}, o_i \sim \pi_{\theta_{\text{old}}}(\cdot|q)} \\
    & \left[
        \frac{1}{|o_i|} \sum_{t=1}^{|o_i|}
        \texttt{sg}(r_{i,t}(\theta))\hat{A}_{i,t}\log \pi_\theta(o_{i,t} \mid q, o_{i,<t})
    \right], 
    \end{aligned}
\label{eq:reinforce}
\end{equation}
where $\texttt{sg}(\cdot)$ denotes the stop-gradient operation.
Rather than clipping the token updates as in PPO/GRPO, we instead clip the importance sampling weight in Eq.~\ref{eq:reinforce} to stabilize training. 
We term our approach \method{} (\textbf{C}lipped \textbf{IS}-weight \textbf{P}olicy \textbf{O}ptimization). Adopting the group relative advantage from GRPO and the token-level loss~\citep{yu2025dapoopensourcellmreinforcement,liu2025understandingr1zeroliketrainingcritical}, \method{} optimizes the following objective:

\begin{equation}
    \begin{aligned}
    \mathcal{J}_{\text{\method{}}}(\theta) &= \mathbb{E}_{(q,a) \sim \mathcal{D}, \{o_i\}_{i=1}^G \sim \pi_{\theta_{\text{old}}}(\cdot|q)} \\
    & \left[
        \frac{1}{\sum_{i=1}^G |o_i|} \sum_{i=1}^G \sum_{t=1}^{|o_i|}
        \texttt{sg}(\hat{r}_{i,t}(\theta))\hat{A}_{i,t}\log \pi_\theta(o_{i,t} \mid q, o_{i,<t})
    \right], 
    \end{aligned}
\label{eq:CISPO}
\end{equation}
where $\hat{r}_{i,t}(\theta)$ is the clipped IS weight:

\begin{equation}
    \hat{r}_{i,t}(\theta) = \text{clip}\left(r_{
    i,t}(\theta), 1-\epsilon^{IS}_{low}, 1+\epsilon^{IS}_{high}\right).
\end{equation}
We note that without weight clipping, $\mathcal{J}_{\text{\method{}}}$ reduces to the standard policy gradient objective. In our experiments, we did not impose a lower bound on the IS weight by setting $\epsilon^{IS}_{low}$ to a large value; instead, we only tuned $\epsilon^{IS}_{high}$.
Although the gradient of Eq.~\ref{eq:CISPO} is slightly biased due to weight clipping, this approach preserves gradient contributions from all tokens, especially in long responses. 
\method{} proves effective in our experiments, helping reduce variance and stabilizing RL training. 
In addition, we utilize the dynamic sampling and length penalty techniques from~\citet{yu2025dapoopensourcellmreinforcement}. There is no KL penalty term in \method{} similar to other recent works~\citep{yu2025dapoopensourcellmreinforcement,hu2025openreasonerzeroopensourceapproach}.
% \begin{equation}
% M_{i,t} = \lnot [ ( \hat{A}_{i,t} > 0 \;\wedge\; r_{i,t} > 1 + \epsilon_{high} ) \;\vee\; ( \hat{A}_{i,t} < 0 \;\wedge\; r_{i,t} < 1 - \epsilon_{low}) ]. \tag{5}
% \end{equation}


\noindent\textbf{A General Formulation.}
While we adopt \method{} in our experiments, here we further present a unified formulation by introducing a token-wise mask into the \method{} objective. This allows for hyperparameter tuning to control whether, and under what conditions, gradients from specific tokens should be dropped:

\begin{equation}
    \begin{aligned}
    \mathcal{J}_{\text{unify}}(\theta) &= \mathbb{E}_{(q,a) \sim \mathcal{D}, \{o_i\}_{i=1}^G \sim \pi_{\theta_{\text{old}}}(\cdot|q)} \\
    & \left[
        \frac{1}{\sum_{i=1}^G |o_i|} \sum_{i=1}^G \sum_{t=1}^{|o_i|}
        \texttt{sg}(\hat{r}_{i,t}(\theta))\hat{A}_{i,t}\log \pi_\theta(o_{i,t} \mid q, o_{i,<t})M_{i,t}
    \right]. 
    \end{aligned}
\label{eq:unify}
\end{equation}
The mask $M_{i,t}$ is equivalent to the mask implicitly defined in the PPO trust region:
\begin{equation}
M_{i,t} = 
\begin{cases} 
0 & \text{if } \hat{A}_{i,t} > 0 \text{ and } r_{i,t}(\theta) > 1 + \epsilon_{\text{high}}, \\
0 & \text{if } \hat{A}_{i,t} < 0 \text{ and } r_{i,t}(\theta) < 1 - \epsilon_{\text{low}}, \\
1 & \text{otherwise}.
\end{cases} 
\end{equation}
% where $\texttt{sg}(\cdot)$ denotes the stop-gradient operation. 
This unified loss formulation can flexibly represent different clipping strategies under a common framework. 

\noindent\textbf{Empirical Validation of \method{}.}
To validate the effectiveness of \method{}, we empirically compare it with DAPO and GRPO in a zero-RL training setting. Specifically, we apply different RL algorithms to train the Qwen2.5-32B-base model on the mathematical reasoning dataset from~\citet{yu2025dapoopensourcellmreinforcement}, and report performance on the AIME 2024 benchmark. As shown in Figure~\ref{fig:method-result}, \method{} significantly outperforms both DAPO and GRPO with the same number of training steps. Notably, \method{} demonstrates superior training efficiency compared to other approaches; for example, it matches DAPO's performance with only 50\% of the training steps.




\subsection{Efficient RL Scaling with Lightning Attention -- Challenges and Recipes}
\label{sec:challenge}



\begin{figure}[!t]
\begin{subfigure}[t]{0.45\textwidth}  % 这里也用 [t]
    \centering
    \includegraphics[width=\textwidth]{figures/precision_comparison.pdf}
\end{subfigure}
\hfill
\begin{subfigure}[t]{0.45\textwidth}  % 注意这里用 [t]
    \centering
    \includegraphics[width=\textwidth]{figures/precision_comparison_v2.pdf}
\end{subfigure}
\caption{Probability of tokens in training-mode code vs. probability of tokens in inference-mode code. Each point in the figures represents an individual token. The Pearson correlation coefficient is indicated in the figures. Theoretically, the two probabilities should be identical, and all the tokens should be exactly on the diagonal line. 
{\bf Left:} Correlation of the M1 model before our fix; {\bf Right:} Correlation of the M1 model after applying our fix of using FP32 precision for the LM output head.}
\label{fig:mis-precision}
\end{figure}



As shown in Figure~\ref{fig:perf-bars-flops} (Right), we emphasize that our hybrid attention inherently enables more efficient RL scaling compared to traditional attention designs, since rollout computation and latency are often the primary bottlenecks in RL training. However, as pioneers in conducting large-scale RL experiments with this novel architecture, we encountered unique challenges and developed targeted solutions, as we describe below.

\noindent \textbf{Computational Precision Mismatch in Generation and Training.} 
RL training is highly sensitive to computational precision. 
During our RL training, we observed a significant discrepancy in the probabilities of rolled-out tokens between training-mode and inference-mode, as shown in Figure~\ref{fig:mis-precision} (Left). This discrepancy arose from a precision mismatch between the training and inference kernels. The issue was detrimental and prevented reward growth in our experiments.
Interestingly, this issue did not appear in smaller, dense models with softmax attention.
Through layer-by-layer analysis, we identified high-magnitude activations in the LM head at the output layer as the primary source of error. To address this, we increased the precision of the LM output head to FP32, thereby realigning the two theoretically identical probabilities, as demonstrated in Figure~\ref{fig:mis-precision} (Right). This adjustment improved the correlation between training and inference probabilities from approximately 0.9x to 0.99x. Notably, this correlation metric remained stable throughout training, enabling successful reward increase.

\noindent \textbf{Optimizer Hyperparameter Sensitivity.}
We employ the AdamW~\citep{loshchilovdecoupled} optimizer, and inappropriate configurations of $\beta_1$, $\beta_2$, and $\epsilon$ can lead to non-convergence during training. ~\cite{molybog2023theoryadaminstabilitylargescale}. For instance, using the default configuration from VeRL~\cite{sheng2024hybridflow}, where betas = (0.9, 0.999) and eps = 1e-8, can result in such issues.
We have observed that the gradient magnitudes in MiniMax-M1 training span a wide range, from 1e-18 to 1e-5, with the majority of the gradients being smaller than 1e-14. Furthermore, the correlation between the gradients of adjacent iterations is weak. Based on this, we set $\beta_1 = 0.9$, $\beta_2 = 0.95$, and eps=1e-15.

\noindent \textbf{Early Truncation via Repetition Detection.}
During RL training, we found that complex prompts could induce pathologically long and repetitive responses, whose large gradients threatened model stability. Our goal was to preemptively terminate these generation loops rather than penalize the already repetitive text. As simple string-matching is ineffective against varied repetition patterns, we developed a heuristic based on token probabilities. We observed that once a model enters a repetitive cycle, the probability for each token soars. Consequently, we implemented an early truncation rule: generation is halted if 3,000 consecutive tokens each have a probability above 0.99. This method successfully prevents model instability and improves generation throughput by eliminating these pathological, long-tail cases.




\section{Scaling Reinforcement Learning with Diverse Data}
\label{sec:data}
In this section, we describe the data and reward we adopted for our RL stage. We incorporate a diverse set of environments in our RL training pipeline, including tasks that can be verified by rules and general tasks that need to be verified through reward models.
All these environments are integrated into the RL stage using a carefully designed curriculum.

\subsection{Reasoning-Intensive Tasks with Rule-based Verification}
Below, we introduce our data that can be verified by deterministic rules. For all the following tasks, we employ rule-based final correctness as the correctness reward, complemented by a format reward.

\noindent \textbf{Mathematical Reasoning.}
Our initial mathematical dataset comprises hundreds of thousands of high-quality, competition-level problems, meticulously curated and organized from public sources and official mathematics competitions. These problems span a wide range of difficulty levels, each paired with a standard reference solution.
Our data cleaning pipeline begins with the removal of incomplete samples and those exhibiting formatting or typographical errors. We subsequently apply embedding-based deduplication across the RL data sources and enforce a strict separation from the SFT dataset to avoid any overlap, as leakage from the SFT phase into the RL stage hinders exploration and undermines training effectiveness. Additionally, we employ both n-gram and embedding-based methods to eliminate potential contamination from commonly used mathematical benchmark test sets, thereby ensuring the integrity and fairness of our evaluations.
We filter out samples containing multiple sub-problems, proof-based questions, and binary questions (e.g., true/false) that are susceptible to random guessing. Multiple-choice questions are reformulated into open-ended formats to better align with our reinforcement learning framework.
Next, we employ our internal model to extract the final answers from the reference solution, retaining only those samples whose extracted answers can be correctly parsed by our rule-based answer checker. Finally, we use a strong reasoning model to compute the pass@10 for each question and retain only those samples with a pass rate strictly between 0 and 0.9, resulting in a curated dataset of nearly 50K high-quality mathematical samples for our RL training.

\noindent \textbf{Logical Reasoning.}
For logical reasoning data, we carefully select 41 logical reasoning tasks requiring non-trivial reasoning ability such as cipher and Sudoku, then we implement a data synthesis framework to synthesize all the data. Concretely, we utilize our SynLogic framework~\citep{liu2025synlogic} to implement the data synthesis pipeline featuring task-specific data generators and rule-based task-specific verifiers, enabling automatic logical data generation. We meticulously configure the difficulty parameters during generation, ensuring the appropriate learning challenge of the generated data. Specifically, to prevent inclusion of overly difficult instances, we establish an upper difficulty bound based on the solvability limits of current strong reasoning models, requiring their pass@10 rates greater than zero. Similarly, we set a lower difficulty bound using the lowest difficulty parameters for which the MiniMax-Text-01 model achieves pass rates between 0 and 0.5. This approach ensures the data maintains a balance between difficulty and learnability. In addition, as the model capabilities improve during training, we increase the difficulty of the data in the later stages. Using this framework, we synthesize approximately 53K logical reasoning samples for RL training.

\noindent \textbf{Competitive Programming.}
For the competitive programming problems, we collect publicly available problems from online judge platforms and popular coding websites. For problems lacking test cases, we develop an LLM-based workflow and use the MiniMax-Text-01 model to generate comprehensive test suites. Similar to our approach with mathematical reasoning datasets, we filter problems based on quality and difficulty using pass rates from model sampling, retaining moderately challenging and high-quality algorithmic problems. Through this process, we generate 30K competitive programming data samples for RL training.

\noindent \textbf{Software Engineering.}
For the software engineering domain, inspired by SWE-bench~\citep{jimenez2024swebenchlanguagemodelsresolve}, we construct verifiable reinforcement learning environments by leveraging real-world data from public GitHub repositories. Our dataset primarily comprises issues and pull requests (PRs) that encapsulate common software development challenges, including bug localization, code repair, and test case synthesis.
To facilitate effective reinforcement learning, we develop a sophisticated containerized sandbox environment that simulates a realistic software development workflow. This environment enables the actual execution of code, providing direct and verifiable feedback on the correctness and efficacy of an agent's proposed interventions. The pass/fail status of pre-defined or newly generated test cases serves as the primary reward signal for our RL framework. A successful execution that passes all relevant test cases yields a positive reward, while compilation errors, runtime failures, or test case regressions result in a zero or negative reward, thus providing a clear signal for policy optimization.
Through this process, we curate several thousand high-quality data samples. Each sample includes a problem description (e.g., bug report from an issue), the initial faulty code, and a set of associated test cases. This setup allows our RL agent to learn to accurately pinpoint bugs, propose correct code fixes, and even synthesize new, effective test cases, with performance directly verifiable through the execution within our sandboxed environment.

\subsection{General Domain Tasks with Model-based Feedbacks}

In this section, we further extend the RL scope to a wider array of general domain tasks. As these tasks cannot be easily verified by rules, we utilize reward models to provide the feedback.

\subsubsection{Data and Reward Models}
Our general RL dataset consists of a total of 25K complex samples. These can be broadly categorized into two types: samples with ground-truth answers that are verifiable but difficult to validate using rules, and samples without ground-truth answers. 

\noindent\textbf{Tasks with Ground Truth.} This category primarily includes STEM and other factual problems where answers are objective but may have multiple valid expressions. Such diversity often renders rule-based answer checkers inaccurate. Our data cleaning process is similar to that used in mathematical reasoning, while we use our Generative Reward Model (GenRM) as a verifier, instead of relying on rule-based checkers.
To evaluate consistency between ground-truth answers and model responses, we adopt a five-grade reward scale to evaluate the two components. First, we construct a human-annotated reward model benchmark, which covers a range of objective tasks across diverse knowledge and task domains, especially the pairs of model response–ground truth that rule-based checkers fail to judge accurately. Second, we evaluate the GenRM's effectiveness by comparing the Best-of-N (BoN) responses selected by GenRM against the pass@N metrics across several benchmarks. GenRM performance is assessed using its accuracy on the human-annotated benchmark and the performance gap between BoN and pass@N. These metrics guide experiments to optimize both the data distribution and the prompt design used during the GenRM training.

\noindent\textbf{Tasks without Ground Truth.} This category encompasses a wider range of tasks, including instruction-following, creative writing, etc. 
Prompts are sampled from a large pool based on our internal tagging system, ensuring a balanced training distribution across fine-grained domains. 
Even though these queries are typically open-ended and do not have a ground-truth answer, we seek to pair a reference answer for each query, which serves as a reference for reward model judgment. To this end, we first generate responses by various internal and external models, and then these reference answers will undergo our internal quality evaluation.
During RL training, we adopt a pairwise comparison framework to evaluate model responses. Each comparison yields a score of -1, 0, or 1, indicating whether the model's output is worse than, similar to, or better than a reference answer. For instruction-following tasks with constraints particularly, we utilize both the rule-based reward to assess whether the response satisfies the constraint, and model-based reward to evaluate response's quality. As with the ground-truth setting, we first build a human-annotated benchmark, incorporating multiple blind preference judgments from reliable annotators. We then refine our scoring criteria and preference prompt to optimize accuracy as well as potential biases, which would be mentioned in \S\ref{sec:genrm-bias} below. To minimize the potential biases, training data are also optimized by several methods, such as multiple-blind consistent judgment, position-switched consistent judgment, etc. Once an optimal GenRM is trained, a Swiss Round scoring system is performed across the training dataset to determine the most suitable reference answer for RL training.


\subsubsection{Addressing Bias of Generative Reward Models for Long CoT}
\label{sec:genrm-bias}

Effective general RL for complex CoT reasoning tasks is critically dependent on accurate and unbiased reward models. Assessing such CoT responses turns out to be challenging, and we found that GenRMs preferred longer outputs over potentially superior concise alternatives, irrespective of actual reasoning quality. This {\bf length bias} is a significant issue as it may substantially misguide RL policy optimization, incentivizing verbosity without substance and inducing reward hacking.
Our initial efforts to improve GenRM fidelity include standard offline strategies: (1) Diversifying training data with a wide range of response lengths, sources, and quality tiers; (2) Incorporating adversarial examples to expose vulnerabilities; and (3) Refining model architectures. However, empirical analysis revealed that purely offline evaluation and preemptive mitigation of length bias in GenRMs frequently failed to prevent length bias during RL training.

Consequently, our core strategy incorporates continuous online monitoring of length bias during RL training. Specific metrics are established to detect whether the RL policy disproportionately extends output lengths to maximize GenRMs rewards without gains in task success or reasoning depth. Upon detecting such detrimental length-seeking behavior, indicative of exploiting GenRMs length bias, immediate GenRMs recalibration is triggered. This iterative adjustment is vital to preempt reward hacking related to output length, ensuring the policy prioritized substantive capability enhancement over superficial text inflation.
Complementing this adaptive approach, RL-side techniques including reward shaping, value clipping, and normalization are systematically employed.
These mechanisms desensitize reward signals to extreme values from superficial characteristics (e.g., length), thereby directing policy optimization toward substantive quality and correctness of its long CoT reasoning.

\subsection{Curriculum of Incorporating Diverse Data}

Given that our RL data spans a wide spectrum of categories, a core challenge is training a single policy capable of excelling on both reasoning-intensive tasks and general domain tasks. 
To address this, our approach entails a carefully managed curriculum and dynamic weighting strategy for reasoning and general-domain tasks during the RL training process with \method{}: we start with only the reasoning-intensive tasks with rule-based reward, and then gradually mix in the general domain tasks. This ensures that the model continues to refine its verifiable skills (e.g., in math and code) while progressively enhancing its performance on a diverse spectrum of general tasks, from complex instruction following to open-ended CoT reasoning.
This mixed RL training encourages the model to learn context-dependent application of its reasoning abilities—applying rigorous, step-by-step deduction for verifiable problems and more flexible, adaptive generation for general queries—all within a unified policy framework. It prevents catastrophic forgetting of specialized skills while fostering broader generalization.


\section{Extending RL Scaling to Longer Thinking}
\label{sec:long-context}
Our first RL training is performed with an output length limit of 40K tokens. Given that the hybrid architecture of M1 natively supports near-linear scaling for longer sequences, as demonstrated in Figure~\ref{fig:perf-bars-flops} (Right), we further extend the generation length during RL training to 80K tokens. This results in a new model, which we refer to as MiniMax-M1-80k.

\noindent\textbf{Data.} 
To efficiently train our RL model for an 80K output length, we utilize our previously trained 40K model to guide the data filtering process. First, we evaluate the pass rates on the curated dataset described in \S\ref{sec:data} and remove samples that are easily solved. We then adjust the data distribution to favor more challenging examples, such as difficult mathematical and coding problems. Additionally, we downsample synthetic reasoning data after observing that it destabilizes long-context RL training. Specifically, outputs generated from this data type often become repetitive and homogenous, and continued exposure to these patterns proves detrimental to the model's overall performance.

\noindent\textbf{Length Scaling Strategy.} To gradually increase the output length, we employ a staged window expansion RL strategy. We begin with an output length of 40K and incrementally expand it to 48K, 56K, 64K, 72K, and ultimately 80K. This staged approach ensures training stability at each step. The transition to a subsequent length is determined by a set of empirical indicators. These include the convergence of perplexity on the generated sequences and whether the 99th percentile of the output lengths is approaching the current context window limit. These signals offer valuable insights into the model's readiness for scaling, which allows us to maintain robust training throughout the process.

\noindent\textbf{Addressing Training Instability During Scaling.} During the scaling process, we encountered a critical issue in the later stages of training at each length window. Specifically, the model exhibited susceptibility to pattern collapse, where the latter portions of generated sequences degraded into incoherent or garbled text. This phenomenon consistently coincided with increased perplexity, indicating compromised generation quality and stability. We identify the root cause: during output length extension, negative samples increase in length substantially faster than positive samples, frequently reaching the context window limit earlier. Consequently, disproportionately large negative gradients accumulate in the latter segments of generation sequences. This imbalance originates from the inherently unequal nature of GRPO's advantage normalization and the token-level loss we adopt.
To address this, we implement three key solutions: (1) Detecting repetitive patterns (consecutive high-probability tokens) with early stopping to prevent excessive context window consumption by repetitive responses; (2) Adopting combined sample-level loss and token-level normalization to alleviate negative-positive sample imbalance and mitigate adverse effects; (3) Decreasing both the gradient clipping threshold and $\epsilon^{IS}_{high}$ to further stabilize generation.



\section{Evaluation}
\label{sec:eval}
This section presents the performance of our \model{} on a suite of challenging benchmarks spanning mathematics, coding, and logical reasoning, as well as other general tasks, comparing it against leading reasoning models.


\subsection{Benchmarks}
\label{subsec:bench}

To comprehensively assess \model{}, we conduct evaluations across a wide range of benchmarks, primarily covering 8 domains: knowledge, coding, math, reasoning, alignment, healthcare, multi-turn, and agent.

\begin{itemize}
    \item \textbf{Knowledge:} GPQA-Diamond~\citep{gpqa}, MMLU-Pro~\citep{mmlu-pro}, C-Eval~\citep{ceval}, Phybench~\citep{qiu2025phybenchholisticevaluationphysical}, AGIEval~\citep{zhong2023agieval}, TriviaQA~\citep{2017arXivtriviaqa}, CMMLU~\citep{li2023cmmlu}.
    \item \textbf{Coding:} LiveCodeBench-v6 (2408 to 2505)~\citep{livecodebench}, CodeForces~\footnote{The Codeforces was assessed through problems from 14 Div. 2 contests of Codeforces, combined with expert-designed test cases, followed by the computation of expected ratings and competitor proportions. It is worth noting that the highest rating attainable is 2209.}, Aider~\footnote{https://aider.chat/docs/benchmarks.html\#the-benchmark}.
    \item \textbf{Math:} AIME 2025~\citep{aime}, Omni-MATH~\citep{gao2024omnimathuniversalolympiadlevel}, HMMT 2025, CNMO 2024, FinanceReasoning~\citep{tang2025financereasoning}, UGMathBench~\citep{xu2025ugmathbench}.
    \item \textbf{Reasoning:} ARC-AGI-1~\citep{chollet2024arc}, BBEH~\citep{kazemi2025big}, ZebraLogic~\citep{lin2025zebralogic}, HLE~\citep{phan2025humanity}.
    \item \textbf{Alignment:} ArenaHard v2~\citep{li2024live}, Creative Writing v3~\citep{creative-writing-bench-v3}, IFEval~\citep{zhou2023instruction}.
    \item \textbf{Healthcare:} HealthBench~\citep{HealthBench}.
    \item \textbf{Multi-turn:} MultiChallenge~\citep{sirdeshmukh2025multichallenge}.
    \item \textbf{Agent:} BFCL v3~\citep{BFCL}.
\end{itemize}


We benchmark \model{} against leading open-weights models (DeepSeek-V3.1-Terminus-Thinking, and Qwen-35B-A22B-Thinking-2507) and proprietary API models (Gemini-2.5-pro, GPT-5-Thinking). All evaluations use controlled experimental conditions with standardized configurations.

\input{tables/benchmarks}


\subsection{Evaluation Settings}
All thinking models are evaluated under a standardized pipeline for a fair comparison. Benchmarks including AIME 2025, LiveCodeBench-v6, ARC-AGI-1, CodeForces, HMMT 2025, ZebraLogic, HLE, and BFCL v3 are assessed with a 128K context window, extended via YaRN \citep{peng2023yarn} for models with insufficient native context. Benchmarks including Aider, CNMO 2024, and BBEH use a 64K context window. Other benchmarks use a 32K context window. For baseline models, we use their official hyperparameters for open-weights models, while using vendor-recommended settings for proprietary APIs. For baselines with officially reported results, we report the higher score between our reproduction and the official release. 

\label{subsec:eval-set}



\subsection{Results}
\label{subsec:eval-res}

Table~\ref{tab:main_results} provides a comprehensive comparison of \model{} against leading thinking models. The following sections provide a detailed analysis of its performance across different aspects:


\paragraph{Mathematical Reasoning} 
\model{} demonstrates leading mathematical reasoning capabilities, as evidenced by its performance on challenging benchmarks. By relying solely on natural language reasoning, it achieves 93.40\% on AIME 25 and 86.72\% on HMMT 25, securing the second-highest rank overall and leading all open-weights models. Furthermore, the model delivers competitive results across specialized mathematical domains, scoring 82.63\% on Omni-MATH and 88.54\% on CNMO 2024. These results highlight a particular proficiency in complex, Olympiad-style problem-solving. This demonstrates that a stable and efficient RL training recipe, together with a diverse and high-quality math training dataset, drives superior mathematical reasoning across competition-level benchmarks. 

Moreover, we evaluate the mathematical reasoning capabilities of \model{} on the IMO 2025. Specifically, \model{} is integrated into the multi-agent framework AWorld~\citep{yu2025aworld} and tasked with solving the problems through pure natural language reasoning, without relying on code generation or external symbolic solvers. The model successfully solved Problems 1, 3, 4, and 5 on its first attempt, a performance corresponding to the IMO silver medal level. On its third attempt, it generated a nearly complete geometric proof for Problem 2. For the most challenging Problem 6, which no AI participant solved correctly during IMO 2025, \model{} converged to the same incorrect answer (4048) as Gemini 2.5 Pro, whereas the correct answer is 2112. A detailed case study is available in Appendix~\ref{sec:imo}. We believe the outstanding natural language reasoning ability will generalize to a broader range of tasks, paving the way for enhanced overall performance.


\paragraph{Coding Capabilities} 
As the results show, \model{} demonstrates exceptional performance in programming tasks that demand iterative refinement and deep logical reasoning, establishing a leading position among both open-weights and closed-weights models. On LiveCodeBench-v6 (2408-2505), it achieves a top score of 78.30\%, outperforming DeepSeek-V3.1 by 2.97 points and Qwen3-235B-A22B-Thinking-2507 by 2.58 percentage points. Furthermore, on CodeForces (rating), \model{} attains a score of 2088, which is the highest score among all models and exceeds the performance of both open-source competitors and closed-source APIs. It indicates that our carefully synthesized dataset shapes \model{}'s robust performance on programming applications, which forms a strong foundation for future endeavors on agentic applications. 

\paragraph{Logical Reasoning} 
\model{} demonstrates promising capabilities in other logical reasoning tasks. Powered by sourcing carefully selected logical games from multiple domains, \model achieves a score of 55.94\% on the challenging ARC-AGI-1 benchmark, ranking second overall. This performance places it only behind GPT-5-Thinking (65.70\%) and represents a substantial improvement of +15.32 percentage points over DeepSeek-V3.1 (40.62\%) and +7.82 points over Qwen3-235B-A22B-Thinking-2507 (48.12\%).

\paragraph{Human Alignment} In addition to the reasoning RL training, we also leverage a general RL training stage to equip the reasoning model with strong performance on general tasks. From Table~\ref{tab:main_results}, \model{} achieves strong alignment with human preferences in complex scenarios. On the ArenaHard v2 benchmark, it attains an 81.59\% win-rate, ranking second overall and trailing GPT-5-Thinking by only 1.32 percentage points. It also leads all models with an Elo rating of 84.52. In Creative Writing v3, \model{} scores 85.40\%, performing within 0.1 percentage points of the leading open-source model. These results confirm \model{}'s effectiveness in balancing human preference alignment with broad capabilities—a critical advantage for real-world deployment.

\paragraph{Healthcare Capabilities} 
On HealthBench, \model{} attains a score of 57.93\%, ranking second overall and leading the field of open-source models. This performance indicates proficient clinical knowledge integration and suggests the model's viability for complex healthcare tasks.



\input{conclusion}

\bibliography{sample_uniform_arxiv}

\newpage 
\section{Related Work}
\label{app:related_work}

\subsection{Extended Connections to Existing Work} \label{app:related_work_ratio_clipping}

In this work, we identify a structural flaw in PPO's ratio-clipping mechanism within the LLM regime: it over-penalizes low-probability tokens and under-penalizes high-probability ones, thereby impairing training efficiency and stability. Our proposed DPPO addresses this issue by directly constraining the policy divergence. 
This methodology aligns with the insights of \citet{wang2019trust, wang2020truly}, who observed similar exploration issues and proposed adaptive clipping based on KL divergence in traditional RL settings. However, in the context of LLMs, computing the exact divergence is prohibitive due to the huge memory footprint. To overcome this, we propose a binary divergence approximation, which empirically captures most of the benefits (see Appendix \ref{appendix:ablation_topk_approximation}). Furthermore, as demonstrated in \Cref{sec:stability} and \Cref{sec:training_efficiency}, the challenges of training stability and efficiency are exacerbated in LLMs by their expansive vocabularies, because low-probability tokens form a non-trivial portion of the entire distribution due to the long-tailed nature (see \Cref{fig:moe_prob_ratio_tv}). Finally, the training-inference mismatch inherent to the LLM era introduces additional algorithmic complexities, as further detailed in \Cref{sec:stability}.

\subsection{Training-inference Mismatch} \label{app:related_work_mismatch}
Recent work has identified a key culprit for training instability: the \textit{training-inference mismatch} (${\trainerpi}_\theta \neq {\rolloutpi}_\theta$), where the policy distribution used for gradient computation ($\trainerpi_\theta$) diverges from the one used for data generation ($\rolloutpi_\theta$), even when using identical model parameters $\theta$ \citep{yao2025offpolicy, qi2025defeating, liu-li-2025, zheng2025stabilizing}. This discrepancy arises from numerical precision errors \citep{qi2025defeating} and subtle differences in implementation \citep{Team2025EveryAM, he2025nondeterminism}. As training progresses, this mismatch can be amplified if the RL algorithm cannot manage it appropriately, leading to catastrophic performance degradation~\citep{qi2025defeating, liu-li-2025}.

Existing efforts to mitigate this issue primarily focus on correcting biased gradients through importance sampling. Building on this principle, techniques such as Truncated Importance Sampling (TIS)~\citep{yao2025offpolicy, zheng2025stabilizing} and Masked Importance Sampling \citep{liu-li-2025, team2025every} have been introduced at both the token and sequence levels. However, as suggested by \citet{qi2025defeating}, these methods often fail to achieve a satisfactory balance between training efficiency and stability. In contrast, our DPPO algorithm significantly enhances both aspects compared to these existing approaches.

Another line of research attempts to resolve the mismatch issue through higher precision \citep{qi2025defeating} or rigorous engineering alignment \citep{Team2025EveryAM, he2025nondeterminism, zhang2025deterministic}. While promising, these methods face limited applicability. For instance, aligning implementation details often requires specific training engines or model architectures, hindering broad adoption. Furthermore, in low-precision settings optimized for high-speed training, we must tolerate a significant training-inference mismatch. In such scenarios, a robust and fast algorithm like DPPO remains essential. Finally, our algorithmic design is orthogonal to these engineering-level optimizations and can be combined with them to achieve even greater performance gains.


\section{Trust Region in LLMs}
\label{app:llm_tr_proof}

\subsection{Proof of Performance Difference Identity}

\begin{proof}[Proof of \Cref{lem:llm_identity}]  
We begin by expressing the difference in expected returns by its definition:  
\begin{align*}  
\mathcal{J}(\trainerpi) - \mathcal{J}(\rolloutpi) 
&= \mathbb{E}_{y \sim \trainerpi}[R(y)] - \mathbb{E}_{y \sim \rolloutpi}[R(y)] \\
&= \sum_{y} \big( \trainerpi(y|x) - \rolloutpi(y|x) \big) R(y).
\end{align*}  
The core of the proof is to establish an identity for the difference in the probabilities of generating a sequence $y$, $\trainerpi(y|x) - \rolloutpi(y|x)$. We use the following telescoping sum identity, which can be verified by expanding the terms:  
\begin{align*}  
\trainerpi(y|x) - \rolloutpi(y|x) &= \sum_{t=1}^{T} \left( \prod_{k=1}^{t-1} \rolloutpi(y_k|s_k) \right) \Big( \trainerpi(y_t|s_t) - \rolloutpi(y_t|s_t) \Big) \left( \prod_{j=t+1}^{T} \trainerpi(y_j|s_j) \right).
\end{align*}  
Substituting this identity into the expression for the performance difference yields:  
\begin{align*}  
\mathcal{J}(\trainerpi) - \mathcal{J}(\rolloutpi) &= \sum_{y} R(y) \sum_{t=1}^{T}  \left( \prod_{k=1}^{t-1} \rolloutpi(y_k|s_k) \right) \Big( \trainerpi(y_t|s_t) - \rolloutpi(y_t|s_t) \Big) \left( \prod_{j=t+1}^{T} \trainerpi(y_j|s_j) \right) \\  
&= \sum_{y} \rolloutpi(y|x) R(y) \sum_{t=1}^{T} \left( \frac{\trainerpi(y_t|s_t)}{\rolloutpi(y_t|s_t)} - 1 \right) \left( \prod_{j=t+1}^{T} \frac{\trainerpi(y_j|s_j)}{\rolloutpi(y_j|s_j)} \right) \\ 
&= \mathbb{E}_{y \sim \rolloutpi} \left[ R(y) \sum_{t=1}^{T} \left( \frac{\trainerpi(y_t|s_t)}{\rolloutpi(y_t|s_t)} - 1 \right) \left( \prod_{j=t+1}^{T} \frac{\trainerpi(y_j|s_j)}{\rolloutpi(y_j|s_j)} \right)  \right] \\
% \end{align*}  
% This expression is exact. To derive the final form as stated in the theorem, we add and subtract a term inside the expectation, which corresponds to setting the future policy ratio term, $\frac{\trainerpi(y_{>t}|s_{t+1})}{\rolloutpi(y_{>t}|s_{t+1})}$, to 1.  
% \begin{align*}  
% \mathcal{J}(\trainerpi) - \mathcal{J}(\rolloutpi) 
&= \mathbb{E}_{y \sim \rolloutpi} \left[ R(y) \sum_{t=1}^{|y|} \left( \frac{\trainerpi(y_t|s_t)}{\rolloutpi(y_t|s_t)} - 1 \right) \right] \\  
& \quad - \mathbb{E}_{y \sim \rolloutpi} \left[ R(y) \sum_{t=1}^{|y|} \left( \frac{\trainerpi(y_t|s_t)}{\rolloutpi(y_t|s_t)} - 1 \right) \left( 1 -  \prod_{j=t+1}^{T} \frac{\trainerpi(y_j|s_j)}{\rolloutpi(y_j|s_j)}   \right) \right].  
\end{align*}  
By identifying the terms with the definitions in the theorem statement, we arrive at:  
\begin{equation*}  
\mathcal{J}(\trainerpi) - \mathcal{J}(\rolloutpi) = L_{\rolloutpi}'(\trainerpi) - \Delta(\rolloutpi, \trainerpi),  
\end{equation*}  
where  
\begin{align*}  
L_{\rolloutpi}'(\trainerpi) &= \mathbb{E}_{y \sim \rolloutpi} \left[ R(y) \sum_{t=1}^{|y|} \left( \frac{\trainerpi(y_t|s_t)}{\rolloutpi(y_t|s_t)} - 1 \right) \right], \\  
\Delta(\rolloutpi, \trainerpi) &= \mathbb{E}_{y \sim \rolloutpi} \left[ R(y) \sum_{t=1}^{|y|} \left( \frac{\trainerpi(y_t|s_t)}{\rolloutpi(y_t|s_t)} - 1 \right) \left( 1 - \prod_{j=t+1}^{T} \frac{\trainerpi(y_j|s_j)}{\rolloutpi(y_j|s_j)} \right) \right].  
\end{align*}  
This completes the proof.  
\end{proof}


\subsection{Proof of Policy Improvement Bound}

\begin{lemma}[Bound on Sequence-Level TV Divergence]  
\label{lem:sequence_tv_bound}  
Let $\rolloutpi$ and $\trainerpi$ be two policies that generate sequences of length $N$. Let ${\rolloutpi}_N(\cdot|s_1)$ and ${\trainerpi}_N(\cdot|s_1)$ denote the distributions over sequences $y=(y_1, \dots, y_N)$. The total variation (TV) divergence between these sequence distributions is bounded by the sum of the expected single-step TV divergences:  
\begin{equation*}  
D_{\mathrm{TV}}\big({\rolloutpi}_N(\cdot|s_1)  \|  {\trainerpi}_N(\cdot|s_1)\big) \le \sum_{t=1}^{N} \mathbb{E}_{s_t \sim \rolloutpi} \left[ D_{\mathrm{TV}}\big(\rolloutpi(\cdot|s_t) \| \trainerpi(\cdot|s_t)\big) \right],  
\end{equation*}  
where the expectation is over the state distribution induced by policy $\rolloutpi$.  
\end{lemma}  
  
\begin{proof}  
Let $P(y) = {\rolloutpi}_N(y|s_1)$ and $Q(y) = {\trainerpi}_N(y|s_1)$.  
\begin{equation*}  
    2 D_{\mathrm{TV}}(P \|  Q) = \sum_{y} |P(y) - Q(y)| = \sum_{y} \left| \prod_{t=1}^N \rolloutpi(y_t|s_t) - \prod_{t=1}^N \trainerpi(y_t|s_t) \right|.  
\end{equation*}  
We use the algebraic identity $a_1\dots a_N - b_1\dots b_N = \sum_{t=1}^N \left(\prod_{k=1}^{t-1} a_k\right) (a_t - b_t) \left(\prod_{j=t+1}^N b_j\right)$. Applying this to the policy probabilities and then using the triangle inequality, we get:  
\begin{align*}  
    2 D_{\mathrm{TV}}(P \|  Q) &\le \sum_{y} \sum_{t=1}^N \left(\prod_{k=1}^{t-1} \rolloutpi(y_k|s_k)\right) |\rolloutpi(y_t|s_t) - \trainerpi(y_t|s_t)| \left(\prod_{j=t+1}^N \trainerpi(y_j|s_j)\right) \\  
    &= \sum_{t=1}^N \sum_{y} \left(\prod_{k=1}^{t-1} \rolloutpi(y_k|s_k)\right) |\rolloutpi(y_t|s_t) - \trainerpi(y_t|s_t)| \left(\prod_{j=t+1}^N \trainerpi(y_j|s_j)\right).  
\end{align*}  
For each term in the outer sum over $t$, we can sum over the variables $y_j$ for $j>t$. Since $\sum_{y_j} \trainerpi(y_j|s_j) = 1$ for all $s_j$, the product of terms for $j>t$ sums to 1 when we integrate out $y_{t+1}, \dots, y_N$. This leaves:  
\begin{align*}  
    2 D_{\mathrm{TV}}(P \|  Q) &\le \sum_{t=1}^N \sum_{y_1, \dots, y_t} \left(\prod_{k=1}^{t-1} \rolloutpi(y_k|s_k)\right) |\rolloutpi(y_t|s_t) - \trainerpi(y_t|s_t)| \\  
    &= \sum_{t=1}^N \sum_{y_1, \dots, y_{t-1}} \left(\prod_{k=1}^{t-1} \rolloutpi(y_k|s_k)\right) \sum_{y_t} |\rolloutpi(y_t|s_t) - \trainerpi(y_t|s_t)|.  
\end{align*}  
The inner sum is $2 D_{\mathrm{TV}}(\rolloutpi(\cdot|s_t) \|  \trainerpi(\cdot|s_t))$. The outer sum over $y_1, \dots, y_{t-1}$ defines an expectation over states $s_t$ under policy $\rolloutpi$. Thus, we have:  
\begin{equation*}  
    2 D_{\mathrm{TV}}(P \|  Q) \le \sum_{t=1}^N \mathbb{E}_{s_t \sim \rolloutpi} \left[ 2 D_{\mathrm{TV}}\big(\rolloutpi(\cdot|s_t) \|  \trainerpi(\cdot|s_t)\big) \right].  
\end{equation*}  
Dividing by 2 yields the desired result.  
\end{proof}


\begin{proof}[Proof of \Cref{thm:llm_tr_bound}]  
From Lemma \ref{lem:llm_identity}, we start with the exact performance difference identity:  
\begin{equation*}  
\mathcal{J}(\trainerpi) - \mathcal{J}(\rolloutpi) = L_{\rolloutpi}'(\trainerpi) - \Delta(\rolloutpi, \trainerpi).  
\end{equation*}  
For brevity, we define $y_{\leq t} = \{x, y_1, \dots, y_t \}$ and $y_{>t} = \{y_{t+1}, y_{t+2}, \dots \}$, then we can rewrite $\Delta(\rolloutpi, \trainerpi)$ as:
\begin{equation*}
\Delta(\rolloutpi, \trainerpi) = \mathbb{E}_{y \sim \rolloutpi} \left[ R(y) \sum_{t=1}^{|y|} \left( \frac{\trainerpi(y_t|s_t)}{\rolloutpi(y_t|s_t)} - 1 \right) \left( 1 - \frac{\trainerpi(y_{>t}|s_{t+1})}{\rolloutpi(y_{>t}|s_{t+1})}  \right) \right].  
\end{equation*}
Our goal is to find an upper bound for the error term $\Delta(\rolloutpi, \trainerpi)$. We begin by bounding the reward by its maximum absolute value, $\xi = \max_{y} |R(y)|$.  
\begin{align}  
\label{eq:delta_bound_intermediate}
\begin{split}
\Delta(\rolloutpi, \trainerpi) &\le \xi \cdot \mathbb{E}_{y \sim \rolloutpi} \left[ \sum_{t=1}^{T} \left| \frac{\trainerpi(y_t|s_t)}{\rolloutpi(y_t|s_t)} - 1 \right| \cdot \left| 1 - \frac{\trainerpi(y_{>t}|s_{t+1})}{\rolloutpi(y_{>t}|s_{t+1})} \right| \right] \\  
&= \xi \cdot \sum_{t=1}^{T} \mathbb{E}_{y_{\le t} \sim \rolloutpi} \left[ \left| \frac{\trainerpi(y_t|s_t)}{\rolloutpi(y_t|s_t)} - 1 \right| \cdot \mathbb{E}_{y_{>t} \sim \rolloutpi(\cdot|s_{t+1})} \left[ \left| 1 - \frac{\trainerpi(y_{>t}|s_{t+1})}{\rolloutpi(y_{>t}|s_{t+1})} \right| \right] \right].  
\end{split}
\end{align}  
The inner expectation is exactly twice the TV divergence between the distributions over future trajectories:  
\begin{equation*}  
\mathbb{E}_{y_{>t} \sim \rolloutpi(\cdot|s_{t+1})} \left[ \left| 1 - \frac{\trainerpi(y_{>t}|s_{t+1})}{\rolloutpi(y_{>t}|s_{t+1})} \right| \right] = 2 D_{\mathrm{TV}}\big({\rolloutpi}_{>t}(\cdot|s_{t+1}) \|  {\trainerpi}_{>t}(\cdot|s_{t+1})\big).  
\end{equation*}  
Using \Cref{lem:sequence_tv_bound} on this sequence-level TV divergence (for a sequence of length $T-t$), we get:  
\begin{equation*}  
D_{\mathrm{TV}}\big({\rolloutpi}_{>t}(\cdot|s_{t+1}) \|  {\trainerpi}_{>t}(\cdot|s_{t+1})\big) \le \sum_{k=t+1}^{T} \mathbb{E}_{s_k \sim \rolloutpi(\cdot|s_{t+1})} \left[ D_{\mathrm{TV}}\big(\rolloutpi(\cdot|s_k) \|  \trainerpi(\cdot|s_k)\big) \right].  
\end{equation*}  
We bound each term in the sum by the maximum single-step TV divergence, $D_{\mathrm{TV}}^{\max}(\rolloutpi \|  \trainerpi) = \max_{s} D_{\mathrm{TV}}(\rolloutpi(\cdot|s) \|  \trainerpi(\cdot|s))$, which gives:  
\begin{equation*}  
D_{\mathrm{TV}}\big({\rolloutpi}_{>t}(\cdot|s_{t+1}) \|  {\trainerpi}_{>t}(\cdot|s_{t+1})\big) \le \sum_{k=t+1}^{T} D_{\mathrm{TV}}^{\max}(\rolloutpi \|  \trainerpi) = (T-t) D_{\mathrm{TV}}^{\max}(\rolloutpi \|  \trainerpi).  
\end{equation*}  
Substituting this back into the bound for $\Delta(\rolloutpi, \trainerpi)$:  
\begin{align}
\label{eq:delta_bound_final}
\begin{split}
\Delta(\rolloutpi, \trainerpi) &\le \xi \cdot \sum_{t=1}^{T} \mathbb{E}_{y_{\le t} \sim \rolloutpi} \left[ \left| \frac{\trainerpi(y_t|s_t)}{\rolloutpi(y_t|s_t)} - 1 \right| \cdot 2(T-t) D_{\mathrm{TV}}^{\max}(\rolloutpi \|  \trainerpi) \right] \\  
&= 2\xi \cdot D_{\mathrm{TV}}^{\max}(\rolloutpi \|  \trainerpi) \sum_{t=1}^{T} (T-t) \mathbb{E}_{s_t \sim \rolloutpi} \left[ \sum_{y_t} \rolloutpi(y_t|s_t) \left| \frac{\trainerpi(y_t|s_t)}{\rolloutpi(y_t|s_t)} - 1 \right| \right] \\
&= 2\xi \cdot D_{\mathrm{TV}}^{\max}(\rolloutpi \|  \trainerpi) \sum_{t=1}^{T} (T-t) \mathbb{E}_{s_t \sim \rolloutpi} \left[ 2 D_{\mathrm{TV}}(\rolloutpi(\cdot|s_t) \|  \trainerpi(\cdot|s_t)) \right] \\
&\le 2\xi \cdot D_{\mathrm{TV}}^{\max}(\rolloutpi \|  \trainerpi) \sum_{t=1}^{T} (T-t) \cdot \mathbb{E}_{s_t \sim \rolloutpi} \left[ 2 D_{\mathrm{TV}}^{\max}(\rolloutpi \|  \trainerpi) \right] \\  
&= 4\xi \cdot {D_{\mathrm{TV}}^{\max}(\rolloutpi \|  \trainerpi)}^2 \sum_{t=1}^{T} (T-t) \\
&= 2\xi T(T-1) \cdot {D_{\mathrm{TV}}^{\max}(\rolloutpi \|  \trainerpi)}^2.  
\end{split}
\end{align}  
Substituting this into the performance difference identity gives the desired result:  
\begin{equation*}  
\mathcal{J}(\trainerpi) - \mathcal{J}(\rolloutpi) \ge L_{\rolloutpi}'(\trainerpi) - 2\xi T(T-1) \cdot {D_{\mathrm{TV}}^{\max}(\rolloutpi \|  \trainerpi)}^2.  
\end{equation*}  
This completes the proof.  
\end{proof}

\subsection{A Tighter Policy Improvement Bound}  
\label{app:tighter_bound}  
  
The policy improvement bound derived in \Cref{eq:delta_bound_final} suffers from a quadratic dependence on the horizon length, $T^2$. This makes the bound excessively loose for typical LLM fine-tuning tasks where sequences can be very long. By leveraging the property that the total variation divergence is always bounded by one, i.e., $D_{\mathrm{TV}}(P \|  Q) \le 1$, we can derive an alternative bound that is only linear in $T$, offering a much tighter and more practical guarantee for long-horizon problems.  
  
We begin from the intermediate step in \Cref{eq:delta_bound_intermediate}:  
\begin{equation*}  
\Delta(\rolloutpi, \trainerpi) \le \xi \cdot \sum_{t=1}^{T} \mathbb{E}_{y_{\le t} \sim \rolloutpi} \left[ \left| \frac{\trainerpi(y_t|s_t)}{\rolloutpi(y_t|s_t)} - 1 \right| \cdot \mathbb{E}_{y_{>t} \sim \rolloutpi(\cdot|s_{t+1})} \left[ \left| 1 - \frac{\trainerpi(y_{>t}|s_{t+1})}{\rolloutpi(y_{>t}|s_{t+1})} \right| \right] \right].  
\end{equation*}  
The inner expectation is exactly twice the TV divergence between the future trajectory distributions, $2 D_{\mathrm{TV}}\big({\rolloutpi}_{>t}(\cdot|s_{t+1}) \|  {\trainerpi}_{>t}(\cdot|s_{t+1})\big)$. Instead of bounding this term with $2 (T-t) D_{\mathrm{TV}}^{\max}(\rolloutpi \|  \trainerpi)$, we now apply the simple upper bound of 2:  
\begin{align}  
\Delta(\rolloutpi, \trainerpi) &\le \xi \cdot \sum_{t=1}^{T} \mathbb{E}_{y_{\le t} \sim \rolloutpi} \left[ \left| \frac{\trainerpi(y_t|s_t)}{\rolloutpi(y_t|s_t)} - 1 \right| \cdot 2 D_{\mathrm{TV}}\big({\rolloutpi}_{>t}(\cdot|s_{t+1}) \|  {\trainerpi}_{>t}(\cdot|s_{t+1})\big) \right] \nonumber \\  
&\le 2\xi \cdot \sum_{t=1}^{T} \mathbb{E}_{y_{\le t} \sim \rolloutpi} \left[ \left| \frac{\trainerpi(y_t|s_t)}{\rolloutpi(y_t|s_t)} - 1 \right| \right] \label{eq:tighter_bound_step1} \\  
&= 2\xi \cdot \sum_{t=1}^{T} \mathbb{E}_{s_t \sim \rho_t^{\rolloutpi}} \mathbb{E}_{y_t \sim \rolloutpi(\cdot|s_t)} \left[ \left| \frac{\trainerpi(y_t|s_t)}{\rolloutpi(y_t|s_t)} - 1 \right| \right] \nonumber \\  
&= 2\xi \cdot \sum_{t=1}^{T} \mathbb{E}_{s_t \sim \rho_t^{\rolloutpi}} \left[ 2 D_{\mathrm{TV}}(\rolloutpi(\cdot|s_t) \|  \trainerpi(\cdot|s_t)) \right] \nonumber \\  
&= 4\xi \cdot \mathbb{E}_{y \sim \rolloutpi} \left[ \sum_{t=1}^{|y|} D_{\mathrm{TV}}(\rolloutpi(\cdot|s_t) \|  \trainerpi(\cdot|s_t)) \right]. \label{eq:tighter_bound_final}  
\end{align}  
This provides a bound that is linear in the expected sum of single-step divergences. By combining this with our original quadratic bound from \Cref{eq:delta_bound_final}, we can form a tighter, composite bound by taking the minimum of the two:  
\begin{align*}  
\mathcal{J}(\trainerpi) - \mathcal{J}(\rolloutpi) &\ge L_{\rolloutpi}'(\trainerpi) - \Delta(\rolloutpi, \trainerpi) \\  
&\ge L_{\rolloutpi}'(\trainerpi) - \min \left( 2\xi T(T-1) \cdot {D_{\mathrm{TV}}^{\max}}^2, 4\xi \cdot \mathbb{E}_{y \sim \rolloutpi} \left[ \sum_{t=1}^{|y|} D_{\mathrm{TV}}(\rolloutpi(\cdot|s_t) \|  \trainerpi(\cdot|s_t)) \right] \right).  
\end{align*}  
This composite bound provides a more robust guarantee on policy improvement, leveraging the quadratic bound for infinitesimal updates and the linear bound for larger updates or longer horizons.

\subsection{Comparing Surrogate Objectives with Classical RL}  
\label{app:compare_classical_rl}  
  
At first glance, the surrogate objective for the LLM regime in \Cref{eq:llm_surrogate} appears distinct from the classical RL surrogate in \Cref{eq:surrogate}. The former is an expectation over full trajectories $y$ weighted by the reward $R(y)$, while the latter is an expectation over state-action pairs $(s,a)$ weighted by the advantage $A^{\rolloutpi}(s,a)$. However, we will now show that their gradients with respect to the policy parameters $\theta$ are fundamentally analogous, confirming that our LLM-specific formulation is a valid adaptation of the standard policy gradient theorem.  
  
Let the policy $\trainerpi$ be parameterized by $\theta$. We will use the identity $\nabla_\theta \pi_\theta(a|s) = \pi_\theta(a|s) \nabla_\theta \log \pi_\theta(a|s)$.  
  
\textbf{Gradient of the Classical Surrogate Objective.}  
We begin with the classical surrogate objective from \Cref{eq:surrogate}:  
\begin{equation*}  
L_{\rolloutpi}({\trainerpi}_\theta) = \frac{1}{1 - \gamma}  \mathbb{E}_{s \sim \rho^{\rolloutpi},\,a \sim \rolloutpi(a|s)} \left[ \frac{\trainerpi_\theta(a| s)}{\rolloutpi(a| s)} A^{\rolloutpi}(s,a) \right].  
\end{equation*}  
Taking the gradient with respect to $\theta$ and moving it inside the expectation, we get:  
\begin{align}  
\label{eq:grad_classical}
\begin{split}
\nabla_\theta L_{\rolloutpi}({\trainerpi}_\theta) &= \frac{1}{1 - \gamma} \mathbb{E}_{s \sim \rho^{\rolloutpi},\,a \sim \rolloutpi(a|s)} \left[ \frac{\nabla_\theta \trainerpi_\theta(a| s)}{\rolloutpi(a| s)} A^{\rolloutpi}(s,a) \right] \\  
&= \frac{1}{1 - \gamma} \mathbb{E}_{s \sim \rho^{\rolloutpi},\,a \sim \rolloutpi(a|s)} \left[ \frac{\trainerpi_\theta(a| s)}{\rolloutpi(a| s)} \nabla_\theta \log {\trainerpi}_\theta(a| s) A^{\rolloutpi}(s,a) \right].  
\end{split}
\end{align}  
\textbf{Gradient of the LLM Surrogate Objective.}  
Next, we consider our LLM-specific surrogate from \Cref{eq:llm_surrogate}:  
\begin{equation*}  
L_{\rolloutpi}'({\trainerpi}_\theta) = \mathbb{E}_{y \sim \rolloutpi} \left[ R(y) \sum_{t=1}^{|y|} \left( \frac{\trainerpi_\theta(y_t|s_t)}{\rolloutpi(y_t|s_t)} - 1 \right) \right].  
\end{equation*}  
Taking the gradient with respect to $\theta$ and noting that the $-1$ term has a zero gradient:  
\begin{align*}  
\nabla_\theta L_{\rolloutpi}'({\trainerpi}_\theta) &= \mathbb{E}_{y \sim \rolloutpi} \left[ R(y) \sum_{t=1}^{|y|} \frac{\nabla_\theta \trainerpi_\theta(y_t|s_t)}{\rolloutpi(y_t|s_t)} \right] \\  
&= \mathbb{E}_{y \sim \rolloutpi} \left[ \sum_{t=1}^{|y|} \frac{\trainerpi_\theta(y_t|s_t) }{\rolloutpi(y_t|s_t)} \nabla_\theta \log {\trainerpi}_\theta(y_t|s_t) R(y) \right].  
\end{align*}  
If we define a sequence-level advantage as $A^{\rolloutpi}(s_t, y_t) = R(y) - V(x)$, where $V(x)$ is a baseline value function for the prompt, the gradient becomes:  
\begin{equation} 
\label{eq:grad_llm_surrogate}
\nabla_\theta L_{\rolloutpi}'({\trainerpi}_\theta) = \mathbb{E}_{y \sim \rolloutpi} \left[ \sum_{t=1}^{|y|}  \frac{\trainerpi_\theta(y_t|s_t) }{\rolloutpi(y_t|s_t)} \nabla_\theta \log {\trainerpi}_\theta(y_t|s_t) A^{\rolloutpi}(s_t, y_t) \right].  
\end{equation}  
This form is directly analogous to the classical policy gradient in \Cref{eq:grad_classical}, where the sum over timesteps in a trajectory replaces the expectation over the state distribution $\rho^{\rolloutpi}$. Thus, our LLM surrogate objective is a theoretically sound adaptation of the classical trust region framework to the undiscounted, sequence-reward setting.



\section{Approximations as Lower Bounds of True Divergence}  
\label{app:divergence_lower_bounds}  
  
In this section, we provide a formal justification for our Binary and Top-K divergence approximations. We demonstrate that both are principled lower bounds on the true divergence and explicitly state the conditions under which these approximations become exact.  
  
Let $\mathcal{C} = \{C_1, \dots, C_m\}$ be any partition of the vocabulary $\mathcal{A}$. Our Binary and Top-K approximations correspond to specific choices of this partition. We will show that the divergence computed on the partitioned space is a lower bound on the true divergence.  
  
\subsection{Total Variation Divergence}  
  
The true TV divergence is $D_{\mathrm{TV}}(\rolloutpi \|  \trainerpi) = \frac{1}{2} \sum_{a \in \mathcal{A}} |\rolloutpi(a|s_t) - \trainerpi(a|s_t)|$. The divergence on a partitioned space $\mathcal{C}$ is $D_{\mathrm{TV}}^{\mathcal{C}} = \frac{1}{2} \sum_{j=1}^m |\rolloutpi(C_j|s_t) - \trainerpi(C_j|s_t)|$.  
  
\textbf{Proof of Lower Bound.}  
By definition, $|\rolloutpi(C_j|s_t) - \trainerpi(C_j|s_t)| = |\sum_{a \in C_j} (\rolloutpi(a|s_t) - \trainerpi(a|s_t))|$. The triangle inequality states that the absolute value of a sum is less than or equal to the sum of the absolute values. Applying this, we get $|\sum_{a \in C_j} (\rolloutpi(a|s_t) - \trainerpi(a|s_t))| \le \sum_{a \in C_j} |\rolloutpi(a|s_t) - \trainerpi(a|s_t)|$. Summing over all partitions $j$:  
\begin{align*}  
D_{\mathrm{TV}}^{\mathcal{C}} &= \frac{1}{2} \sum_{j=1}^m \left| \sum_{a \in C_j} (\rolloutpi(a|s_t) - \trainerpi(a|s_t)) \right| \\  
&\le \frac{1}{2} \sum_{j=1}^m \sum_{a \in C_j} |\rolloutpi(a|s_t) - \trainerpi(a|s_t)| = D_{\mathrm{TV}}(\rolloutpi \|  \trainerpi).  
\end{align*}  
Thus, $D_{\mathrm{TV}}(\rolloutpi \|  \trainerpi) \ge D_{\mathrm{TV}}^{\mathcal{C}}$. This holds for both Binary and Top-K partitions.  
  
\textbf{Analysis of the Approximation Gap.}  
The gap between the true and approximated TV divergence is the sum of the gaps within each partition. For any partition $C_j$, the gap is $\frac{1}{2} \left( \sum_{a \in C_j} |\rolloutpi(a|s_t) - \trainerpi(a|s_t)| - \left|\sum_{a \in C_j} (\rolloutpi(a|s_t) - \trainerpi(a|s_t))\right| \right)$. This gap is bounded by the total probability mass of the partition:  
\begin{equation*}  
\text{Gap}(C_j) \le \frac{1}{2} \sum_{a \in C_j} (\rolloutpi(a|s_t) + \trainerpi(a|s_t)) = \frac{1}{2} (\rolloutpi(C_j|s_t) + \trainerpi(C_j|s_t)).  
\end{equation*}  
For the Top-K approximation, the only partition with a potential gap is the "other" category, which contains the tail of the distribution. The total probability mass of this tail, $\rolloutpi(C_{\text{other}}|s_t)$, is typically very small. Therefore, the approximation gap is also small, justifying Top-K TV as a high-fidelity approximation.  
  
\textbf{Equality Condition.}  
Equality $D_{\mathrm{TV}} = D_{\mathrm{TV}}^{\mathcal{C}}$ holds if the gap is zero for all partitions. This occurs when $\rolloutpi(a|s_t) - \trainerpi(a|s_t)$ has the same sign for all tokens $a$ within each partition $C_j$.  
  
\subsection{KL Divergence}  
  
The true KL divergence is $D_{\mathrm{KL}}(\rolloutpi \| \trainerpi) = \sum_{a \in \mathcal{A}} \rolloutpi(a|s_t) \log \frac{\rolloutpi(a|s_t)}{\trainerpi(a|s_t)}$. The divergence on the partitioned space is $D_{\mathrm{KL}}^{\mathcal{C}} = \sum_{j=1}^m \rolloutpi(C_j|s_t) \log \frac{\rolloutpi(C_j|s_t)}{\trainerpi(C_j|s_t)}$.  
  
\textbf{Proof of Lower Bound.}  
The proof relies on the log-sum inequality, which states that for any two sets of non-negative numbers $\{x_1, \dots, x_n\}$ and $\{y_1, \dots, y_n\}$:  
\begin{equation*}  
\sum_{i=1}^n x_i \log \frac{x_i}{y_i} \ge \left(\sum_{i=1}^n x_i\right) \log \frac{\sum_{i=1}^n x_i}{\sum_{i=1}^n y_i}.  
\end{equation*}  
We apply this inequality to each partition $C_j$ in our vocabulary, setting $x_a = \rolloutpi(a|s_t)$ and $y_a = \trainerpi(a|s_t)$:  
\begin{align*}  
\sum_{a \in C_j} \rolloutpi(a|s_t) \log \frac{\rolloutpi(a|s_t)}{\trainerpi(a|s_t)} &\ge \left(\sum_{a \in C_j} \rolloutpi(a|s_t)\right) \log \frac{\sum_{a \in C_j} \rolloutpi(a|s_t)}{\sum_{a \in C_j} \trainerpi(a|s_t)} \\  
&= \rolloutpi(C_j|s_t) \log \frac{\rolloutpi(C_j|s_t)}{\trainerpi(C_j|s_t)}.  
\end{align*}  
Summing over all partitions $j$ gives the desired result:  
\begin{align*}  
D_{\mathrm{KL}}(\rolloutpi \| \trainerpi) &= \sum_{j=1}^m \sum_{a \in C_j} \rolloutpi(a|s_t) \log \frac{\rolloutpi(a|s_t)}{\trainerpi(a|s_t)} \\  
&\ge \sum_{j=1}^m \rolloutpi(C_j|s_t) \log \frac{\rolloutpi(C_j|s_t)}{\trainerpi(C_j|s_t)} = D_{\mathrm{KL}}^{\mathcal{C}}.  
\end{align*}  
  
\textbf{Equality Condition.}  
The log-sum inequality holds with equality if and only if the ratio $\frac{x_i}{y_i}$ is constant for all $i$. In our context, this means that for each partition $C_j$, the ratio $\frac{\rolloutpi(a|s_t)}{\trainerpi(a|s_t)}$ must be constant for all tokens $a \in C_j$. For both Binary and Top-K approximations, this implies the policy update must scale the probabilities of all tokens within the "other" category by a uniform factor.


\section{More Details for Stability Analysis}
\label{app:sanity_test}

Our experimental setup strictly follows the sanity test established in \citet{qi2025defeating}. Each policy iteration begins by sampling a batch of 64 questions. For each question, we generate 8 responses (rollouts) using a maximum context length of 8,000. The collected data is then used to perform 4 gradient steps. All experiments are conducted using the VeRL framework \citep{sheng2024hybridflow} together with the ODC optimization \citep{wan2026revisiting}, and models are trained in BFloat16 precision to better expose potential numerical instabilities between algorithms. For the evaluation on AIME, we sample 32 responses for each test question to ensure a robust assessment.

\subsection{Algorithmic Details for Stability Analysis}  
\label{app:sanity_test_details}  
  
In this section, we provide the specific policy gradient formulations for each algorithm evaluated in our stability analysis (\Cref{sec:stability}). To facilitate a direct comparison, we show how each algorithm's gradient update can be interpreted through the lens of a single, unified framework.  
  
\textbf{A Unified Policy Gradient Formulation.}  
The policy gradient for the algorithms we tested can be generalized into the following form, where the gradient of the objective $L(\theta)$ is expressed as:  
\begin{equation}  
\label{eq:unified_grad}  
\nabla_\theta L(\theta) = \mathbb{E}_{y \sim \rolloutpi_{\theta'}} \left[ \sum_{t=1}^{|y|} M_t \cdot \min \left( \frac{\trainerpi_\theta(y_t|s_t)}{\rolloutpi_{\theta'}(y_t|s_t)}, C \right) \cdot \hat{A}_t \cdot \nabla_\theta \log {\trainerpi}_\theta(y_t|s_t) \right].  
\end{equation}  
In this formulation, $\hat{A}_t$ is the advantage, estimated following the GRPO method but without standard deviation normalization \citep{shao2024deepseekmath, liu2025understanding}. The algorithms differ primarily in their definition of the binary mask $M_t$ and the clipping bound $C$.  
  
\begin{itemize}  
    \item For \textbf{PG-IS}, we have $M_t = 1$ and $C = \infty$.  
    \item For \textbf{PG-TIS (CISPO)}, we have $M_t = 1$ and $C = 3$.  
    \item For \textbf{GRPO}, the mask $M_t$ is the PPO-style clipping mask, and $C = \infty$.  
    \item For \textbf{MiniRL}, the mask $M_t$ is also a PPO-style clipping mask but is conditioned on a recomputed policy ratio. For this algorithm, $C = \infty$.  
    \item For \textbf{MiniRL-TIS}, the mask $M_t$ is the same as in MiniRL, but with $C = 3$.  
    \item For \textbf{DPPO (Ours)}, the mask $M_t$ is conditioned on the policy divergence, and $C = \infty$.  
\end{itemize}  
  
\textbf{Mask Definitions.}  
The specific forms of the masks are as follows:  
\begin{itemize}  
    \item For \textbf{GRPO}, the mask uses the rollout ratio $r_t = \frac{\trainerpi_\theta(y_t|s_t)}{\rolloutpi_{\theta'}(y_t|s_t)}$ and experimental hyperparameters $\epsilon_\text{high}=0.28, \epsilon_\text{low}=0.2$:  
    \begin{equation*}  
    M_t =  
        \begin{cases}  
            0, & \text{if } (\hat{A}_t > 0 \text{ and } r_t > 1 + \epsilon_\text{high}) \text{ or } (\hat{A}_t < 0 \text{ and } r_t < 1 - \epsilon_\text{low}) \\  
            1, & \text{otherwise}.  
        \end{cases}  
    \end{equation*}  
  
    \item For \textbf{MiniRL} and \textbf{MiniRL-TIS}, the mask is structurally identical to GRPO's and uses the same hyperparameters, but it is conditioned on the recomputed ratio $r'_t = \frac{\trainerpi_\theta(y_t|s_t)}{\trainerpi_{\theta'}(y_t|s_t)}$:  
    \begin{equation*}  
    M_t =  
        \begin{cases}  
            0, & \text{if } (\hat{A}_t > 0 \text{ and } r'_t > 1 + \epsilon_\text{high}) \text{ or } (\hat{A}_t < 0 \text{ and } r'_t < 1 - \epsilon_\text{low}) \\  
            1, & \text{otherwise}.  
        \end{cases}  
    \end{equation*}  
  
    \item For \textbf{DPPO}, our mask is conditioned on the policy divergence $D_t$:  
    \begin{equation*}  
    M_t =  
        \begin{cases}  
            0, & \text{if } (\hat{A}_t > 0 \text{ and } r_t > 1 \text{ and } D_t > \delta) \text{ or } (\hat{A}_t < 0 \text{ and } r_t < 1 \text{ and } D_t > \delta) \\  
            1, & \text{otherwise}.  
        \end{cases}  
    \end{equation*}  
    In our experiments, we set the divergence threshold $\delta=0.15$ for TV divergence and $\delta=0.05$ for KL divergence.  
\end{itemize}

% \subsection{The Unexpected Harm of Truncated Importance Sampling}  
  
% Our results also reveal a surprising finding regarding Truncated Importance Sampling (TIS), a technique widely adopted to control the variance of policy gradient estimates \citep{yao2025offpolicy, chen2025minimax}. Contrary to its intended purpose, TIS consistently degrades training stability in our experiments. As shown in \Cref{fig:sanity_test}, the TIS-enabled variants (PG-TIS (CISPO) and MiniRL-TIS) collapse significantly earlier and perform much worse than their untruncated counterparts.  
  
% We hypothesize that this detrimental effect stems from the same core issue as PPO's ratio clipping: low-probability tokens, which naturally produce high-variance ratios, are the most likely to be truncated by TIS. While this does reduce variance, it systematically down-weights the gradient signal from these tokens, introducing a significant and harmful bias into the policy update. This suggests that naive truncation can be just as damaging as naive clipping.


\section{Characterizing Clipped Tokens}  
\label{app:clipped_tokens}  
  
To understand the practical consequences of ratio clipping, we analyzed which tokens are most frequently penalized by a PPO-style algorithm. We trained a Qwen3-4B-Base model on the DAPO dataset with GRPO and, at training step 50, collected two sets of tokens:  
\begin{itemize}  
    \item \textbf{Clipped Positive Tokens:} From samples with $\hat{A}_t > 0$, tokens whose updates were blocked due to a high ratio ($r_t > 1.28$).  
    \item \textbf{Clipped Negative Tokens:} From samples with $\hat{A}_t < 0$, tokens whose updates were blocked due to a low ratio ($r_t < 0.8$).  
\end{itemize}  
  
The 50 most frequent tokens in each category reveal a striking pattern. Far from being random noise, the clipped tokens are often critical for task performance. The lists for both positive and negative samples are dominated by two key categories:  
\begin{enumerate}  
    \item \textbf{Numerical and Mathematical Tokens:} A significant portion of the clipped tokens are numbers (e.g., `\textcolor{red}{1}', `\textcolor{red}{4}') and mathematical symbols (e.g., `\textcolor{red}{+}', `\textcolor{red}{=}', `\textcolor{red}{div}').  
    \item \textbf{Reasoning and Structural Words:} The list also includes many words essential for logical exposition, such as `\textcolor{blue}{Wait}', `\textcolor{blue}{Next}', `\textcolor{blue}{Thus}', and `\textcolor{blue}{Since}'.  
\end{enumerate}
  
These findings highlight a fundamental flaw in ratio-based clipping. For positive samples, it blocks beneficial updates to tokens that are integral to constructing correct solutions. For negative samples, it blocks the necessary suppression of these same tokens when they are part of an incorrect reasoning path. By systematically interfering with the learning signal for these high-utility tokens, the algorithm inadvertently slows learning, stifles exploration, and hinders the model's ability to refine its problem-solving capabilities.


\begin{AIbox}{The 50 most frequently clipped tokens from \textbf{positively-rewarded} samples.}
\begin{lstlisting}
' the', ' \\(', '<@\textcolor{red}{1}@>', '<@\textcolor{blue}{Let}@>', ' in', ' ', ',', 'We', ' <@\textcolor{red}{+}@>', ' \\', ' numbers', ':\n\n', '<@\textcolor{blue}{Wait}@>', '<@\textcolor{red}{4}@>', '<@\textcolor{red}{6}@>', ' Identify', '(', '<@\textcolor{blue}{Next}@>', ' from', ')', ' k', ' <@\textcolor{red}{-}@>', '<@\textcolor{blue}{Since}@>', ' solve', '\\[', ' how', ' ->', ' to', ' are', '<@\textcolor{red}{Sub}@>', 'I', '):\n', '  \n\n', ' spiral', ' <@\textcolor{blue}{Instead}@>', ' this', '<@\textcolor{blue}{If}@>', '<@\textcolor{red}{div}@>', ' Conditions', ' vector', ' have', ' <@\textcolor{red}{=}@>', ' feasible', 'Or', ' inconsistency', ' express', '_{', ' increase', ' exact', ' consider'
\end{lstlisting}
\end{AIbox}



\begin{AIbox}{The 50 most frequently clipped tokens from \textbf{negatively-rewarded} samples.}
\begin{lstlisting}
' \\(', ' the', ',', ' a', ' \\', ' ', '<@\textcolor{red}{2}@>', '<@\textcolor{red}{1}@>', ':\n\n', '<@\textcolor{red}{0}@>', '<@\textcolor{red}{3}@>', ' and', ' (', ' that', '<@\textcolor{red}{-}@>', ' to', '<@\textcolor{red}{5}@>', ' of', '<@\textcolor{blue}{However}@>', '\\', ' is', ' <@\textcolor{red}{=}@>', '<@\textcolor{red}{4}@>', ' in', ' for', ' all', ' we', 'We', ')', '.\n\n', ' our', '.', ':\n', ' <@\textcolor{blue}{but}@>', ' with', '<@\textcolor{blue}{So}@>', ' both', 'From', ' <@\textcolor{blue}{Let}@>', ' this', '<@\textcolor{blue}{Thus}@>', '<@\textcolor{blue}{Wait}@>', ' if', ' <@\textcolor{red}{-}@>', ' <@\textcolor{red}{+}@>', '^', ' only', ' at', '<@\textcolor{blue}{Since}@>', ' integer'
\end{lstlisting}
\end{AIbox}



% \begin{AIbox}{The 50 most frequently clipped tokens from \textbf{positively-rewarded} samples.}
% \begin{lstlisting}
% ' the', ' \\(', ' \\', '\\', ',', ' <@\textcolor{red}{+}@>', '.\n\n', '<@\textcolor{red}{2}@>', ':\n\n', '<@\textcolor{red}{1}@>', ' (', ' a', ' ', '<@\textcolor{blue}{Given}@>', '<@\textcolor{red}{3}@>', '<@\textcolor{red}{4}@>', ' in', '<@\textcolor{blue}{So}@>', ' and', '<@\textcolor{blue}{Wait}@>', ' <@\textcolor{red}{=}@>', ' from', '.', '<@\textcolor{red}{5}@>', '<@\textcolor{red}{6}@>', ' all', ' to', '<@\textcolor{blue}{Now}@>', ' C', '<@\textcolor{red}{7}@>', ' point', ' invalid', '<@\textcolor{red}{8}@>', ' <@\textcolor{blue}{Since}@>', ' I', ' it', ' This', ' with', '<@\textcolor{red}{frac}@>', ' of', ' not', ' is', ' correct', '}', 'We', ' we', ' x', ' differences', ')', ' step'
% \end{lstlisting}
% \end{AIbox}


% \begin{AIbox}{The 50 most frequently clipped tokens from \textbf{negatively-rewarded} samples.}
% \begin{lstlisting}
% ' the', ' \\(', ' \\', ',', ' ', '<@\textcolor{red}{1}@>', ' a', '<@\textcolor{red}{2}@>', ' is', '.\n\n', ':\n\n', ' we', ':\n', ' and', '<@\textcolor{blue}{Wait}@>', ' in', '<@\textcolor{blue}{Given}@>', ' <@\textcolor{red}{=}@>', '<@\textcolor{red}{0}@>', '<@\textcolor{red}{3}@>', '\\', ' for', ' of', '<@\textcolor{blue}{So}@>', 'This', ' each', ' this', '<@\textcolor{blue}{But}@>', ' can', ' to', '{', ' <@\textcolor{red}{+}@>', ' all', ' with', '<@\textcolor{red}{4}@>', ' (', ')', ' are', '<@\textcolor{red}{9}@>', '<@\textcolor{blue}{Since}@>', '.', '<@\textcolor{blue}{Let}@>', ' find', ' if', ' <@\textcolor{blue}{but}@>', '<@\textcolor{blue}{Now}@>', ' The', ':', ' <@\textcolor{blue}{given}@>', ' it'
% \end{lstlisting}
% \end{AIbox}


\section{More Details for Scaling Experiments}
\label{appendix:detailed_experimental_settings}

In this section, we provide detailed training and evaluation settings of the \textbf{scaling experiments} in \Cref{sec:scaling_exp}. 

\textbf{Training Settings.}
We conduct experiments using the VeRL framework~\citep{sheng2024hybridflow} on NVIDIA H Series GPUs. All methods follow the hyperparameter configurations detailed in \Cref{tab:detailed_experimental_settings}. 
Rollout router replay (R3) \citep{ma2025stabilizing} records the routed experts used in the inference engine and replays them in the training engine, which mitigates the training-inference mismatch and stabilizes RL training for MoE models. We only use R3 in the MoE Base w/ R3 experiment and do not use it in all other experiments.
For experiments that utilize LoRA, as suggested by \citet{schulman2025lora}, we employ a larger learning rate of $1\times 10^{-5}$. For the MoE Base w/ LoRA experiment, we set \texttt{lora\_rank=32} and \texttt{lora\_alpha=64}.

As suggested in \Cref{sec:correct_ancher_for_truct_region}, for all methods, we use the behavior policy ($\rolloutpi_{\theta'}$) instead of recomputed policy distribution ($\trainerpi_{\theta'}$) to construct the trust region (i.e., for clipping or masking). Under the unified policy gradient formulation (Equation~\ref{eq:unified_grad}), the method-specific hyperparameters ($C=5$ by default) are configured as follows:

\begin{itemize}  
    \item For \textbf{GRPO-ClipHigher}, we have 
    \begin{equation*}  
    M_t =  
        \begin{cases}  
            0, & \text{if } (\hat{A}_t > 0 \text{ and } r_t > 1 + \epsilon_\text{high}) \text{ or } (\hat{A}_t < 0 \text{ and } r_t < 1 - \epsilon_\text{low}) \\  
            1, & \text{otherwise}.  
        \end{cases}  
    \end{equation*} 
    where $\epsilon_\text{high}=0.27$ and $\epsilon_\text{low}=0.2$, which follows the hyperparameters used in \citet{zheng2025stabilizing}.

    \item For \textbf{CISPO}, we have $M_t = 1$.

    \item For \textbf{DPPO-Binary-KL} and \textbf{DPPO-Binary-TV}, we have 
    \begin{equation*}  
    M_t =  
        \begin{cases}  
            0, & \text{if } (\hat{A}_t > 0 \text{ and } r_t > 1 \text{ and } D_t > \delta) \text{ or } (\hat{A}_t < 0 \text{ and } r_t < 1 \text{ and } D_t > \delta) \\  
            1, & \text{otherwise}.  
        \end{cases}  
    \end{equation*} 
    where $D_t$ is binary approximation of KL or TV as defined in \Cref{sec:method_binary}. For \textbf{DPPO-Binary-KL}, $\delta=0.05$ for all scaling experiments. For \textbf{DPPO-Binary-TV}, we use $\delta=0.15$ for MoE Base w/ LoRA experiment and $\delta=0.2$ for all other scaling experiments.
\end{itemize}


\begin{table}[h]
    % \vspace{-.4cm}
    % \fontsize{7.5}{9}\selectfont
    % \vspace{-0.9cm}
    % \tabcolsep 2.0pt
    \renewcommand{\arraystretch}{1.0}
    \caption{Detailed RL training hyperparameters of scaling experiments.}
    % \vspace{-0.25cm}
    \label{tab:detailed_experimental_settings}
    \centering
    % \input{tables/qwen3_4b_without_condition_y}
\begin{tabular}{l|cccccc}
\toprule
\textbf{Hyperparameters} & MoE Base & MoE Base w/ R3 & MoE Thinking & Dense Base & MoE Base w/ LoRA \\
\midrule
\texttt{max\_prompt\_length} & 1024 & 1024 & 1024 & 1024 & 1024 \\ 
\texttt{max\_response\_length} & 16384 & 16384 & 16384 & 8000 & 8000 \\ 
\texttt{train\_batch\_size} & 256 & 256 & 256 & 128 & 128 \\ 
\texttt{ppo\_mini\_batch\_size} & 32 & 32 & 32 & 32 & 16\\ 
\texttt{optim.lr} & 1e-6 & 1e-6 & 1e-6 & 1e-6 & 1e-5 \\ 
% \texttt{tis\_imp\_ratio\_cap} & 5 & 5 & 5 & 5 & 5 \\ 
\texttt{rollout.temperature} & 1.0 & 1.0  & 1.0 & 1.0 & 1.0 \\ 
\texttt{rollout.n} & 16 & 16 & 16 & 8 & 8\\ 
\midrule
\textbf{Detailed Results} & \Cref{fig:appendix_base_woR3} & \Cref{fig:appendix_base_wR3} & \Cref{fig:appendix_a3b} & \Cref{fig:appendix_8b} & \Cref{fig:appendix_lora} \\
\bottomrule
\end{tabular}
    % \vspace{-.3cm}
\end{table}

\textbf{Evaluation Settings.}
We perform online evaluation for each method and experimental configuration, monitoring AIME24 and AIME25 scores throughout RL training. Evaluations are conducted every 5 training steps for MoE Base, MoE Base w/ R3, and MoE Thinking, and every 10 steps for Dense Base and MoE Base w/ LoRA.

Across all scaling experiments, we use consistent sampling parameters: \texttt{temperature=0.7}, \texttt{top\_p=0.95}, and \texttt{n=32}. The \texttt{n=32} setting indicates that each question from AIME24 and AIME25 is sampled 32 times, and we report the average scores. The \texttt{max\_response\_length} remains identical to that used during training rollouts.


\section{More Empirical Results}

\subsection{Extended Main Results}
\label{appendix:extended_main_results}

In addition to the results provided in \Cref{sec:scaling_exp}, here we provide more detailed results of the five scaling experiments: \Cref{fig:appendix_base_woR3} for MoE Base w/o R3, \Cref{fig:appendix_base_wR3} for MoE Base w/ R3, \Cref{fig:appendix_a3b} for MoE Thinking, \Cref{fig:appendix_8b} for Dense Base, \Cref{fig:appendix_lora} for MoE Base w/ LoRA. We record the following metrics throughout the RL training: training rewards (denoted as ``\textbf{Rewards}''), \textbf{AIME 2024} Avg@32 scores, \textbf{AIME 2025} Avg@32 scores, mean of $\vert\rolloutpi_{\theta'} - \trainerpi_{\theta'}\vert$ (denoted as ``\textbf{Mean of $\vert \pi-\mu \vert$}''), mean of the response length (denoted as ``\textbf{Response Length}''), and mean of token entropy (denoted as ``\textbf{Entropy}''). 
For clearer visualization, all metrics except AIME24 and AIME25 are smoothed using a Gaussian filter with standard deviation $\sigma=2$. The original unsmoothed curves are shown in the background as shaded regions.


\begin{figure}
    \centering  
    \includegraphics[width=\linewidth]{figs/appendix-base_woR3.pdf}  
    \caption{Evolution of metrics for \textbf{MoE Base w/o R3} experiment (based on Qwen3-30B-A3B-Base, without rollout router replay).}  
    \label{fig:appendix_base_woR3}  
\end{figure}

\begin{figure}
    \centering  
    \includegraphics[width=\linewidth]{figs/appendix-base_wR3.pdf}  
    \caption{Evolution of metrics for \textbf{MoE Base w/ R3} experiment (based on Qwen3-30B-A3B-Base, with rollout router replay).}  
    \label{fig:appendix_base_wR3}  
\end{figure}

Overall, across the five experiments, our method DPPO demonstrates consistent and robust improvements in training rewards, highlighting its \textit{stability} and \textit{efficiency}. On both AIME~24 and AIME~25 benchmarks, DPPO exhibits a clear, stable upward trend during training and maintains superior performance after convergence. The stability of our approach is evidenced by learning curves that generally show less fluctuation compared to baseline methods. Its efficiency is reflected in the rapid increase of training rewards and the strong final performance.

DPPO variants consistently demonstrate healthy training dynamics. The training-inference mismatch (measured by the mean absolute deviation $\vert \pi - \mu\vert$) and policy entropy remain within a stable, proper region throughout RL training. DPPO also effectively increases the generated response length across all scaling experiments, except for MoE Thinking. We note that the model Qwen3-30B-A3B already produces extremely long responses; as our training enforces a maximum length of approximately 16k tokens, RL training naturally shortens responses to fit this constraint.

In contrast, the GRPO-ClipHigher baseline, which relies on the ratio clipping mechanism of PPO, shows lower stability than DPPO and achieves inferior final performance in all five large-scale experiments. For example, in MoE Base w/o R3 (see \Cref{fig:appendix_base_woR3}), GRPO-ClipHigher, though more stable than CISPO, improves more slowly and converges to lower training rewards and AIME scores than DPPO. In MoE Thinking (see \Cref{fig:appendix_a3b}), GRPO-ClipHigher suffers a significant training collapse. Notably, GRPO-ClipHigher consistently leads to excessively high entropy in all large-scale experiments, a phenomenon not observed with other methods.

The CISPO baseline, which retains gradients for all tokens, is generally less stable and prone to collapse in certain settings. For instance, in MoE~Base~w/o~R3 (see \Cref{fig:appendix_base_woR3}), CISPO experiences a sudden and severe collapse leading to complete failure. In Dense Base (see \Cref{fig:appendix_8b}), CISPO shows a degenerative trend, particularly on AIME25. In MoE Base w/ LoRA (see \Cref{fig:appendix_lora}), the AIME24 scores, mean of $\vert \pi - \mu\vert$, and response length exhibit noticeable fluctuations, further indicating instability.

We also analyze the effect of rollout router replay (R3). Remarkably, DPPO variants \textit{without} R3 already outperform baselines that use R3, underscoring the importance of a proper masking mechanism in RL training (see \Cref{fig:appendix_base_woR3,fig:appendix_base_wR3}). Furthermore, incorporating R3 yields additional gains for DPPO, suggesting that the benefits of R3 and DPPO are largely orthogonal. This implies that DPPO provides a robust foundation for LLM RL fine-tuning, capable of further improvement even when training-inference mismatch is mitigated by other techniques.


\begin{figure}
    \centering  
    \includegraphics[width=\linewidth]{figs/appendix-a3b.pdf}  
    \caption{Evolution of metrics for \textbf{MoE Thinking} experiment (based on Qwen3-30B-A3B).}  
    \label{fig:appendix_a3b}  
\end{figure}

\begin{figure}
    \centering  
    \includegraphics[width=\linewidth]{figs/appendix-8b.pdf}  
    \caption{Evolution of metrics for \textbf{Dense Base} experiment (based on Qwen3-8B-Base).}  
    \label{fig:appendix_8b}  
\end{figure}

\begin{figure}
    \centering  
    \includegraphics[width=\linewidth]{figs/appendix-lora.pdf}  
    \caption{Evolution of metrics for \textbf{MoE Base w/ LoRA} experiment (based on Qwen3-30B-A3B-Base, with LoRA).}  
    \label{fig:appendix_lora}  
\end{figure}




\subsection{Ablation on TV/KL Approximation}
\label{appendix:ablation_topk_approximation}

In the scaling experiments, we compared DPPO variants using binary TV/KL approximations (Equations~\ref{eq:binary_tv} and \ref{eq:binary_kl}) against several baselines. To further investigate the approximation strategy, we experiment with DPPO variants with top-K TV/KL approximations (Equations~\ref{eq:topk_tv} and \ref{eq:topk_kl}), where we set $K=20$; these variants are denoted as \textbf{DPPO-TopK-TV} and \textbf{DPPO-TopK-KL}. The choice $K=20$ is limited by vLLM \citep{vllm}, which supports returning log probabilities for at most 20 candidate tokens per step. We strictly replicate the experimental setting of MoE Base w/o R3. As in the main scaling experiments, for \textbf{DPPO-Binary-TV} and \textbf{DPPO-TopK-TV} we set the clip threshold $\delta=0.2$, while for \textbf{DPPO-Binary-KL} and \textbf{DPPO-TopK-KL} we set $\delta=0.05$.


\begin{figure}
    \centering  
    \includegraphics[width=\linewidth]{figs/appendix-topk.pdf}  
    \caption{Evolution of metrics for baselines, DPPO with binary TV/KL approximation, and DPPO with Top-K (K=20) approximation under the same setting as MoE Base w/o R3.}  
    \label{fig:appendix_topk}  
\end{figure}


As presented in \Cref{fig:appendix_topk}, introducing the top-K approximation does not yield significant performance gains, indicating that the simpler binary approximation already provides a sufficient and efficient proxy for constructing the trust region. This finding is encouraging, suggesting that DPPO with binary TV/KL remains highly scalable without sacrificing effectiveness.


\label{appendix:more_settings}
\begin{figure*}[h]
    \centering
    \includegraphics[width=\linewidth]{figs/more_settings.pdf}
    \caption{Learning curve comparison of using ratio (PPO-Ratio) and TV divergence (DPPO-Binary-TV) for the trust region clipping.}
    \label{fig:more_settings}
\end{figure*}

\subsection{Extended Results for Different Model $\times$ Task Combinations}
\label{appendix:extended_models_tasks}


Besides experimental results presented in \Cref{sec:scaling_exp}, we evaluate DPPO on more model $\times$ task settings to validate its advantage over the GRPO baseline. The settings we considered include:

\begin{enumerate}
    \item \textbf{Different model family}. Training on a new model different from the Qwen family, OctoThinker-3B-Hybrid-Base~\citep{wang2025octothinker}, on the standard math reasoning dataset~\citep{hendrycks2021measuring}.

    \item \textbf{Abstract reasoning and induction}. Training the Qwen3-1.7B-Base model on abstract reasoning task (Arc1D) and induction task (Acre) from the Gem library~\citep{liu2025gem}.

    \item \textbf{Multi-turn reasoning}. Training the Qwen3-1.7B-Base model on the multi-turn reasoning environment (Sudoku-v0-easy) from Gem the library~\citep{liu2025gem}.
\end{enumerate}

The training is conducted using Oat~\citep{liu2025oat} with their example scripts (thereby the standard hyper-parameters) for math RL and Gem RL. For the TV divergence clipping, we use a threshold of $\delta=0.2$.
\Cref{fig:more_settings} shows the comparison between the TV variant of DPPO and the vanilla ratio-based PPO, both based on the GRPO algorithmic framework with the only difference being the trust region masking strategy. We can observe DPPO improves the efficiency (and sometimes asymptotic performance) over the baseline across different settings, validating its general effectiveness.

%%%%%%%%%










\end{document}
\documentclass[11pt]{article}

% tex/env.tex — Packages and environment configuration
\usepackage[utf8]{inputenc}
\usepackage[T1]{fontenc}
\usepackage{amsmath,amssymb,amsthm}
\usepackage{mathtools}
\usepackage{booktabs}
\usepackage{graphicx}
\usepackage{xcolor}
\usepackage{algorithm}
\usepackage{algorithmic}
\usepackage{hyperref}
\usepackage{cleveref}
\usepackage[margin=1in]{geometry}
\usepackage{natbib}
\usepackage{thmtools}
\usepackage{listings}

% Listing style for Python code
\lstset{
  language=Python,
  basicstyle=\ttfamily\small,
  keywordstyle=\color{blue},
  commentstyle=\color{gray},
  frame=single,
  breaklines=true
}

% tex/math.tex — Math symbols, operators, and shared notation
%
% NOTE: theorem/lemma/definition environments are defined in `tex/env.tex` to
% keep all environment configuration in a single place, aligned with the
% reference material under `tex/literature/`.

%==============================================================================
% MATH OPERATORS
%==============================================================================
\DeclareMathOperator*{\argmax}{arg\,max}
\DeclareMathOperator*{\argmin}{arg\,min}
\DeclareMathOperator{\sign}{sign}
\DeclareMathOperator{\Tr}{Tr}

%==============================================================================
% PROBABILITY AND STATISTICS
%==============================================================================
\newcommand{\E}{\mathbb{E}}
\newcommand{\Var}{\mathrm{Var}}
\newcommand{\Cov}{\mathrm{Cov}}
\newcommand{\R}{\mathbb{R}}

%==============================================================================
% POLICY NOTATION
%==============================================================================
\newcommand{\piroll}{\pi_{\mathrm{roll}}}
\newcommand{\pitheta}{\pi_{\theta}}
\newcommand{\piold}{\pi_{\mathrm{old}}}
\newcommand{\piref}{\pi_{\mathrm{ref}}}
\newcommand{\ylt}{y_{<t}}
\newcommand{\yle}{y_{\le t}}

%==============================================================================
% DIVERGENCE MEASURES (following TRM paper convention)
%==============================================================================
\newcommand{\DKL}{D_{\mathrm{KL}}}
\newcommand{\DTV}{D_{\mathrm{TV}}}
\newcommand{\Dkltok}{D_{\mathrm{KL}}^{\mathrm{tok}}}
\newcommand{\Dtvtok}{D_{\mathrm{TV}}^{\mathrm{tok}}}
\newcommand{\Dkltokmax}{D_{\mathrm{KL}}^{\mathrm{tok,max}}}
\newcommand{\Dtvtokmax}{D_{\mathrm{TV}}^{\mathrm{tok,max}}}
\newcommand{\Dklseq}{D_{\mathrm{KL}}^{\mathrm{seq}}}
\newcommand{\Dtvseq}{D_{\mathrm{TV}}^{\mathrm{seq}}}
\newcommand{\Dbar}{\bar{D}}

%==============================================================================
% COMMON MATH SHORTCUTS
%==============================================================================
\newcommand{\Ls}{\mathcal{L}}
\def\eps{{\epsilon}}
\def\ceil#1{\lceil #1 \rceil}
\def\floor#1{\lfloor #1 \rfloor}


\title{REINFORCE Pro Max: Variance-Reduced Advantage Estimation\\
with Causal Off-Policy Correction for LLM-RL}

\author{Anonymous}
\date{}

\begin{document}

\maketitle

\begin{abstract}
Critic-free reinforcement learning methods for Large Language Models (LLMs), including REINFORCE++, GRPO, and RLOO, eliminate the need for
a value network and save significant GPU memory. However, they face two intertwined challenges: high-variance advantage estimation and
off-policy mismatch between inference engines and training frameworks. We present REINFORCE Pro Max, a unified framework addressing both
challenges with theoretical analysis. The first component, REINFORCE Max, combines a leave-one-out (RLOO) baseline---which provides
statistical independence between each sample and its baseline, enabling clean second-moment analysis---with an adaptive asymmetric
normalization that preserves gradient direction while enforcing zero empirical mean and unit empirical variance on non-zero tokens. The
second component, REINFORCE Pro, introduces prefix cumulative importance sampling correction that preserves the causal structure of
autoregressive generation. We show that, under explicit sufficient conditions on the per-token log-ratios stated in
Theorem~\ref{thm:prefix-tighter}, prefix cumulative IS can provide tighter masking (i.e., mask at least as many off-policy tokens) than
both token-level methods (which ignore causal dependencies) and sequence-level methods (which lose positional information) in three
analyzed deviation patterns (early deviation, late deviation, and monotone drift). We relate prefix cumulative IS to the per-position
structure of the Adaptive bound in trust region analyses: the KL chain rule yields an expectation identity for the cumulative log-ratio,
and thresholding the sample-level prefix log-ratio provides a practical, per-position proxy for trust-region style filtering. Together,
the two components target the two factors in the surrogate objective error decomposition: normalization stabilizes the advantage magnitude,
while prefix masking filters out tokens with large cumulative $\piold/\piroll$ mismatch.
\end{abstract}


%==============================================================================
% INTRODUCTION
%==============================================================================
\section{Introduction}

Reinforcement Learning (RL) has become a widely adopted paradigm for aligning Large Language Models (LLMs) with human preferences and improving their reasoning capabilities~\citep{zeng2025simplerl, guo2025deepseek}. As tasks shift from short-form chat to long-horizon reasoning---mathematical problem solving, code generation, and multi-step planning---the computational cost of maintaining a critic (value) network becomes prohibitive. Critic-free methods such as REINFORCE++ Baseline~\citep{hu2025reinforce}, GRPO~\citep{shao2024deepseekmath}, and RLOO~\citep{ahmadian2024back} have emerged as practical alternatives, eliminating the critic and saving approximately 25\% of GPU memory in comparable settings~\citep{hu2025reinforce}.

Despite their efficiency, critic-free methods face two fundamental challenges that become increasingly severe as sequence lengths grow:

\textbf{Challenge 1: High-variance advantage estimation.} Without a learned value function, the advantage must be estimated from reward signals alone. The standard mean baseline $b = \frac{1}{n}\sum_j r_j$ introduces correlation between each sample's baseline and its own reward, which can inflate gradient variance. Global normalization $\hat{A} = (A - \mu)/\sigma$ applies a uniform scaling that distorts the relative magnitude of positive and negative advantages, particularly problematic in reasoning tasks where reward distributions are highly skewed (most samples fail, few succeed).

\textbf{Challenge 2: Causal off-policy mismatch.} PPO-style LLM-RL optimizes a policy $\pitheta$ using trajectories generated by a fixed
rollout policy $\piroll$ (often implemented as a frozen ``old'' snapshot). Distributed staleness, numerical effects, and architectural
discontinuities (e.g., MoE routing flips~\citep{liu2024deepseek}) can amplify the divergence between $\piroll$ and $\pitheta$, producing
large per-token importance ratios $\rho_t=\pitheta(y_t\mid c_t)/\piroll(y_t\mid c_t)$. Because generation is autoregressive, an early ratio
deviation shifts the
future context distribution, so token-independent ratio interventions may fail to respect the causal structure of error accumulation.

These challenges are not independent: noisy advantage estimates amplify the effect of off-policy errors, and biased importance sampling weights distort the advantage signal, creating a compounding instability. Formally, the surrogate objective error decomposes into per-position terms involving both an advantage factor and a context distribution shift factor (see \Cref{eq:error-decomp}); high variance inflates the former while off-policy mismatch inflates the latter. Existing methods address each challenge in isolation---RLOO improves the baseline but ignores off-policy correction; TIS and ICEPOP correct for policy mismatch but ignore causal structure and use suboptimal advantage estimation.

We present \textbf{REINFORCE Pro Max}, a unified framework that addresses both challenges simultaneously. By combining variance-reduced
advantage estimation (\Cref{sec:reinforce-max}) with prefix-level Causal Trust Region masking (\Cref{sec:reinforce-pro}), both factors in the error
decomposition are targeted---the advantage magnitude via normalization and the context-shift factor via sample-level prefix filtering; the
joint benefit is discussed in \Cref{sec:unified}. Our contributions are:

\begin{enumerate}
    \item \textbf{Leave-one-out baseline:} We adopt the leave-one-out (RLOO) baseline for its statistical independence between each sample and its baseline, avoiding correlated baseline artifacts. See \Cref{app:rloo-proof} for a detailed analysis.

    \item \textbf{Adaptive asymmetric normalization} (\Cref{sec:adaptive-norm}): We introduce a normalization scheme that applies
    different scaling factors $\alpha, \beta$ to positive and negative advantages. The scheme provably preserves gradient direction while
    enforcing zero empirical mean and unit empirical variance on non-zero tokens (the constraints in \Cref{def:adaptive-norm}).

    \item \textbf{Prefix Causal Trust Region masking with causal structure (\Cref{sec:prefix-is}):} We propose a prefix-cumulative masking
    strategy constructed from the per-token ratio $\rho_t=\pitheta/\piroll$. Motivated by long-horizon trust-region analyses for LLM-RL
    (e.g., TRM~\citep{li2025trm}), we apply a per-position prefix gate to the token-level loss, respecting the autoregressive causal
    structure. This differs from TRM, which uses a sequence-level acceptance gate $M(x,y)$ to discard entire trajectories.
    Under the explicit sufficient conditions in Theorem~\ref{thm:prefix-tighter}, prefix masking can provide tighter filtering (i.e., mask at
    least as many ratio-violating tokens) than token-level methods (which ignore causal dependencies) and sequence-level methods (which lose
    positional information) in three analyzed deviation patterns (early deviation, late deviation, and monotone drift).

    \item \textbf{Connection to the Adaptive bound (\Cref{sec:adaptive-connection}):} We relate prefix masking to the per-position
    structure of the Adaptive bound from the trust-region framework~\citep{li2025trm}: the KL chain rule yields an expectation identity for
    cumulative log-ratios, and thresholding the sample-level prefix log-ratio provides a practical proxy for Causal Trust Region filtering at
    each position.
\end{enumerate}

\textbf{Paper organization.} \Cref{sec:preliminaries} introduces notation, assumptions, and the surrogate objective error decomposition.
\Cref{sec:reinforce-max} presents REINFORCE Max (leave-one-out baseline, token expansion, and adaptive normalization).
\Cref{sec:reinforce-pro} develops REINFORCE Pro (prefix-level Causal Trust Region masking and its connection to the Adaptive bound).
\Cref{sec:unified} unifies both components and presents the full algorithm. \Cref{sec:related} discusses related work,
\Cref{sec:experiments} describes experimental evaluation, and \Cref{sec:conclusion} concludes.



%==============================================================================
% PRELIMINARIES
%==============================================================================
\section{Preliminaries}
\label{sec:preliminaries}

\subsection{Autoregressive Generation and Objective}

A policy $\pitheta$ generates a response $y = (y_1, \ldots, y_T)$ given prompt $x$, with trajectory probability $P^{\pitheta}(y \mid x) = \prod_{t=1}^T \pitheta(y_t \mid x, y_{<t})$. We define the context $c_t = (x, y_{<t})$ and the context visitation distribution (i.e., the marginal distribution over length-$t$ prefixes):
\begin{equation}
    d_t^{\pi}(c_t) = P(x) \prod_{s=1}^{t-1} \pi(y_s \mid c_s).
\end{equation}
Given a scalar reward $R(x, y)$, the objective is $J(\pitheta) = \E_{x \sim P(x),\, y \sim \pitheta(\cdot|x)}[R(x, y)]$.

\subsection{Assumptions}
\label{sec:assumptions}

The following assumptions are used throughout the paper:

\begin{assumption}[Policy Support Overlap]
\label{asm:support}
For all contexts $c_t$ and tokens $y_t$ in the vocabulary, $\piroll(y_t|c_t) > 0 \implies \piold(y_t|c_t) > 0$. This ensures that the importance sampling ratio $w_t = \piold(y_t|c_t)/\piroll(y_t|c_t)$ is well-defined.
\end{assumption}

\begin{assumption}[Bounded Rewards]
\label{asm:bounded-reward}
There exists $R_{\max} < \infty$ such that $|R(x, y)| \le R_{\max}$ for all prompt-response pairs $(x, y)$.
\end{assumption}

\begin{assumption}[Finite Sequence Length]
\label{asm:finite-length}
The response length satisfies $T < \infty$ for all generated sequences.
\end{assumption}

\subsection{The Surrogate Objective and Error Decomposition}

Samples are generated from a rollout policy $\piroll$ that generally differs from $\pitheta$. Following~\citet{kakade2002approximately}, the standard surrogate objective uses per-step advantages $A_t^{\piroll}(c_t, y_t) := \E_{\piroll}[R \mid c_t, y_t] - \E_{\piroll}[R \mid c_t]$:
\begin{equation}
    L_{\piroll}(\pitheta) = \sum_{t=1}^T \E_{c_t \sim d_t^{\piroll}}\!\Big[\E_{y_t \sim \piroll(\cdot|c_t)}\!\big[\rho_t \, A_t^{\piroll}(c_t, y_t)\big]\Big],
    \label{eq:surrogate}
\end{equation}
where $\rho_t = \pitheta(y_t \mid c_t) / \piroll(y_t \mid c_t)$ is the per-token importance ratio.

\textbf{From per-step to trajectory-level advantage.} The per-step advantage $A_t^{\piroll}(c_t, y_t) = \E_{\piroll}[R \mid c_t, y_t] - \E_{\piroll}[R \mid c_t]$ depends on the full future trajectory conditioned on $(c_t, y_t)$ and is generally intractable without a learned value function. In the critic-free setting, the REINFORCE estimator replaces it with the trajectory-level advantage $A = R(x,y) - b$, where $b$ is a prompt-level baseline independent of the current trajectory. This trajectory-level advantage is constant across all positions within a trajectory, effectively substituting the future-conditional per-step quantity with a trajectory-wide scalar. Substituting this constant advantage into Eq.~\eqref{eq:surrogate} yields the simplified surrogate $\E_{\piroll}[A \cdot \sum_t \rho_t]$, which is the form used in our algorithm (\Cref{alg:promax}).

\textbf{Gradient estimator unbiasedness.} It is important to distinguish two properties. The REINFORCE identity guarantees that $\E_{\piroll}[A \cdot \sum_t \nabla_\theta \log \pitheta(y_t|c_t)] = \nabla_\theta J(\pitheta)$ when $\piroll = \pitheta$ (on-policy), providing an unbiased estimator of the \emph{policy gradient} $\nabla J(\pitheta)$. However, this is not the same as being an unbiased estimator of $\nabla_\theta L_{\piroll}(\pitheta)$: the surrogate in Eq.~\eqref{eq:surrogate} uses per-step advantages, while the REINFORCE estimator uses the trajectory-level advantage. Our algorithm uses the trajectory-level advantage as a practical approximation; its theoretical guarantee comes from the REINFORCE identity's gradient unbiasedness, not from strict equivalence with the PDI surrogate. The error decomposition (Eq.~\eqref{eq:error-decomp}) is stated in terms of the per-step surrogate and serves as the theoretical motivation for off-policy correction, even though the implemented loss uses the trajectory-level approximation. The approximation error is:
\begin{equation}
    \mathrm{Error}(\pitheta) := J(\pitheta) - J(\piroll) - L_{\piroll}(\pitheta).
    \label{eq:error-def}
\end{equation}
Via the Performance Difference Identity~\citep{kakade2002approximately}, this error decomposes as:
\begin{equation}
    |\mathrm{Error}| \le \sum_{t=1}^T 2\|g_t\|_\infty \cdot \|d_t^{\pitheta} - d_t^{\piroll}\|_{\mathrm{TV}},
    \label{eq:error-decomp}
\end{equation}
where $g_t(c_t) = \E_{y_t \sim \pitheta}[A_t^{\piroll}(c_t, y_t)]$ is the expected advantage shift, with the per-step advantage defined as $A_t^{\piroll}(c_t, y_t) := \E_{\piroll}[R \mid c_t, y_t] - \E_{\piroll}[R \mid c_t]$. Classical bounds on this error scale as $O(T^2)$~\citep{kakade2002approximately, achiam2017constrained}, which is vacuous for long-horizon LLM tasks ($T > 4000$). Recent work~\citep{li2025trm} derives tighter bounds: the Pinsker-Marginal bound scales as $O(T^{3/2})$ using a global TV divergence; the Mixed bound achieves $O(T)$ by combining TV and KL terms; and the Adaptive bound is data-dependent, operating at per-position granularity. All three depend on the maximum token-level divergence $\Dkltokmax$, but their dependency forms differ: the Pinsker-Marginal bound uses a global aggregation, the Mixed bound combines TV and KL terms with different weighting, and the Adaptive bound uses per-position $\Dbar_t$ with a position-dependent minimum over three terms.

The Adaptive bound is particularly relevant to our work:
\begin{equation}
    B_{\mathrm{Adap}}^{*} = 4 \sum_{t=1}^{T} \Dbar_t \cdot \min\!\Big(1,\;(T\!-\!t)\,\epsilon,\;\sqrt{\tfrac{(T-t)\,\delta}{2}}\Big),
    \label{eq:adaptive-bound}
\end{equation}
where $\Dbar_t = \E_{c_t \sim d_t^{\piroll}}[\Dtvtok(c_t)]$ is the expected per-position TV divergence, $\epsilon = \Dtvtokmax$, and $\delta = \Dkltokmax$. This bound operates at per-position granularity and preserves the causal structure of the error accumulation---properties that motivate our prefix cumulative IS correction.

\subsection{Three Policies in LLM-RL}

Modern LLM-RL pipelines involve three distinct policies:
\begin{itemize}
    \item $\piroll$: the rollout policy used by vLLM to generate samples;
    \item $\piold$: the actor policy at the start of each training step (the ``old'' policy in PPO);
    \item $\pitheta$: the current actor policy being updated during gradient steps.
\end{itemize}
The standard PPO ratio $\rho_t^{\mathrm{PPO}} = \pitheta(y_t|c_t) / \piold(y_t|c_t)$ accounts for the drift between $\pitheta$ and $\piold$ during mini-batch updates. The IS correction ratio $w_t = \piold(y_t|c_t) / \piroll(y_t|c_t)$ accounts for the mismatch between the rollout engine and the training actor. In practice, $\rho_t^{\mathrm{PPO}}$ carries gradients while $w_t$ is treated as a fixed coefficient; see \Cref{sec:combined-loss} for implementation details.

\subsection{Existing IS Correction Methods}
\label{sec:existing-is}

Several methods have been proposed to correct the $\piroll \neq \piold$ mismatch. Let $\ell_t = \log \piold(y_t|c_t) - \log \piroll(y_t|c_t)$ denote the per-token log IS ratio.

\begin{table}[h]
\centering
\caption{Existing IS correction methods for off-policy mismatch in LLM-RL.}
\label{tab:existing-is}
\renewcommand{\arraystretch}{1.2}
\begin{tabular}{llll}
\toprule
\textbf{Method} & \textbf{IS Weight} & \textbf{Granularity} & \textbf{Causal} \\
\midrule
TIS & $\exp(\ell_t)$ clamped to $[\lambda, \Lambda]$ & Token & No \\
ICEPOP & $\exp(\ell_t) \cdot \mathbb{I}[\exp(\ell_t) \in [\lambda, \Lambda]]$ & Token & No \\
seq-mask-tis & $\mathbb{I}[\exp(\bar{\ell}) \in [\lambda, \Lambda]] \cdot \exp(\ell_t)$ & Sequence & No \\
GSPO & $\exp\!\big(\sum_t \ell_t m_t / \sum_t m_t\big)$ & Sequence & No \\
\bottomrule
\end{tabular}
\end{table}

\noindent Here $\bar{\ell} = \sum_t \ell_t m_t / \sum_t m_t$ is the sequence-level geometric mean log-ratio and $m_t$ is the action mask. Most existing methods in the critic-free LLM-RL setting share a common limitation: they either operate at the token level (ignoring that early deviations affect the entire future trajectory) or at the sequence level (losing the ability to distinguish where divergence occurs within a sequence).

\begin{lemma}[Context Shift Propagation]
\label{lem:coupling}
Let $s$ be the \emph{first} position at which two policies $\pi$ and $\pi'$ differ: $\pi(\cdot|c_{s'}) = \pi'(\cdot|c_{s'})$ for all reachable $c_{s'}$ with $s' < s$, and $\Dtvtok(c_s) > 0$ for some context $c_s$ satisfying $d_s^{\pi}(c_s) > 0$. If for all subsequent positions $t > s$ the policies assign positive probability to all tokens reachable under either policy (i.e., $\pi(y_t|c_t) > 0 \iff \pi'(y_t|c_t) > 0$ for all reachable $c_t$), then for all $t > s$, the context visitation distributions satisfy $\|d_t^{\pi} - d_t^{\pi'}\|_{\mathrm{TV}} > 0$. That is, a token-level divergence at the first point of disagreement propagates to all future context distributions.
\end{lemma}

\begin{proof}
By definition, $d_t^{\pi}(c_t) = P(x)\prod_{s'=1}^{t-1}\pi(y_{s'}|c_{s'})$. We proceed by explicit construction.

\emph{Base case ($t = s+1$).} Since $\Dtvtok(c_s) > 0$, there exists a token $v^*$ such that $\pi(v^*|c_s) \neq \pi'(v^*|c_s)$. Consider the context $c_{s+1} = (c_s, v^*)$. By the product structure of the visitation distribution:
\begin{equation*}
    d_{s+1}^{\pi}(c_s, v^*) = d_s^{\pi}(c_s) \cdot \pi(v^*|c_s), \qquad d_{s+1}^{\pi'}(c_s, v^*) = d_s^{\pi}(c_s) \cdot \pi'(v^*|c_s),
\end{equation*}
where $d_s^{\pi}(c_s) = d_s^{\pi'}(c_s)$ holds because $s$ is the first position of disagreement, so $\pi(\cdot|c_{s'}) = \pi'(\cdot|c_{s'})$ for all reachable $c_{s'}$ with $s' < s$, and hence the two policies induce identical prefix distributions up to position $s$. Since $d_s^{\pi}(c_s) > 0$ and $\pi(v^*|c_s) \neq \pi'(v^*|c_s)$, we have $d_{s+1}^{\pi}(c_s, v^*) \neq d_{s+1}^{\pi'}(c_s, v^*)$, and therefore $\|d_{s+1}^{\pi} - d_{s+1}^{\pi'}\|_{\mathrm{TV}} > 0$.

\emph{Inductive step ($t > s+1$).} We construct a path along which the two policies \emph{agree} at all positions after $s$, so that the visitation ratio is determined entirely by the base-case divergence. By the support overlap condition, for each position $s' > s$ there exist tokens reachable under both policies. Choose any sequence $(y_{s+1}^*, \ldots, y_{t-1}^*)$ such that $\pi(y_{s'}^*|c_{s'}^*) = \pi'(y_{s'}^*|c_{s'}^*)$ for all $s+1 \le s' \le t-1$ (such tokens exist: since both policies are distributions over the same vocabulary with overlapping support, and they can only differ on a strict subset of tokens, there must exist at least one token on which they agree at each position). Along this path, the visitation ratio satisfies:
\begin{equation*}
    \frac{d_t^{\pi}(c_t^*)}{d_t^{\pi'}(c_t^*)} = \frac{d_{s+1}^{\pi}(c_s, v^*)}{d_{s+1}^{\pi'}(c_s, v^*)} \cdot \prod_{s'=s+1}^{t-1} \frac{\pi(y_{s'}^*|c_{s'}^*)}{\pi'(y_{s'}^*|c_{s'}^*)} = \frac{d_{s+1}^{\pi}(c_s, v^*)}{d_{s+1}^{\pi'}(c_s, v^*)} \neq 1,
\end{equation*}
where the product equals 1 by construction and the final inequality follows from the base case. This exhibits a specific context $c_t^*$ at which the two visitation distributions differ, so $\|d_t^{\pi} - d_t^{\pi'}\|_{\mathrm{TV}} > 0$. The exact magnitude can be bounded via the KL chain rule and Pinsker's inequality (see \Cref{thm:prefix-adaptive}).
\end{proof}



%==============================================================================
% METHOD: REINFORCE MAX + REINFORCE PRO
%==============================================================================
\section{REINFORCE Max: Variance-Reduced Advantage Estimation}
\label{sec:reinforce-max}

Given a prompt $x$, we sample $n$ responses $\{y^{(1)}, \ldots, y^{(n)}\}$ with rewards $\{r_1, \ldots, r_n\}$. Each response
$y^{(i)}$ consists of $T_i$ action tokens. REINFORCE Max combines three ingredients: a leave-one-out baseline
(\Cref{eq:rloo-baseline}), token-level expansion (\Cref{sec:token-expand}), and adaptive normalization (\Cref{sec:adaptive-norm}).

We adopt the leave-one-out (RLOO) baseline~\citep{ahmadian2024back} for its statistical independence, which avoids correlated baseline artifacts present in the mean baseline (see \Cref{app:rloo-proof} for a detailed analysis).

\begin{definition}[Leave-One-Out Baseline]
\label{def:rloo}
Assume $n \ge 2$. For sample $i$, the baseline and shaped reward are:
\begin{equation}
    b_i = \frac{1}{n-1}\sum_{j \neq i} r_j, \qquad \tilde{r}_i = r_i - b_i.
    \label{eq:rloo-baseline}
\end{equation}
\end{definition}

\subsection{Token-Level Expansion}
\label{sec:token-expand}

Since the reward is a scalar per response and we use $\gamma = 1.0$ (see \Cref{app:gamma-one} for a formal justification), the token-level advantage is simply the shaped reward itself:
\begin{equation}
    A_{i,t} = \tilde{r}_i = r_i - b_i, \quad \forall\, t \in \{1, \ldots, T_i\}.
    \label{eq:token-advantage}
\end{equation}

\subsection{Adaptive Asymmetric Normalization}
\label{sec:adaptive-norm}

Standard normalization $\hat{A} = (A - \mu)/\sigma$ applies a uniform scaling to all advantages. In reasoning tasks, the advantage distribution is typically highly skewed: most samples fail (negative advantages) while few succeed (positive advantages). Uniform scaling distorts the relative balance between reinforcement and punishment signals.

\begin{definition}[Adaptive Normalization]
\label{def:adaptive-norm}
Fix a prompt group with token-level advantages $\{A_{i,t}\}$, and exclude zero-advantage positions from the normalization statistics.
Define the index sets and summary statistics
\begin{align}
    \mathcal{P}
    &:= \{(i,t) : A_{i,t} > 0\},
    &
    \mathcal{N}
    &:= \{(i,t) : A_{i,t} < 0\},
    \\
    S^+
    &:= \sum_{(i,t) \in \mathcal{P}} A_{i,t},
    &
    S^-
    &:= \sum_{(i,t) \in \mathcal{N}} A_{i,t},
    \\
    Q^+
    &:= \sum_{(i,t) \in \mathcal{P}} A_{i,t}^2,
    &
    Q^-
    &:= \sum_{(i,t) \in \mathcal{N}} A_{i,t}^2,
    \\
    N
    &:= |\mathcal{P}| + |\mathcal{N}|.
\end{align}
The normalized advantage is defined by a sign-dependent rescaling:
\begin{equation}
    \hat{A}_{i,t} = \begin{cases} \alpha \cdot A_{i,t} & \text{if } A_{i,t} > 0, \\ \beta \cdot A_{i,t} & \text{if } A_{i,t} < 0, \\ 0 & \text{if } A_{i,t} = 0, \end{cases}
    \label{eq:adaptive-norm}
\end{equation}
where $\alpha,\beta>0$ are chosen to satisfy the following \emph{empirical} constraints over non-zero tokens:
\begin{align}
    \frac{1}{N}\sum_{(i,t)\in\mathcal{P}\cup\mathcal{N}}\hat{A}_{i,t} &= 0,
    \label{eq:adaptive-mean}
    \\
    \frac{1}{N}\sum_{(i,t)\in\mathcal{P}\cup\mathcal{N}}\hat{A}_{i,t}^2 &= 1.
    \label{eq:adaptive-var}
\end{align}
Equivalently, these constraints can be written as
\begin{equation}
    \alpha S^+ + \beta S^- = 0,
    \qquad
    \alpha^2 Q^+ + \beta^2 Q^- = N.
\end{equation}
\end{definition}

\begin{proposition}[Closed-Form Solution]
\label{prop:alpha-beta}
Assume $\mathcal{P} \neq \emptyset$ and $\mathcal{N} \neq \emptyset$. The unique solution satisfying $\alpha S^+ + \beta S^- = 0$ and $\alpha^2 Q^+ + \beta^2 Q^- = N$ is:
\begin{equation}
    \alpha = \sqrt{\frac{N}{Q^+ + \left(\frac{S^+}{S^-}\right)^{\!2} Q^-}}, \qquad \beta = -\alpha \cdot \frac{S^+}{S^-}.
    \label{eq:alpha-beta}
\end{equation}
\end{proposition}

\begin{proof}
We solve the two scalar constraints in \Cref{def:adaptive-norm} and then enforce $\alpha,\beta>0$.

\textbf{Step 1 (zero empirical mean).} The constraint $\alpha S^+ + \beta S^- = 0$ implies
\begin{equation}
    \beta = -\alpha \cdot \frac{S^+}{S^-}.
    \label{eq:beta-from-mean}
\end{equation}
Since $\mathcal{P}\neq\emptyset$ and $\mathcal{N}\neq\emptyset$, we have $S^+>0$ and $S^-<0$, so $S^-/S^+$ is well-defined.

\textbf{Step 2 (unit empirical second moment).} Substituting \eqref{eq:beta-from-mean} into $\alpha^2 Q^+ + \beta^2 Q^- = N$ yields
\begin{align}
    \alpha^2 Q^+ + \beta^2 Q^-
    &= N,
    \\
    \alpha^2 Q^+ + \alpha^2\Big(\frac{S^+}{S^-}\Big)^{\!2} Q^-
    &= N,
    \\
    \alpha^2
    &= \frac{N}{Q^+ + \left(\frac{S^+}{S^-}\right)^{2} Q^-}.
\end{align}
The denominator is strictly positive because $Q^+>0$, $Q^->0$, and $N>0$ under the same assumption. Taking the positive square root
gives the stated $\alpha>0$, and then \eqref{eq:beta-from-mean} yields the stated $\beta$.

\textbf{Uniqueness.} The mean constraint uniquely specifies the ratio $\beta/\alpha$, and the second-moment constraint uniquely
specifies $\alpha^2$. Enforcing $\alpha>0$ and $\beta>0$ therefore yields a unique solution.
\end{proof}

\begin{proposition}[Advantage Sign Preservation]
\label{prop:gradient-direction}
The adaptive normalization preserves the sign of the advantage at every token: $\sign(\hat{A}_{i,t}) = \sign(A_{i,t})$ for all $(i,t)$
with $A_{i,t} \neq 0$. The relative ordering of magnitudes within $\mathcal{P}$ and within $\mathcal{N}$ is also preserved.
Consequently, the policy gradient update reinforces (or penalizes) the same tokens as the unnormalized advantage.
\end{proposition}

\begin{proof}
Since $\alpha > 0$ and $\beta > 0$ (\Cref{prop:alpha-beta}), multiplication by a positive constant preserves both sign and relative ordering.
\end{proof}

\begin{remark}[Connection to asymmetric clipping]
Adaptive normalization is a soft analog of PPO's asymmetric clipping. PPO clips the importance ratio differently depending on the advantage sign; adaptive normalization scales the advantage magnitude differently. Both control the effective step size asymmetrically, but normalization acts through the advantage rather than the ratio, avoiding the gradient leakage problem identified by~\citet{li2025trm}.
\end{remark}

\begin{remark}[Uniform Scale Mode]
When all $n$ samples share the same reward, the leave-one-out baseline yields zero advantages, eliminating gradient signal. An optional \emph{uniform scale} mode recovers this signal by setting $\tilde{r}_i = r_i/n$ and skipping normalization, improving sample utilization. See \Cref{app:uniform-scale} for details.
\end{remark}


\section{REINFORCE Pro: Prefix-Cumulative Ratio Masking}
\label{sec:reinforce-pro}

\subsection{Two-Policy Ratio Setup}
\label{sec:offpolicy-problem}

We adopt the standard \emph{two-policy} view of PPO-style LLM-RL: samples are generated by a fixed rollout policy $\piroll$, and the
training policy $\pitheta$ is updated from $\piroll$ by gradient steps. This matches the trust-region setup in~\citet{li2025trm}.

The per-token importance ratio between these two policies is
\begin{equation}
    \rho_t = \frac{\pitheta(y_t \mid c_t)}{\piroll(y_t \mid c_t)},
\end{equation}
as defined in \eqref{eq:ratio}. Throughout this section, we restrict attention to masking strategies constructed solely from
$(\rho_t)_{t=1}^T$ (or equivalently, the log-ratios $r_t=\log\rho_t$ in \eqref{eq:log-ratio}), and we do not introduce additional
intermediate policies.

\subsection{Limitations of Existing Methods}
\label{sec:is-limitations}

\begin{proposition}[Token-Level IS Ignores Causal Structure]
\label{prop:token-is-fails}
Fix thresholds $0<\lambda\le 1\le \Lambda$. Consider a length-$T$ trajectory such that
\begin{equation}
    \rho_1 \notin [\lambda,\Lambda],
    \qquad
    \rho_t \in [\lambda,\Lambda]\ \text{ for all }\ t\in\{2,\ldots,T\}.
    \label{eq:token-early-deviation}
\end{equation}
Then the token-level ICEPOP mask,
\begin{equation}
    M_t^{\mathrm{token}} := \mathbb{I}\!\big[\rho_t \in [\lambda,\Lambda]\big],
\end{equation}
masks only the first position. In particular, it does not prevent later positions $t\ge 2$ from contributing gradients even though the
contexts $(c_t)_{t\ge2}$ are sampled under the behavior distribution induced by $\piroll$, which differs from the context distribution that
would be induced by the updated policy $\pitheta$ when $\rho_1 \neq 1$.
\end{proposition}

\begin{proof}
By definition, \eqref{eq:token-early-deviation} implies $M_1^{\mathrm{token}}=0$ and $M_t^{\mathrm{token}}=1$ for all $t\ge2$.

On the other hand, since $\rho_1 \neq 1$, we have $\pitheta(y_1\mid c_1) \neq \piroll(y_1\mid c_1)$, hence the conditional token
distributions $\pitheta(\cdot\mid c_1)$ and $\piroll(\cdot\mid c_1)$ are not equal. Therefore $\Dtvtok(c_1;\piroll,\pitheta)>0$.
Applying the context-shift propagation lemma (\Cref{lem:coupling}) with $(\pi,\pi')=(\piroll,\pitheta)$ and $s=1$ yields
\begin{equation}
    \|d_t^{\piroll} - d_t^{\pitheta}\|_{\mathrm{TV}} > 0
    \qquad
    \text{for all } t \ge 2.
\end{equation}
Therefore, even though ICEPOP masks the local deviation at $t=1$, the remaining accepted tokens $t\ge2$ are still evaluated under
mismatched context visitation distributions, which is precisely the causal mismatch that token-independent masking fails to address.
\end{proof}

\begin{proposition}[Sequence-Level IS Loses Positional Information]
\label{prop:seq-is-fails}
Let $\bar{r}$ denote the sequence-level average log-ratio,
\begin{equation}
    \bar{r} := \frac{1}{T}\sum_{t=1}^T r_t.
\end{equation}
Assume there exists $t^*\in\{1,\ldots,T\}$ such that $|r_{t^*}| = L$ and $|r_t|\le \varepsilon$ for all $t\neq t^*$. Then
\begin{equation}
    \big|\bar{r} - L/T\big| \le \frac{T-1}{T}\,\varepsilon \le \varepsilon.
    \label{eq:seq-mean-dilution}
\end{equation}
In particular, in the pure single-spike case $r_{t^*}=L$ and $r_t=0$ for $t\neq t^*$, one has $\bar{r}=L/T$, so
$\exp(\bar{r})=\exp(L/T)\to 1$ as $T\to\infty$ for any fixed $L$.
\end{proposition}

\begin{proof}
Write
\begin{equation}
    \bar{r}
    = \frac{1}{T}r_{t^*} + \frac{1}{T}\sum_{t\neq t^*}r_t.
\end{equation}
By the triangle inequality and the bound $|r_t|\le\varepsilon$ for $t\neq t^*$,
\begin{equation}
    \big|\bar{r} - r_{t^*}/T\big|
    = \frac{1}{T}\Big|\sum_{t\neq t^*}r_t\Big|
    \le \frac{T-1}{T}\,\varepsilon,
\end{equation}
which implies \eqref{eq:seq-mean-dilution}. The single-spike specialization follows by setting $r_t=0$ for $t\neq t^*$.
\end{proof}

\begin{example}[Numerical dilution at long horizons]
Consider the pure single-spike case with $T=4096$, $r_{t^*}=3$, and $r_t=0$ for $t\neq t^*$. Then
\begin{equation}
    \exp(\bar{r}) = \exp(3/4096) \approx 1.00073,
\end{equation}
which can easily fall inside practical acceptance intervals $[\lambda,\Lambda]$ even though the local deviation at $t^*$ is large in
log-ratio magnitude.
\end{example}

\subsection{Prefix Cumulative IS: Definition and Properties}
\label{sec:prefix-is}

\begin{definition}[Prefix Cumulative IS]
\label{def:prefix-is}
For a trajectory with action mask $m_t \in \{0,1\}$ and per-token log-ratios $r_t$ defined in \eqref{eq:log-ratio}, define the
cumulative sums
\begin{align}
    L_t &= \sum_{s=1}^{t} r_s \cdot m_s, \qquad P_t = \sum_{s=1}^{t} m_s, \label{eq:prefix-cumsum} \\
    \bar{r}_t &= \frac{L_t}{P_t}, \qquad \mathrm{prefix\_is}(t) = \exp(\bar{r}_t), \label{eq:prefix-is}
\end{align}
where $\bar{r}_t$ is defined only for $t$ such that $P_t\ge 1$ (i.e., the prefix contains at least one active action token). The
statistic $\mathrm{prefix\_is}(t)$ is the geometric mean of IS ratios over active positions up to $t$.
\end{definition}

\begin{definition}[Prefix IS Masking]
Given thresholds $[\lambda, \Lambda]$:
\begin{equation}
    M_t^{\mathrm{prefix}} := \mathbb{I}\!\big[\mathrm{prefix\_is}(t) \in [\lambda, \Lambda]\big].
    \label{eq:prefix-mask}
\end{equation}
\end{definition}

\begin{remark}[Action-token indexing]
\label{rem:action-indexing}
For the theoretical analysis, we may assume (without loss of generality) that we index only over active action-token positions and that the
trajectory is not padded. In this case, one can take $m_t\equiv 1$ and $P_t=t$. We retain the mask notation to match the batched,
variable-length implementation.
\end{remark}

Three structural properties distinguish prefix IS from existing methods:

\textbf{Causality.} $\mathrm{prefix\_is}(t)$ depends only on $\{r_s\}_{s \le t}$, respecting the autoregressive order.

\textbf{Stability under in-bound increments.} The following lemma formalizes a basic stability property of prefix averages.

\begin{lemma}[Stability of prefix averages]
\label{lem:prefix-stability}
Fix thresholds $0<\lambda\le 1\le \Lambda$, and define $\bar{r}_t$ as in \eqref{eq:prefix-is}. Suppose $P_t\ge 1$ and
$\bar{r}_t\in[\log\lambda,\log\Lambda]$. If the next position is active ($m_{t+1}=1$) and satisfies
$r_{t+1}\in[\log\lambda,\log\Lambda]$, then $\bar{r}_{t+1}\in[\log\lambda,\log\Lambda]$.
\end{lemma}

\begin{proof}
When $m_{t+1}=1$, we have $L_{t+1}=L_t+r_{t+1}$ and $P_{t+1}=P_t+1$. Therefore
\begin{equation}
    \bar{r}_{t+1}
    = \frac{L_{t+1}}{P_{t+1}}
    = \frac{P_t}{P_t+1}\,\bar{r}_t + \frac{1}{P_t+1}\,r_{t+1}.
\end{equation}
The coefficients are nonnegative and sum to one, so $\bar{r}_{t+1}$ is a convex combination of two numbers in
$[\log\lambda,\log\Lambda]$. Hence $\bar{r}_{t+1}\in[\log\lambda,\log\Lambda]$.
\end{proof}

\textbf{Granularity.} Different tokens in the same sequence are independently masked based on cumulative divergence up to that point.

\noindent\textbf{Notation.} To avoid overloading symbols, we reserve $(\epsilon,\delta)$ for the divergence quantities in
\Cref{sec:preliminaries}. In the deviation-pattern analysis below, we use $\varepsilon$ to denote a uniform bound on the per-token
log-ratios $r_t$.

\begin{theorem}[Causality-Aware Masking with Conditional Dominance]
\label{thm:prefix-tighter}
Let $M^{\mathrm{prefix}}_t$, $M^{\mathrm{token}}_t$, and $M^{\mathrm{seq}}_t$ denote the masks produced by prefix cumulative IS, token-level IS (ICEPOP), and sequence-level IS (seq-mask-tis) respectively, all with the same threshold $[\lambda, \Lambda]$ where $0 < \lambda \le 1 \le \Lambda$. Assume policy support overlap (\Cref{asm:support}). Then:

\begin{enumerate}
    \item[(a)] \textbf{Early deviation: prefix masks a (possibly longer) prefix.} Assume $\rho_1 \notin [\lambda, \Lambda]$ and
    $|r_t| \le \varepsilon$ for all $t>1$, where $0 \le \varepsilon \le \min(\log\Lambda, |\log\lambda|)$. Then token-level ICEPOP masks
    only position $t=1$, while prefix IS masks at least position $t=1$. Moreover, if $r_1>\log\Lambda$ then for every position $t$
    with $P_t \ge 1$ satisfying
    \begin{equation}
        P_t < \frac{r_1+\varepsilon}{\log\Lambda+\varepsilon},
        \label{eq:early-upper-sufficient}
    \end{equation}
    we have $M_t^{\mathrm{prefix}}=0$. Symmetrically, if $r_1<\log\lambda$ then for every position $t$ with $P_t \ge 1$ satisfying
    \begin{equation}
        P_t < \frac{-r_1+\varepsilon}{|\log\lambda|+\varepsilon},
        \label{eq:early-lower-sufficient}
    \end{equation}
    we have $M_t^{\mathrm{prefix}}=0$. In particular, in the early-deviation regime prefix IS masks at least as many tokens as
    token-level ICEPOP.

    \item[(b)] \textbf{Late deviation: prefix is more local than sequence masking.} Suppose there exists $t^*\in\{1,\ldots,T\}$ such that
    $|r_t| \le \varepsilon$ for all $t\neq t^*$ and $\rho_{t^*}\notin[\lambda,\Lambda]$, where $0 \le \varepsilon \le \min(\log\Lambda,
    |\log\lambda|)$. Then prefix IS can only mask tokens at positions $t \ge t^*$, so
    \begin{equation}
        \big|\{t : M_t^{\mathrm{prefix}} = 0\}\big| \le T - t^* + 1.
    \end{equation}
    In contrast, sequence-level masking is all-or-nothing: $|\{t : M_t^{\mathrm{seq}} = 0\}| \in \{0, T\}$. Moreover, in the pure
    single-spike case $r_{t^*}=L$ and $r_t=0$ for $t\neq t^*$, one has $\frac{L}{P_{t^*}} \ge \frac{L}{P_T}$, so there exist regimes
    where $M_{t^*}^{\mathrm{prefix}}=0$ while $M_t^{\mathrm{seq}}=1$ for all $t$ (sequence geometric mean diluted by large $T$).

    \item[(c)] \textbf{Monotone violation: all methods reject.} If $r_s > \log\Lambda$ for all $s$ (upper violation), or
    $r_s < \log\lambda$ for all $s$ (lower violation), then $M_t^{\mathrm{prefix}} = 0$ and $M_t^{\mathrm{seq}} = 0$ for all $t$,
    so both methods reject all $T$ tokens.
\end{enumerate}
\end{theorem}

\begin{proof}[Proof sketch]
The full proof is in \Cref{app:prefix-proof}. We highlight the key steps.

\textbf{Part (a).} The $\varepsilon$-bounded condition implies $r_t \in [\log\lambda, \log\Lambda]$ for all $t>1$, hence ICEPOP masks
only $t=1$. For prefix IS, write $L_t=r_1+\sum_{s=2}^{t}r_s m_s$ and use the bounds
\begin{align}
    \sum_{s=2}^{t}r_s m_s &\ge -(P_t-1)\varepsilon,
    \\
    \sum_{s=2}^{t}r_s m_s &\le (P_t-1)\varepsilon,
\end{align}
which yield the sufficient conditions \Cref{eq:early-upper-sufficient,eq:early-lower-sufficient}.

\textbf{Part (b).} For $t<t^*$, all active log-ratios in the prefix satisfy $r_s\in[\log\lambda,\log\Lambda]$, so repeated
application of \Cref{lem:prefix-stability} implies $\bar{r}_t\in[\log\lambda,\log\Lambda]$ and hence
$M_t^{\mathrm{prefix}}=1$. Therefore masking can only occur at positions $t\ge t^*$, yielding the stated bound. The single-spike
comparison follows from $P_{t^*}\le P_T$.

\textbf{Part (c).} If each $r_s$ is strictly beyond the same threshold, then their running average is also strictly beyond the
threshold, so both prefix and sequence masks reject all tokens.
\end{proof}

\begin{remark}[Same-sign but in-bound deviations]
When all $r_s$ share the same sign but do not all exceed the corresponding threshold, the running average $L_t/P_t$ need not be
monotone in $t$ and may return within bounds after crossing a threshold. In this regime there is no unconditional dominance ordering
between prefix masking and token-level masking: token-level methods detect large single-token outliers, while prefix masking detects
cumulative drift from many moderate deviations.
\end{remark}

\subsection{Connection to the Adaptive Bound}
\label{sec:adaptive-connection}

We relate prefix cumulative IS to the per-position, causal structure that appears in trust-region style error bounds for long-horizon
LLM-RL~\citep{li2025trm}. In our notation, these bounds analyze the policy pair $(\piroll,\pitheta)$.

\begin{lemma}[KL chain rule for the cumulative log-ratio]
\label{lem:cumsum-kl}
Under Assumptions~\ref{asm:support}--\ref{asm:finite-length}, for any $t \le T$,
\begin{equation}
    \E_{\piroll}\!\Big[\sum_{s=1}^{t} r_s\Big]
    = -\sum_{s=1}^{t} \E_{c_s \sim d_s^{\piroll}}\!\big[\DKL(\piroll(\cdot|c_s) \| \pitheta(\cdot|c_s))\big]
    = -\DKL(d_{t+1}^{\piroll} \| d_{t+1}^{\pitheta}),
    \label{eq:cumsum-kl}
\end{equation}
where the final equality is the KL chain rule applied to the joint prefix distributions.
\end{lemma}

\begin{proof}
We start from the definition $r_s = \log \pitheta(y_s\mid c_s) - \log \piroll(y_s\mid c_s)$ in \eqref{eq:log-ratio}. Conditioning on
$c_s$ and taking expectation over $y_s\sim \piroll(\cdot\mid c_s)$ gives
\begin{align}
    \E_{\piroll}\!\big[r_s \mid c_s\big]
    &= \sum_{v} \piroll(v\mid c_s)\,\log\frac{\pitheta(v\mid c_s)}{\piroll(v\mid c_s)}
    \\
    &= - \sum_{v} \piroll(v\mid c_s)\,\log\frac{\piroll(v\mid c_s)}{\pitheta(v\mid c_s)}
    \\
    &= -\DKL\!\big(\piroll(\cdot\mid c_s)\,\|\,\pitheta(\cdot\mid c_s)\big).
\end{align}
Taking expectation over $c_s\sim d_s^{\piroll}$ and summing from $s=1$ to $t$ yields the first equality in \eqref{eq:cumsum-kl}.

For the second equality, apply the KL chain rule to the joint prefix distributions under the autoregressive factorizations:
\begin{equation}
    \DKL\!\big(d_{t+1}^{\piroll}\,\|\,d_{t+1}^{\pitheta}\big)
    = \sum_{s=1}^{t}\E_{c_s \sim d_s^{\piroll}}\!\Big[\DKL\!\big(\piroll(\cdot\mid c_s)\,\|\,\pitheta(\cdot\mid c_s)\big)\Big].
\end{equation}
Combining the two displays yields \eqref{eq:cumsum-kl}.
\end{proof}

\begin{remark}[Prefix filtering as a sample-level proxy]
\label{rem:prefix-proxy}
Thresholding $\mathrm{prefix\_is}(t)$ is equivalent to a per-position, sample-level constraint on the prefix average log-ratio:
\begin{equation}
    \mathrm{prefix\_is}(t) \in [\lambda, \Lambda]
    \iff \frac{1}{P_t}\sum_{s=1}^{t}r_s m_s \in [\log\lambda, \log\Lambda].
    \label{eq:threshold-kl}
\end{equation}
In contrast, \Cref{lem:cumsum-kl} is an \emph{expectation} identity over the behavior distribution. Therefore, the threshold condition
should be interpreted as a practical, sample-level proxy for trust-region style filtering rather than a population-level guarantee
without additional assumptions. Separately, Pinsker's inequality relates the population-level KL in \eqref{eq:cumsum-kl} to a TV bound:
\begin{equation}
    \|d_{t+1}^{\pitheta} - d_{t+1}^{\piroll}\|_{\mathrm{TV}}
    \le \sqrt{\frac{\DKL(d_{t+1}^{\piroll} \| d_{t+1}^{\pitheta})}{2}}.
    \label{eq:context-shift-prefix}
\end{equation}
\end{remark}

\begin{corollary}[Per-Position Sample-Level Trust Region Constraint]
\label{cor:per-position-trust}
Under prefix IS masking with threshold $[\lambda, \Lambda]$, for all accepted tokens ($M_t^{\mathrm{prefix}} = 1$), the sample-level cumulative log-ratio satisfies:
\begin{equation}
    \frac{1}{P_t}\sum_{s=1}^{t} \big(\log \pitheta(y_s|c_s) - \log \piroll(y_s|c_s)\big) m_s \in [\log\lambda, \log\Lambda].
\end{equation}
This provides a per-position, sample-level trust region constraint analogous to the per-position structure in the Adaptive bound. The constraint operates on individual trajectories; its connection to the population-level KL is through the expectation identity in Eq.~\eqref{eq:cumsum-kl}.
\end{corollary}

\begin{proof}
Follows directly from \Cref{def:prefix-is} and Eq.~\eqref{eq:prefix-mask}: $M_t^{\mathrm{prefix}} = 1$ iff $\mathrm{prefix\_is}(t) \in [\lambda, \Lambda]$ iff $L_t/P_t \in [\log\lambda, \log\Lambda]$.
\end{proof}

\subsection{The Combined Loss}
\label{sec:combined-loss}

We define the loss as the negative surrogate objective, $\mathcal{L} = -L_{\piroll}(\pitheta)$, so that gradient descent on
$\mathcal{L}$ corresponds to maximizing the surrogate. We write the per-token PPO clipped surrogate as
\begin{equation}
    \mathrm{PPOClip}(\rho_t, \hat{A}_t)
    := \min\!\Big(
        \rho_t \hat{A}_t,\;
        \mathrm{clip}(\rho_t, 1-\epsilon_c, 1+\epsilon_c)\,\hat{A}_t
    \Big),
    \label{eq:ppo-clip}
\end{equation}
where $\epsilon_c$ is the PPO clipping threshold. The REINFORCE Pro Max loss combines both components as
\begin{equation}
    \mathcal{L}^{\mathrm{ProMax}}
    = -\sum_t M_t^{\mathrm{prefix}} \cdot \mathrm{PPOClip}(\rho_t, \hat{A}_t).
    \label{eq:promax-loss}
\end{equation}
Here $\hat{A}_t$ is the adaptively normalized advantage from \Cref{sec:adaptive-norm}, $M_t^{\mathrm{prefix}}$ is the prefix IS mask,
$\rho_t$ is the per-token importance ratio defined in \eqref{eq:ratio}.

\textbf{Gradient flow.} Gradients propagate only through $\rho_t$. The prefix mask $M_t^{\mathrm{prefix}}$ and the normalized advantage
$\hat{A}_t$ are detached from the computation graph. When $M_t^{\mathrm{prefix}} = 0$, the loss at position $t$ is zero and contributes
no gradient.



%==============================================================================
% UNIFIED FRAMEWORK
%==============================================================================
\section{The Unified Framework}
\label{sec:unified}

\subsection{Why the Two Components Are Complementary}

The error decomposition (\Cref{eq:error-decomp}) has two factors at each position $t$:
\begin{equation}
    |\mathrm{Error}| \le \sum_{t=1}^T \underbrace{2\|g_t\|_\infty}_{\text{advantage factor}} \cdot \underbrace{\|d_t^{\pitheta} - d_t^{\piroll}\|_{\mathrm{TV}}}_{\text{context shift factor}}.
\end{equation}

\textbf{Component 1 (REINFORCE Max)} controls the advantage factor. The leave-one-out baseline provides statistical independence between each sample and its baseline (\Cref{prop:rloo-variance}), and the adaptive normalization ensures $\Var[\hat{A}] = 1$, bounding the effective advantage magnitude and preventing gradient explosion from outlier advantages.

\textbf{Component 2 (REINFORCE Pro)} controls the context shift factor. By masking tokens where the prefix cumulative IS exceeds the threshold, we enforce that the cumulative divergence up to each position is bounded on retained samples. By \Cref{thm:prefix-adaptive}, this helps control the context distribution shift $\|d_t^{\piold} - d_t^{\piroll}\|_{\mathrm{TV}}$ via the KL chain rule and Pinsker's inequality, acting as a sample-level proxy for the population-level constraint.

Together, both factors are addressed simultaneously, intuitively yielding a tighter effective bound than addressing either factor alone. This is an intuitive summary rather than a formal inequality: the joint benefit arises because both terms in the error decomposition (\Cref{eq:error-decomp}) are simultaneously reduced---the advantage magnitude by normalization, and the context shift by sample-level filtering.

\begin{remark}[Variance Reduction]
\label{rem:variance-decomp}
Intuitively, the two components of REINFORCE Pro Max reduce gradient variance through complementary mechanisms. Component~1 (leave-one-out baseline + adaptive normalization) reduces the variance of the advantage estimates themselves (\Cref{prop:rloo-variance}, \Cref{prop:alpha-beta}). Component~2 (prefix IS masking) attenuates high-variance gradient contributions from off-policy tokens (\Cref{thm:prefix-tighter}), where the importance weights would otherwise amplify noise.
\end{remark}

\subsection{Algorithm}

\begin{algorithm}[t]
\caption{REINFORCE Pro Max (minimizing $\mathcal{L} = -L_{\piroll}(\pitheta)$)}
\label{alg:promax}
\begin{algorithmic}[1]
    \REQUIRE Prompt $q$; $n$ samples; thresholds $[\lambda, \Lambda]$
    \STATE \textbf{Rollout:} Generate $\{o_1, \ldots, o_n\} \sim \piroll(\cdot|q)$ using vLLM; store $\log \piroll(y_t|c_t)$
    \STATE \textbf{Reward:} Compute rewards $\{r_1, \ldots, r_n\}$
    \STATE \textbf{Baseline:} $\tilde{r}_i \leftarrow r_i - \frac{1}{n-1}\sum_{j \neq i} r_j$ \hfill \COMMENT{leave-one-out}
    \STATE \textbf{Token Expansion:} $A_{i,t} \leftarrow \tilde{r}_i$ \hfill \COMMENT{sparse reward, $\gamma\!=\!1$}
    \STATE \textbf{Adaptive Normalization:} Compute $\alpha, \beta$ via Eq.~\eqref{eq:alpha-beta}; $\hat{A}_{i,t} \leftarrow \alpha A_{i,t}$ or $\beta A_{i,t}$
    \STATE \hspace{2.6em} (If $\mathcal{P} = \emptyset$ or $\mathcal{N} = \emptyset$: skip normalization)
    \STATE \textbf{Forward Pass:} Compute $\log \pitheta(y_t|c_t)$ and $\log \piold(y_t|c_t)$
    \STATE \textbf{Prefix IS:} $\ell_t \leftarrow \log \piold(y_t|c_t) - \log \piroll(y_t|c_t)$
    \STATE \hspace{2.6em} $\mathrm{prefix\_is}(t) \leftarrow \exp\!\big(\sum_{s \le t} \ell_s m_s \,/\, \sum_{s \le t} m_s\big)$
    \STATE \hspace{2.6em} $M_t \leftarrow \mathbb{I}[\mathrm{prefix\_is}(t) \in [\lambda, \Lambda]]$
    \STATE \textbf{Loss:} $\mathcal{L} \leftarrow -\sum_t M_t \cdot \exp(\ell_t) \cdot \mathrm{PPOClip}(\rho_t^{\mathrm{PPO}}, \hat{A}_t)$ \hfill \COMMENT{negative surrogate}
    \STATE \textbf{Update:} $\theta \leftarrow \theta - \alpha_{\mathrm{lr}} \nabla_\theta \mathcal{L}$
\end{algorithmic}
\end{algorithm}

\subsection{Comparison with Existing Methods}

\begin{table}[h]
\centering
\caption{Comparison of critic-free LLM-RL methods.}
\label{tab:comparison}
\renewcommand{\arraystretch}{1.2}
\begin{tabular}{lcccc}
\toprule
\textbf{Method} & \textbf{Baseline} & \textbf{Normalization} & \textbf{IS Correction} & \textbf{Causal IS} \\
\midrule
REINFORCE++ & Mean & Global $\mu/\sigma$ & None & --- \\
GRPO & Mean & Global $\mu/\sigma$ & None & --- \\
RLOO & Leave-one-out & None & None & --- \\
DAPO\footnote{DAPO: Decoupled Alignment via Policy Optimization~\citep{yu2025dapo}.} + TIS & Mean & Global $(\mu,\sigma)$ & Token clamp & No \\
DAPO + ICEPOP & Mean & Global $(\mu,\sigma)$ & Token mask & No \\
\midrule
\textbf{Pro Max} & \textbf{RLOO} & \textbf{Adaptive $\alpha/\beta$} & \textbf{Prefix cumulative} & \textbf{Yes} \\
\bottomrule
\end{tabular}
\end{table}



%==============================================================================
% RELATED WORK
%==============================================================================
\section{Related Work}
\label{sec:related}

\subsection{Critic-Free RL for LLMs}

The high memory cost of maintaining a critic network has motivated a family of critic-free reinforcement learning methods for large language models. REINFORCE++ Baseline~\citep{hu2025reinforce} uses the batch mean reward as a baseline and normalizes advantages globally, providing a simple and memory-efficient alternative to PPO. GRPO~\citep{shao2024deepseekmath} extends this idea by normalizing advantages with both the mean and standard deviation within each prompt group, achieving strong results on mathematical reasoning tasks. RLOO~\citep{ahmadian2024back} revisits the classical leave-one-out control variate, computing each sample's baseline from the remaining samples in the group, which eliminates the correlation between a sample and its own baseline. All three methods avoid the need for a value network, saving approximately a quarter of GPU memory compared to standard PPO~\citep{schulman2017proximal} in comparable settings\footnote{The 25\% memory saving is reported for single-model-size configurations where the critic has the same architecture as the actor; the actual saving depends on model size, parallel strategy, and batch configuration.}.

\subsection{Off-Policy Correction in LLM-RL}

Modern LLM-RL pipelines decouple inference from training: high-throughput engines such as vLLM~\citep{kwon2023efficient} generate rollouts, while gradient updates are performed in separate training frameworks. This architectural separation introduces a policy mismatch between the rollout engine and the training actor~\citep{yao2025offpolicy}.
Recent empirical analyses further argue that the trust region must be defined with respect to the \emph{original} rollout distribution (the
behavior policy that generated the data) rather than a recomputed anchor evaluated inside the training stack; anchoring to a recomputed
on-policy distribution can lead to instability under training--inference mismatch~\citep{qi2026rethinkingtrustregionllm}.

Several importance sampling correction methods have been proposed to address this mismatch. Truncated Importance Sampling (TIS)~\citep{yao2025offpolicy} clamps the per-token importance weight to a fixed interval, preventing extreme corrections. Its limitation is that it ignores the causal structure of autoregressive generation: an early deviation affects all future tokens, but TIS treats each token independently. IcePop~\citep{zhang2025icepop} takes a harder approach: it zeroes out tokens whose importance weight falls outside the acceptable range. Like TIS, it operates at the token level and does not account for how deviations propagate through the sequence.

At the sequence level, Trust Region Masking (TRM)~\citep{li2025trm} discards entire trajectories that violate a trust region criterion defined
by the maximum token-level divergence (e.g., $\max_t \Dkltok(c_t;\piroll,\pitheta)$). GSPO~\citep{gspo2025} instead computes a single
sequence-level importance ratio coefficient (the geometric-mean ratio $\rhoseq$) and applies it uniformly across the sequence.
\par
\noindent\textbf{Prefix-level refinement.} Our REINFORCE Pro component can be viewed as a prefix-level refinement of TRM's rejection
mechanism: instead of a single sequence gate $M^{\mathrm{TRM}}(x,y)$, we compute a per-position mask $M_t^{\mathrm{pre}}$ from a ratio-based prefix
statistic and multiply it into the token-level loss, masking only the positions whose prefixes violate a proxy trust region. This retains
per-position granularity while preserving the causal order of autoregressive generation.

These methods share a common limitation: token-level methods ignore that an early deviation affects the entire future trajectory, while
sequence-level methods either reject the entire sequence or lose the ability to distinguish where within a sequence the divergence occurs.

\subsection{Trust Region Methods}

Trust region methods constrain the policy update to remain close to the current policy, ensuring stable optimization. TRPO~\citep{schulman2015trust} enforces a hard KL divergence constraint, while PPO~\citep{schulman2017proximal} relaxes this to a clipped surrogate objective that is simpler to implement. Recent work on Trust Region Masking~\citep{li2025trm} derives tighter error bounds for long-horizon LLM reinforcement learning, showing that classical $O(T^2)$ trust region bounds become theoretically vacuous for sequences of thousands of tokens. Their Adaptive bound achieves per-position granularity and preserves the causal structure of error accumulation, providing the theoretical foundation that motivates our prefix-level Causal Trust Region masking. Our work borrows the per-position error decomposition perspective from TRM but does not directly reuse its optimization objective.

\subsection{Summary of Differences}

Compared with existing critic-free methods, REINFORCE Pro Max differs along four dimensions: (1)~\emph{baseline}: RLOO (independent of each sample) vs.\ mean (correlated); (2)~\emph{normalization}: adaptive asymmetric $\alpha/\beta$ vs.\ global $\mu/\sigma$; (3)~\emph{IS granularity}: prefix cumulative (per-position) vs.\ token-level or sequence-level; (4)~\emph{causal structure}: prefix masking respects autoregressive ordering, while the ratio-based stabilization methods considered in this paper do not.



%==============================================================================
% EXPERIMENTS
%==============================================================================
\section{Experiments}
\label{sec:experiments}

This section provides controlled simulations that validate two core design claims of REINFORCE Pro Max: (i)~the leave-one-out baseline and adaptive asymmetric normalization stabilize advantage statistics without flipping advantage sign; and (ii)~prefix-level masking behaves as predicted under the deviation patterns analyzed in \Cref{thm:prefix-tighter}. Large-scale LLM benchmark results (e.g., AIME/MATH with a specific backbone) are orthogonal to these structural properties and are left as future work.

\subsection{Experimental Setup}

We simulate prompt groups with $n=8$ samples. Rewards are drawn from a skewed mixture that mimics sparse-success reasoning settings: with probability $p=0.15$ a sample is drawn from a ``success'' cluster centered near $1$, otherwise from a ``failure'' cluster centered near $0$, both with Gaussian noise $\sigma_{\mathrm{rew}}=0.10$. To reflect token-level weighting effects after expansion (\Cref{sec:token-expand}), we assign longer sequences to lower-quality samples (failure trajectories), setting lengths to
\[
    T_i = \begin{cases}
        256 + u_i, & \text{success sample},\\
        512 + u_i, & \text{failure sample},
    \end{cases}
    \qquad u_i \sim \mathrm{Unif}\{-32,\ldots,32\}.
\]

For IS masking, we use the acceptance interval $[1-\epslow,1+\epshigh]=[0.9,1.1]$ unless otherwise stated (i.e., $\epslow=\epshigh=0.1$).
We also define the corresponding log-thresholds
\[
    \tau_+ := \log(1+\epshigh),
    \qquad
    \tau_- := -\log(1-\epslow),
\]
so that $\rho_t \in [1-\epslow,1+\epshigh]$ is equivalent to $\log\rho_t \in [-\tau_-,\tau_+]$.
Finally, we construct synthetic log-ratio patterns with $T=32$ to directly instantiate the three regimes in \Cref{thm:prefix-tighter}.

\subsection{Baselines}

\textbf{Baselines and normalization.} We compare the mean baseline (REINFORCE++/GRPO-style) to the leave-one-out (RLOO) baseline (\Cref{def:rloo}), and we compare global normalization $(A-\mu)/\sigma$ to adaptive asymmetric normalization (\Cref{def:adaptive-norm}).

\textbf{IS masking.} We compare token-level masking (IcePop), sequence-level masking (GSPO-style, based on $\rhoseq$), and prefix-level
masking based on the prefix proxy in \Cref{def:prefix-is}.

\subsection{Metrics}

\textbf{Baseline self-correlation.} We measure $\mathrm{corr}(r_i, b_i)$ between each sample reward $r_i$ and its baseline $b_i$. A
correlated baseline can couple a sample with its own control variate, complicating the clean second-moment analysis enabled by RLOO (see
\Cref{app:rloo-proof}).

\textbf{Sign flip rate under normalization.} After token expansion, we compute the token-weighted fraction of non-zero tokens whose normalized advantage changes sign, i.e., $\mathbb{I}[\mathrm{sign}(\hat{A}_{i,t}) \neq \mathrm{sign}(A_{i,t})]$. Sign flips are undesirable because they can turn punishment into reinforcement (or vice versa).

\textbf{Masked-token coverage.} For each synthetic log-ratio pattern, we report the number of masked tokens and the first/last masked
positions for IcePop, sequence-level IS masking, and prefix masking.

\subsection{Main Results}

\begin{table}[ht]
\centering
\caption{Controlled simulation of baseline correlation and sign flips under token weighting. Numbers are estimated over $20{,}000$ prompt groups with $n=8$.}
\renewcommand{\arraystretch}{1.2}
\begin{tabular}{lcc}
\toprule
\textbf{Metric} & \textbf{Mean / global} & \textbf{RLOO / adaptive} \\
\midrule
$\mathrm{corr}(r_i, b_i)$ & $0.350$ & $-0.008$ \\
Sign flip rate (token-weighted) & $6.6\%$ & $0.0\%$ \\
\bottomrule
\end{tabular}
\end{table}

\noindent The mean baseline exhibits substantial self-correlation, while RLOO makes the baseline effectively independent of the current sample. Under token-length weighting, global normalization introduces non-negligible sign flips, whereas adaptive normalization preserves sign by construction (\Cref{par:adv-sign-preservation}).

\begin{table}[ht]
\centering
\caption{Masking behavior under synthetic deviation patterns. We use $T=32$ and $[1-\epslow,1+\epshigh]=[0.9,1.1]$. We report the number of masked tokens and the masked position range.}
\renewcommand{\arraystretch}{1.2}
\begin{tabular}{lccc}
\toprule
\textbf{Pattern} & \textbf{IcePop (token)} & \textbf{Seq (GSPO-style)} & \textbf{Prefix (ours)} \\
\midrule
Early ($\log\rho_1{=}8\tau_+$; else $0$) & $1$ ($[1,1]$) & $0$ (---) & $7$ ($[1,7]$) \\
Late ($\log\rho_{24}{=}32\tau_+$; else $0$) & $1$ ($[24,24]$) & $0$ (---) & $8$ ($[24,31]$) \\
Drift ($\log\rho_t{=}2\tau_+,\ \forall t$) & $32$ ($[1,32]$) & $32$ ($[1,32]$) & $32$ ($[1,32]$) \\
\bottomrule
\end{tabular}
\end{table}

\noindent The early and late patterns instantiate \Cref{thm:prefix-tighter}(a,b) in the simplest form with $\varepsilon=0$ (all non-spike
tokens satisfy $\rho_t=1$ and hence lie within the trust-region interval).
\par
\noindent\textbf{Early deviation.} IcePop reacts only to the local outlier at $t=1$ and therefore masks a single token. Sequence-level
masking aggregates over the full horizon and dilutes the violation. Prefix masking, in contrast, propagates the early mismatch forward by
masking a short prefix:
\[
    \log\rhoprefix{t} = \frac{1}{t}\log\rho_1 = \frac{8}{t}\tau_+,
    \qquad
    \log\rhoseq = \frac{1}{T}\log\rho_1 = \frac{1}{4}\tau_+,
    \qquad (T=32).
\]
Thus $\log\rhoprefix{t}>\tau_+$ for $t\in\{1,\ldots,7\}$, while $\log\rhoseq$ remains in-bound.
\par
\noindent\textbf{Late deviation.} For a single spike at $t^*=24$, prefix masking is strictly suffix-local (it can only start at $t\ge t^*$),
while sequence-level masking remains all-or-nothing. With $\log\rho_{24}=32\tau_+$ and $\log\rho_t=0$ otherwise, the sequence statistic
satisfies $\log\rhoseq=\tau_+$ and is accepted, but the prefix average exceeds $\tau_+$ for $t\in\{24,\ldots,31\}$, yielding a localized
masked suffix. These behaviors align with the regimes in \Cref{thm:prefix-tighter}.

\subsection{Ablation Studies}

\textbf{Without adaptive normalization.} Global normalization can flip the sign of token-expanded advantages when token-length weighting makes the token-level mean non-zero, while adaptive normalization preserves sign by using separate positive/negative scalings (\Cref{par:adv-sign-preservation}).

\textbf{Without prefix masking.} Token-level masking only reacts to per-token outliers and may leave future tokens unfiltered even when the context has already drifted; prefix masking filters based on the prefix statistic, which is aligned with the causal accumulation structure (\Cref{lem:coupling}, \Cref{thm:prefix-tighter}).

\subsection{Sensitivity to Thresholds}

Masking behavior depends on $[1-\epslow,1+\epshigh]$. Stricter thresholds can cause sequence-level IS masking to reject an entire trajectory,
while prefix masking remains localized in position. For example, in the late-deviation pattern above, the sequence statistic satisfies
$\rhoseq=1+\epshigh=1.1$ under the default thresholds. If we instead use the stricter symmetric interval $[1-\epslow,1+\epshigh]=[0.95,1.05]$
(i.e., $\epslow=\epshigh=0.05$), then $\rhoseq>1.05$ and the sequence-level mask rejects all $32$ tokens, while prefix masking still rejects only
suffix positions starting at $t=24$.

\subsection{Reproducibility Checklist}

\begin{itemize}
    \item Random seed: $0$.
    \item Monte Carlo trials: $20{,}000$ prompt groups; $n=8$ samples per group.
    \item Reward distribution: mixture with $p=0.15$ and Gaussian noise $\sigma_{\mathrm{rew}}=0.10$.
    \item Token lengths: success $T_i = 256 + u_i$, failure $T_i = 512 + u_i$, $u_i \sim \mathrm{Unif}\{-32,\ldots,32\}$.
    \item Thresholds: default $(\epslow,\epshigh)=(0.1,0.1)$ (i.e., $[1-\epslow,1+\epshigh]=[0.9,1.1]$); sensitivity example $(\epslow,\epshigh)=(0.05,0.05)$ (i.e., $[0.95,1.05]$).
    \item Code version: git commit \texttt{9a3879e}.
\end{itemize}



%==============================================================================
% CONCLUSION
%==============================================================================
\section{Conclusion}
\label{sec:conclusion}

We presented REINFORCE Pro Max, a unified framework for critic-free LLM reinforcement learning that addresses two fundamental challenges:
high-variance advantage estimation and causal off-policy mismatch between behavior and training policies.

The first component, REINFORCE Max, combines a leave-one-out baseline (statistically independent of each sample, enabling clean
second-moment analysis, \Cref{prop:rloo-variance}) with adaptive asymmetric normalization (preserving advantage sign while enforcing the
empirical constraints in \Cref{def:adaptive-norm}, \Cref{prop:alpha-beta}). The second component, REINFORCE Pro, introduces
prefix-cumulative ratio masking (prefix IS) that preserves the causal structure of autoregressive generation---we showed it can provide
tighter masking than both token-level and sequence-level alternatives in the analyzed deviation patterns (\Cref{thm:prefix-tighter}),
subject to the conditions stated therein. The connection to the Adaptive bound from the trust region framework highlights that prefix IS
acts as a sample-level proxy for per-position trust region filtering: the KL chain rule yields an expectation identity for the cumulative
log-ratio (\Cref{lem:cumsum-kl}), while the prefix thresholding rule enforces a per-position sample constraint
(\Cref{rem:prefix-proxy}).

\textbf{Limitations and future work.} The prefix IS threshold $[\lambda, \Lambda]$ is a hyperparameter that requires tuning. The sample-level prefix IS constraint is a proxy for the population-level KL bound, and the gap between the two is not formally characterized. The theoretical analysis assumes bounded rewards and overlapping policy supports (\Cref{sec:assumptions}), which may not hold in all practical settings. Empirical validation on large-scale benchmarks is ongoing; we plan to evaluate on mathematical reasoning tasks (AIME, MATH) with ablations on each component.



%==============================================================================
% REFERENCES
%==============================================================================
\bibliographystyle{plainnat}
\bibliography{main/reference}


%%%%%%%%%%%%%%%%%%%%%%%%%%%%%%%%%%%%%%%%%%%%%%%%%%%%%%%%%%%%%%%%%%%%%%%%%%%%%%%
% APPENDIX
%%%%%%%%%%%%%%%%%%%%%%%%%%%%%%%%%%%%%%%%%%%%%%%%%%%%%%%%%%%%%%%%%%%%%%%%%%%%%%%
\newpage
\appendix

\section{Proof of Leave-One-Out Baseline Properties}
\label{app:rloo-proof}

We provide a detailed property analysis of the leave-one-out and mean-baseline gradient estimators.

\begin{proposition}[Properties of the Leave-One-Out Baseline]
\label{prop:rloo-variance}
Let $f_i = \nabla_\theta \log P^{\pitheta}(o_i | q)$ denote the score function for sample $i$, and let $g_i(b) = (r_i - b) f_i$ denote the per-sample policy gradient contribution with baseline $b$. The leave-one-out gradient estimator $\hat{g} = \frac{1}{n}\sum_i g_i(b_i)$ satisfies:
\begin{enumerate}
    \item \textbf{Unbiasedness:} $\E[\hat{g}] = \nabla_\theta J(\pitheta)$.
    \item \textbf{Zero cross-covariance:} For each $i$, $\Cov\!\big(f_i,\; b_i\big) = 0$.
    \item \textbf{Second-moment factorization:} The independence between $b_i$ and $(r_i, f_i)$ yields the factorization $\E[(r_i - b_i)^2 f_i f_i^\top] = \E[(r_i - b_i)^2]\,\E[f_i f_i^\top]$, which does not hold for the mean baseline $b^{\mathrm{mean}}$ due to its dependence on $r_i$.
\end{enumerate}
\end{proposition}

\begin{remark}[Relationship between the two estimators]
\label{rem:rloo-mean-scaling}
Note that $r_i - b^{\mathrm{mean}} = \frac{n-1}{n}(r_i - b_i)$, so $\hat{g}^{\mathrm{mean}} = \frac{n-1}{n}\hat{g}$. The two estimators are deterministically proportional, and the mean-baseline estimator has smaller raw variance by a factor of $((n-1)/n)^2$. However, this scaling also shrinks the signal by $\frac{n-1}{n}$, so the variance reduction comes at the cost of a smaller effective step size. The practical advantage of the leave-one-out baseline lies in its independence structure (Parts~2--3), which simplifies variance analysis, enables clean factorization of second moments, and avoids the correlated baseline artifacts that complicate optimization with the mean baseline.
\end{remark}

\begin{remark}
The leave-one-out baseline is equivalent to the leave-one-out control variate from the classical REINFORCE literature~\citep{williams1992simple}. In the LLM-RL setting with $n$ samples per prompt, each sample's baseline is independent of its own trajectory, making it particularly well-suited for group-based sampling strategies.
\end{remark}

\textbf{Proof of \Cref{prop:rloo-variance}.}

Consider $n$ i.i.d.\ samples $\{o_1, \ldots, o_n\}$ from $\pitheta(\cdot|q)$ with rewards $\{r_1, \ldots, r_n\}$. Define $f_i = \nabla_\theta \log P^{\pitheta}(o_i|q)$ (the score function for sample $i$), $\mu = \E[r_i]$, and $\sigma^2 = \Var[r_i]$.

\textbf{Leave-one-out estimator:}
\begin{equation}
    \hat{g} = \frac{1}{n}\sum_{i=1}^n (r_i - b_i) f_i, \qquad b_i = \frac{1}{n-1}\sum_{j \neq i} r_j.
\end{equation}

\textbf{Mean baseline estimator:}
\begin{equation}
    \hat{g}^{\mathrm{mean}} = \frac{1}{n}\sum_{i=1}^n (r_i - b^{\mathrm{mean}}) f_i, \qquad b^{\mathrm{mean}} = \frac{1}{n}\sum_{j=1}^n r_j.
\end{equation}

\textbf{Step 1: Deterministic scaling relationship.}

Note that $b^{\mathrm{mean}} = \frac{r_i}{n} + \frac{n-1}{n}b_i$, so:
\begin{equation}
    r_i - b^{\mathrm{mean}} = \frac{n-1}{n}(r_i - b_i).
\end{equation}
Summing over $i$: $\hat{g}^{\mathrm{mean}} = \frac{n-1}{n}\hat{g}$. Therefore $\Var[\hat{g}^{\mathrm{mean}}] = \big(\frac{n-1}{n}\big)^2 \Var[\hat{g}]$. The mean-baseline estimator has smaller raw variance, but this is entirely due to the $\frac{n-1}{n}$ signal shrinkage---it also shrinks the expected gradient by the same factor.

\textbf{Step 2: Independence and second-moment factorization.}

Since $b_i$ is independent of $(r_i, f_i)$, the per-sample second moment factorizes:
\begin{equation}
    \E[(r_i - b_i)^2 f_i f_i^\top] = \E[(r_i - b_i)^2]\, \E[f_i f_i^\top].
\end{equation}
We have $\E[(r_i - b_i)^2] = \Var[r_i] + \Var[b_i] = \sigma^2 + \sigma^2/(n-1) = \sigma^2 n/(n-1)$, where the cross term vanishes by independence. This clean factorization is the key structural advantage of the leave-one-out baseline: it separates the reward variance from the score function variance, making the gradient variance analytically tractable.

\textbf{Step 3: Non-factorization of the mean baseline.}

For the mean baseline, $b^{\mathrm{mean}}$ depends on $r_i$, so:
\begin{equation}
    \E[(r_i - b^{\mathrm{mean}})^2 f_i f_i^\top] \neq \E[(r_i - b^{\mathrm{mean}})^2]\, \E[f_i f_i^\top]
\end{equation}
in general. The dependence between $b^{\mathrm{mean}}$ and $(r_i, f_i)$ creates cross terms involving $\Cov[f_i, r_i] = \E[r_i f_i] = \nabla_\theta J(\pitheta)$ (by the score function identity, since $\E[f_i] = 0$). This non-zero covariance complicates the variance analysis and means the mean baseline does not achieve the clean separation of reward and score function contributions.

\textbf{Step 4: Practical implications.}

The leave-one-out baseline's independence structure provides three practical advantages over the mean baseline:
\begin{enumerate}
    \item \textbf{Analytical tractability:} The factorization in Step~2 enables closed-form variance expressions, facilitating hyperparameter tuning (e.g., choosing $n$).
    \item \textbf{No correlated baseline artifacts:} With the mean baseline, $\Cov[f_i, b^{\mathrm{mean}}] = \frac{1}{n}\nabla J \neq 0$, meaning the baseline itself correlates with the gradient direction. This correlation may interact with adaptive optimizers (e.g., Adam) in unpredictable ways, though the practical impact depends on the specific optimization dynamics. The leave-one-out baseline eliminates this correlation entirely.
    \item \textbf{Equivalent signal-to-noise ratio:} Although $\hat{g}^{\mathrm{mean}}$ has smaller raw variance, its signal is also smaller by $\frac{n-1}{n}$. The signal-to-noise ratio $\|\E[\hat{g}]\|^2 / \mathrm{tr}(\Var[\hat{g}])$ is identical for both estimators, since both the signal and noise scale by the same factor.
\end{enumerate}


\section{Optimality of $\gamma = 1.0$}
\label{app:gamma-one}

\begin{proposition}[Optimality of $\gamma = 1.0$]
\label{prop:gamma-one}
When the extrinsic reward is sparse (assigned only at the final token), $\gamma = 1.0$ is the unique discount factor that preserves unbiasedness of the REINFORCE gradient estimator for the undiscounted objective $J(\pitheta) = \E[R(x,y)]$.
\end{proposition}

\begin{proof}
With discount factor $\gamma < 1$, the gradient estimate becomes $\E\!\big[\sum_t \gamma^{T-t} \tilde{r}_i \nabla_\theta \log \pitheta(y_t|c_t)\big]$, which is a biased estimator of $\nabla J(\pitheta) = \E\!\big[\tilde{r}_i \sum_t \nabla_\theta \log \pitheta(y_t|c_t)\big]$ since $\gamma^{T-t} \neq 1$ for $t < T$. Only $\gamma = 1$ makes all coefficients equal to 1, recovering the unbiased estimator.
\end{proof}


\section{Illustrative Examples for Theorem~\ref{thm:prefix-tighter}}
\label{app:prefix-proof}

The following examples illustrate the behavior of prefix IS under the specific deviation patterns analyzed in \Cref{thm:prefix-tighter}. They demonstrate the theorem's claims on concrete instances but do not constitute a general proof covering all possible deviation patterns.

\textbf{Detailed analysis of Part (a).}

Let $w_1 = e^{\ell_1}$ with $|\ell_1| > \log\Lambda$ (so $w_1 \notin [\lambda, \Lambda]$), and $w_t = 1$ (i.e., $\ell_t = 0$) for all $t > 1$.

Under ICEPOP: $M_1^{\mathrm{token}} = 0$, $M_t^{\mathrm{token}} = 1$ for $t > 1$. Total masked: 1 token.

Under prefix IS: $\mathrm{prefix\_is}(t) = \exp(\ell_1 / t)$ for $t \ge 1$ (assuming all $m_t = 1$). The mask is:
\begin{equation}
    M_t^{\mathrm{prefix}} = \mathbb{I}\!\big[\exp(\ell_1/t) \in [\lambda, \Lambda]\big] = \mathbb{I}\!\big[\ell_1/t \in [\log\lambda, \log\Lambda]\big].
\end{equation}
For $\ell_1 > 0$: $M_t^{\mathrm{prefix}} = 0$ when $\ell_1/t > \log\Lambda$, i.e., $t < \ell_1/\log\Lambda$. So the first $\lfloor \ell_1/\log\Lambda \rfloor$ tokens are masked.

For example, with $\ell_1 = 3$ and $\Lambda = 5$ ($\log\Lambda \approx 1.61$): prefix IS masks the first $\lfloor 3/1.61 \rfloor = 1$ token. With $\Lambda = 2$ ($\log\Lambda \approx 0.69$): prefix IS masks the first $\lfloor 3/0.69 \rfloor = 4$ tokens. In both cases, prefix IS masks at least as many tokens as ICEPOP.

\textbf{Detailed analysis of Part (b).}

Let $\ell_t = 0$ for $t < t^*$ and $\ell_{t^*} = L$ with $|L| > \log\Lambda$. For $t \ge t^*$:
\begin{equation}
    \mathrm{prefix\_is}(t) = \exp\!\Big(\frac{L}{t}\Big).
\end{equation}
This is masked when $|L|/t > \log\Lambda$, i.e., $t < |L|/\log\Lambda$. If $t^* > |L|/\log\Lambda$ (late deviation), then $\mathrm{prefix\_is}(t^*) = \exp(L/t^*)$ may already be within bounds, and no tokens are masked. In contrast, seq-mask-tis computes $\exp(L/T)$; if this crosses the threshold, all $T$ tokens are rejected.


\section{Numerical Stability}
\label{app:numerical}

The adaptive normalization (\Cref{sec:adaptive-norm}) includes several safeguards:

\begin{enumerate}
    \item \textbf{Sum threshold:} If $|S^+| < \epsilon_{\mathrm{num}}$ or $|S^-| < \epsilon_{\mathrm{num}}$ (default $\epsilon_{\mathrm{num}} = 10^{-8}$), normalization is skipped and the original advantages are returned.

    \item \textbf{Scale clamping:} $\alpha$ and $\beta$ are clamped to $[\epsilon_{\mathrm{num}}, s_{\max}]$ (default $s_{\max} = 10$) to prevent extreme scaling.

    \item \textbf{Intermediate overflow protection:} The quantity $(S^+/S^-)^2 Q^-$ is clamped to $10^8$ before computing $\alpha$.

    \item \textbf{Finiteness check:} If $\alpha$ or $\beta$ is NaN or $\pm\infty$ after computation, normalization is skipped.
\end{enumerate}

These safeguards may cause the normalized advantages to deviate slightly from the exact $\E[\hat{A}] = 0$, $\Var[\hat{A}] = 1$ constraints, but in practice the deviation is negligible ($|\text{mean}| < 0.001$, $|\text{var} - 1| < 0.01$). These thresholds are heuristic defaults; formal validation is deferred to the experimental evaluation.

\textbf{Fallback conditions.} Normalization is skipped entirely when:
\begin{itemize}
    \item All advantages have the same sign ($\mathcal{P} = \emptyset$ or $\mathcal{N} = \emptyset$), indicating the group has uniform quality and the leave-one-out-shaped advantages are already meaningful.
    \item The uniform scale mode (\Cref{app:uniform-scale}) is active and the group has uniform rewards.
\end{itemize}


\section{Uniform Scale Mode}
\label{app:uniform-scale}

The uniform scale mode is an optional technique to improve sample utilization. When all $n$ samples for a prompt have identical rewards (all correct or all incorrect), the leave-one-out baseline yields $\tilde{r}_i = 0$ for all $i$, eliminating the gradient signal. Uniform scale mode recovers this signal.

\begin{definition}[Uniform Scale]
For prompt groups where $\mathrm{std}(\{r_1, \ldots, r_n\}) < \epsilon_{\mathrm{tol}}$ (uniform-reward groups), the shaped reward is set to $\tilde{r}_i = r_i / n$, and the adaptive normalization (\Cref{sec:adaptive-norm}) is skipped.
\end{definition}

\begin{proposition}
Under uniform scale mode, the gradient for uniform-reward groups is proportional to $r_i \sum_t \nabla_\theta \log \pitheta(y_{i,t}|c_{i,t})$, which pushes the policy toward all sampled responses when $r_i > 0$ (all correct) and away from them when $r_i < 0$ (all incorrect), preserving gradient signal that the leave-one-out baseline would otherwise eliminate.
\end{proposition}

\begin{proof}
With $\tilde{r}_i = r_i/n$ and no normalization, the gradient contribution of sample $i$ is $(r_i/n) \sum_t \nabla_\theta \log \pitheta(y_{i,t}|c_{i,t})$. The sign of $r_i$ determines the direction: positive $r_i$ increases the log-probability of all tokens, negative $r_i$ decreases it.
\end{proof}



\end{document}
